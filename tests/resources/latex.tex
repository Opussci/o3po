%%%%%%%%%%%%%%%%%%%%%%%%%%%%%%%%%%%%%%%%%%%%%%%%%%%%%%%
%                                                     %
% Dissertation                                        %
%                                                     %
% Christian Gogolin 2012-2014                         %
%                                                     %
%%%%%%%%%%%%%%%%%%%%%%%%%%%%%%%%%%%%%%%%%%%%%%%%%%%%%%%

\documentclass[a4paper,12pt,listof=totoc,index=totoc,bibliography=totoc,headsepline=false,headings=normal,BCOR16.153846mm,DIV12,headinclude,twoside,cleardoublepage=empty,numbers=noenddot,final]{scrreprt}

% Setup fonts and headings
\usepackage[automark]{scrpage2}
% \setkomafont{pagehead}{\normalcolor\sffamily\small}
% \setkomafont{pagefoot}{\normalcolor\sffamily\small}
\setkomafont{pageheadfoot}{\normalcolor\sffamily\small}
\setkomafont{pagenumber}{\normalcolor\sffamily\small}
\setkomafont{captionlabel}{\bfseries}
\setkomafont{caption}{\small}
\pagestyle{useheadings}
\setlength{\parskip}{10pt}

\def\?#1{\if.#1{}\else#1\fi}
\def\_#1#2{#2foo#1}

\usepackage[isbn=false,maxbibnames=999,style=alphabetic,maxcitenames=5,firstinits=true,sortcites]{biblatex}
%\usepackage[isbn=false,maxbibnames=999,style=numeric-comp,maxcitenames=5,firstinits=true]{biblatex}
\newcommand{\refsub}[2]{\hyperref[#1]{\ref*{#1}#2}}
\renewcommand*{\multicitedelim}{\addcomma\space} % use comma as separator between different entries in a \cite
%\DeclareFieldFormat{url}{\iffieldundef{doi}{\iffieldundef{eprint}{\mkbibacro{URL}\addcolon\space\url{#1}}{foo}}{bar}} %Print url only if neither doi nor eprint is set
\AtEveryBibitem{\iffieldundef{doi}{\iffieldundef{eprint}{}{\clearfield{url}}}{\clearfield{url}}} %Print url only if neither doi nor eprint is set
\AtEveryBibitem{\iffieldundef{eprint}{}{\iffieldundef{issue}{\iffieldundef{volume}{\clearfield{pages}}{}}{}}} %suppress pages in arxiv only entries
\AtEveryCitekey{\iffieldundef{doi}{\iffieldundef{eprint}{}{\clearfield{url}}}{\clearfield{url}}} %do the same also in fullcite
\AtEveryCitekey{\iffieldundef{eprint}{}{\iffieldundef{issue}{\iffieldundef{volume}{\clearfield{pages}}{}}{}}} %do the same also in fullcite

\DeclareFieldFormat{eprint:arXiv}{% generate correct links for old arxiv identifiers
  arXiv:\addcolon\space%
  \ifhyperref%
    {\iffieldundef{eprintclass}%
      {\href{http://arxiv.org/abs/#1}{\nolinkurl{#1}}}%
      {\href{http://arxiv.org/abs/\strfield{eprintclass}/#1}{\nolinkurl{\strfield{eprintclass}/#1}}}}%
    {\nolinkurl{#1}}}
% \renewbibmacro*{issue+date}{%
%   \printtext[parens]{%
%     \iffieldundef{issue}
%     {\usebibmacro{date}}
%     {\printfield{issue}%
%       \setunit*{\addspace}%
%       \usebibmacro{date}}}%
%   \newunit}
\renewcommand*{\mkbibacro}[1]{#1} %make url and doi appear in normal font
\DeclareFieldFormat[article]{pages}{#1} %get rid of p. and pp. for articles
% \renewbibmacro*{pages}{%
%   \iffieldequalstr{archivePrefix}{arXiv}
%     {foo}
%     {%
%       \iffieldequalstr{eprinttype}{arxiv}
%         {bar}
%         {%
%           \setunit{\bibpagespunct}%
%           \printfield{pages}%
%         }%
%     }%
% }
\AtEveryBibitem{\clearfield{month}} %get rid of month
\AtEveryCitekey{\clearfield{month}} %get rid of month also in fullcite
\renewbibmacro{in:}{\ifentrytype{article}{}{\printtext{\bibstring{in}\intitlepunct}}} %get rid of In:
\renewcommand{\newunitpunct}{\addcomma\space} %use , to separate units in the same BibTeX entry
\renewcommand{\finentrypunct}{\addperiod\space} %use . at the end
%\DeclareFieldFormat[article]{volume}{{\bf~#1}}

%\include{makecitesort} %makes the entries in \cite commands appear in the same order as in the bibliography

\newcommand{\stoc
    }[1]{\if\lName1\skp{ }{Proceedings of the #1 {ACM} Symposium on the Theory of Computing ({STOC})}{ }\else{STOC}\fi}

\bibliography{bibliography}

% Packages
\usepackage{filecontents}
\usepackage[utf8]{inputenc}
\usepackage[T1]{fontenc}
\usepackage{csquotes}
\usepackage[russian,german,ngerman,english]{babel}
\usepackage[final]{pdfpages}
\usepackage[pdftex,usenames]{color}
\usepackage{graphicx}
%\usepackage{subfig}
\usepackage{caption}
\usepackage{subcaption}
\urlstyle{same}
%\usepackage[strings]{underscore}
\usepackage{amsmath}
\usepackage{amsthm,amssymb,mathrsfs,wasysym,dsfont,bbm}
\usepackage{eurosym}
\usepackage{centernot}
\usepackage{mathtools}
\usepackage{setspace}
\usepackage{times}
\usepackage{avant}
\usepackage{ifthen}
\usepackage{enumitem}
\setlist{nolistsep,itemsep=\parskip,labelindent=0pt,leftmargin=\parindent,font=\normalfont}
\newlist{notationoverview}{description}{10}
\setlist[notationoverview]{nolistsep,itemsep=0pt,labelindent=0pt,leftmargin=0.5\textwidth,font=\normalfont,style=sameline}
%\setlist[description]{font=\normalfont\itshape}
\usepackage{tikz}
\usetikzlibrary{calc}
\usetikzlibrary{decorations.pathreplacing}
%\usetikzlibrary{shapes}
%\usepackage{pgf}
%\usepackage{pgfbaseimage}
%\usepackage{xxcolor}
\definecolor{structure}{rgb}{0.23,0.4,0.7}
%\usepackage{mathstyle} % must be loaded after tikz and amsmath (nicer super and subscripts but breaks mathchoice)
\usepackage{cancel}

%Load hyperref last...
%\usepackage[pdftex,bookmarks,colorlinks=true,pdfusetitle]{hyperref}
\usepackage[pdftex,bookmarks,colorlinks=false,pdfborder={0 0 0},pdfusetitle]{hyperref}

\pgfrealjobname{dissertation}
\newcommand{\inputTikZ}[1]{%
  \beginpgfgraphicnamed{tikz/#1-external}%
  \input{tikz/#1.tikz}%
  \endpgfgraphicnamed%
}

% Theorems
\newtheoremstyle{mystyle}% name
{\parskip}% space above
{0pt}% space below
{\itshape}% body font
{}% indent amount
{\bfseries}% theorem head font
{\normalfont{.}}% punctuation after theorem head
{.5em}% space after theorem head
{}% theorem head spec
\theoremstyle{mystyle}

\numberwithin{equation}{section}
\numberwithin{figure}{section}

\newtheorem{lemma}{Lemma}
\numberwithin{lemma}{section}
\newtheorem{theorem}{Theorem}
\numberwithin{theorem}{section}
\newtheorem{corollary}{Corollary}
\numberwithin{corollary}{section}
\newtheorem{definition}{Definition}
\numberwithin{definition}{section}
\newtheorem{conjecture}{Conjecture}
\numberwithin{conjecture}{section}
\newtheorem{observation}{Observation}
\numberwithin{observation}{section}

% Compatibility command with LaTeX Beamer
\newsavebox{\blocksavebox}
\newenvironment{block}[1]{%
\begin{lrbox}{\blocksavebox}%
\begin{minipage}%
%{\dimexpr\columnwidth-2\fboxsep\relax}%
{\linewidth}%
}{%
\end{minipage}\end{lrbox}%
% \begin{tikzpicture}
%   \node (blocknode) {\usebox{\blocksavebox}};
%   \draw[rounded corners=0.2cm] (blocknode.north east) rectangle (blocknode.south west);
% \end{tikzpicture}%
\usebox{\blocksavebox}
}
% \newcommand{\uncover}[4]{#4}
% \newcommand{\only}[4]{#4}
% \newcommand{\hover}[4]{#4}
%\definecolor{niceblue}{rgb}{0.33,0.5,0.8}
\definecolor{niceblue}{rgb}{0.7,0.7,0.7}

% Listings
%\setkomafont{descriptionlabel}{\it}

% Captions
%\captionsetup{labelfont=bf,labelsep=period,textfont=small} %is done by koma script
\captionsetup[subfigure]{font={small},labelfont={small}}

% Hacks and fixes
%\mathcode `f="8000 %produces an error for each f in math mode useful for changes in notation.
%\DeclareMathSymbol{f}{\mathalpha}{operators}{`Y} %replaces all f by Y useful for changes in notation.

\newcommand{\+}{\mkern2mu}
\newcommand{\coloneqq}{\mathrel{\vcentcolon\mkern-1.2mu=}} %fix a bug in the old version of amsmath
\newcommand{\eqqcolon}{\mathrel{=\mkern-1.2mu\vcentcolon}}

% Shortcuts

\newcommand{\union}{\cup}
\newcommand{\intersection}{\cap}
\newcommand{\dunion}{\mathbin{\dot{\cup}}}
\newcommand{\floor}[1]{\left\lfloor #1 \right\rfloor}
\newcommand{\ceiling}[1]{\left\lceil #1 \right\rceil}
\newcommand{\texteqref}[1]{Eq.~\eqref{#1}}
\newcommand{\argdot}{{\,\cdot\,}}
\newcommand{\oftype}{\colon}
\newcommand{\suchthat}{\colon}
%\newcommand{\itholds}{\ifinner :\ \else :\quad \fi}
\newcommand{\itholds}{\colon\mathchoice{\quad}{}{}{}}
\renewcommand{\max}{\mathchoice{\operatorname*{max}}{\operatorname*{max}}{\mathrm{max}}{\mathrm{max}}} %It allows you to use \max both as a math operator (like in \max_{x\in[0,1]} x^2 = 1) and in sub- and superscripts (like in f_\max) and it will in both cases be typeset in the correct font and with the correct spacing.
\renewcommand{\min}{\mathchoice{\operatorname*{min}}{\operatorname*{min}}{\mathrm{min}}{\mathrm{min}}}
%\newcommand{\img}{\mathchoice{\operatorname*{img}}{\operatorname*{img}}{\mathrm{img}}{\mathrm{img}}}
\newcommand{\ltwo}{\ell^2}
\newcommand{\levelspaceing}{\epsilon}
\renewcommand{\H}{H}
\newcommand{\Basis}{\mathds{B}}
\newcommand{\CH}{\mathscr{H}}
\newcommand{\rhog}{g}
\newcommand{\rhomc}{{\sqcap}}
%\newcommand{\rhomc}{\mu}
\newcommand{\muhaar}{\mu_{\mathrm{Haar}}}
\newcommand{\const}{\mathrm{const}}
\newcommand{\muC}{\mu_C}
\newcommand{\animalc}{\alpha}
\newcommand{\Vset}{\mathcal{V}}
\newcommand{\Eset}{\mathcal{E}}
\newcommand{\bra}[1]{\langle #1|}
\newcommand{\ket}[1]{|#1\rangle}
\newcommand{\braket}[2]{\langle #1 | #2 \rangle}
\newcommand{\ketbra}[2]{\ket{#1}\!\bra{#2}}
\newcommand{\ex}[2]{\langle #1 \rangle_{#2}}
\newcommand{\abs}[1]{|{#1}|}
\newcommand{\norm}[2][]{
  \ifthenelse{\equal{#1}{}}
    {\left\| {#2} \right\|}
    {\ifthenelse{\equal{#1}{uinv}}
      {\left\vert\kern-0.25ex\left\vert\kern-0.25ex\left\vert {#2} \right\vert\kern-0.25ex\right\vert\kern-0.25ex\right\vert}
      {\left\| {#2} \right\|_{#1}}
    }
}
\newcommand{\notimplies}{\centernot\implies}
\newcommand{\notimpliedby}{\centernot\impliedby}
\newcommand{\notiff}{\centernot\iff}
\newcommand{\taverage}[2][]{
  \ifthenelse{\equal{#1}{}}
  {\overline{#2}}
  {\overline{#2}^{#1}}
}
\newcommand{\eqcoef}{C_{\mathrm{eq}}}
\newcommand{\deff}{d^{\mathrm{eff}}}
\newcommand{\tracedistance}[3][]{
  \ifthenelse{\equal{#2}{}}
  {\ifthenelse{\equal{#3}{}}
    {\mathcal{D}_{#1}}{}
  }{
    \ifthenelse{\equal{#1}{}}
    {\operatorname{\mathcal{D}}(#2,#3)}
    {\operatorname{\mathcal{D}}_{#1}(#2,#3)}
  }
}
\DeclareMathOperator{\geometricentanglement}{E}
\DeclareMathOperator{\landauO}{O}
\DeclareMathOperator{\landauOmega}{\Omega}
\DeclareMathOperator{\landauTheta}{\Theta}
\DeclareMathOperator*{\probability}{\mathbb{P}}
\DeclareMathOperator*{\expectation}{\mathbb{E}}
%\DeclareMathOperator{\boundary}{\partial}
%\DeclareMathOperator{\lspan}{span}
\DeclareMathOperator*{\maxprime}{\mathrm{max}^\prime}
%\DeclareMathOperator{\arg}{arg}

% \newcommand{\rhomc}[3][]{
%   \ifthenelse{\equal{#1}{}}
%     {\sqcap_{[#2,#2+#3]}}
%     {\sqcap_{[#2,#2+#3]}[#1]}
% }
\newcommand{\Sr}[3][]{
  \ifthenelse{\equal{#1}{}}
    {\operatorname{\mathnormal{S}}(#2\|#3)}
    {\operatorname{\mathnormal{S}}_{#1}(#2\|#3)}
}
\newcommand{\dd}[1]{\mathop{\mathrm{d}#1}}
\newcommand{\del}{\mathop{}\!\partial}
\newcommand{\ddel}{\mathop{}\!d}
\newcommand{\ad}{^\dagger}
\newcommand{\compl}[1]{{{#1}^c}}
\newcommand{\trunc}[2]{{#1}_{\upharpoonright \mathnormal{#2}}}

\renewcommand{\Re}{\operatorname{Re}}
\renewcommand{\Im}{\operatorname{Im}}
\newcommand{\e}{\mathrm{e}}
\renewcommand{\i}{\mathrm{i}}
\DeclareMathOperator{\1}{\mathds{1}}
\DeclareMathOperator{\id}{\mathrm{id}}
\newcommand{\POVMs}{\mathcal{M}}
\DeclareMathOperator{\Bop}{\mathcal{B}}
\DeclareMathOperator{\Tcl}{\mathcal{T}}
\DeclareMathOperator{\Obs}{\mathcal{O}}
\DeclareMathOperator{\Qst}{\mathcal{S}}
\DeclareMathOperator{\Cst}{\mathscr{S}}
\DeclareMathOperator{\Qch}{\mathcal{T^+}}
\DeclareMathOperator{\Chann}{\mathnormal{C}}
\DeclareMathOperator{\Svn}{\mathnormal{S}}
%\DeclareMathOperator{\Powerset}{\mathcal{P}}
%\DeclareMathOperator{\End}{End}
\DeclareMathOperator{\Sym}{Sym}
\DeclareMathOperator{\Tr}{Tr}
\DeclareMathOperator{\Perm}{Perm}
\DeclareMathOperator{\Erfc}{Erfc}
\DeclareMathOperator{\Erf}{Erf}
\DeclareMathOperator{\Poly}{Poly}
\DeclareMathOperator{\supp}{supp}
\DeclareMathOperator{\cov}{cov}
\DeclareMathOperator{\dist}{d}
\DeclareMathOperator{\rank}{rank}
\DeclareMathOperator{\dom}{dom}
\DeclareMathOperator{\spec}{spec}
\DeclareMathOperator{\diag}{diag}
\DeclareMathOperator{\sgn}{sgn}

% mathcal
\newcommand{\mc}[1]{\mathcal{#1}}
\newcommand{\mcA}{\mc{A}}
\newcommand{\mcH}{\mc{H}}
\newcommand{\mcS}{\mc{S}}
\newcommand{\mcP}{\mc{P}}
\newcommand{\mcQ}{\mc{Q}}
\newcommand{\mcD}{\mc{D}}
\newcommand{\mcG}{\mc{G}}
\newcommand{\mcC}{\mc{C}}

% mathbb
\newcommand{\mb}[1]{\mathbb{#1}}
\newcommand{\N}{\mb{N}}
\newcommand{\Z}{\mb{Z}}
\newcommand{\R}{\mb{R}}
\renewcommand{\C}{\mb{C}} %\C seems to be defined by the russian babel package
%\DeclareMathOperator*{\E}{\mathbb{E}}

%hyphenation
\hyphenation{after Studien-stiftung Singapore}

\definecolor{christian}{rgb}{.7,.1,0}
\newcommand{\cg}[1]{{\color{christian} #1}}


\begin{document}


% %%% Set pdf meta data (does not work with hyperref)
% \pdfinfo{
% 	/Author (Christian Gogolin)
% 	/Title (Statistical mechanics from non-equilibrium quantum mechanics)
% %	/Subject (03.67.-a, 89.70.Eg, 05.30.Jp, 42.50.-p)
% %	/Keywords (boson-sampling, sample complexity, quantum simulation, certification, verification, state discrimination, birthday paradox, linear quantum optics)
% 	}

%Quantum information. 03.67.-a,
%computational complexity in, 89.70.Eg
%Boson systems, 05.30.Jp
%Quantum optics, 42.50.-p


\onehalfspacing
\pagenumbering{roman}
\emergencystretch 1em

%\frontmatter

%%%% Title %%%%%%%%%%%%%%%%%%%%%%%%%%%%%%%%%%%%%%%%%%%%%
\titlehead{\begin{center}\vspace{-1.4cm}\vbox{\sffamily{\Large Freie Universität Berlin}\\Dahlem Center for Complex Quantum Systems}
\vspace{7mm}
%\hyperref[toc]{\includegraphics[width=0.35\linewidth]{Fub_siegel}}
\includegraphics[width=0.35\linewidth]{Fub_siegel}
\vspace{-3mm}
\end{center}}
%\subject{\normalfont\normalsize\sffamily Im Fachbereich Physik der\\ Freien Universität Berlin\\ eingereichte Dissertation}
\title{Equilibration and thermalization in quantum systems}
\subtitle{\normalfont\normalsize\sffamily Im Fachbereich Physik der\\ Freien Universität Berlin\\ eingereichte Dissertation}
\author{Christian Gogolin}
\date{\normalfont\normalsize\sffamily Berlin, 2014}


{\sffamily \maketitle}
\newpage
\null
\vfill
\thispagestyle{empty}
  \begin{tabbing}%
    Erstgutachter:      \hspace{1,5cm} \= Prof.\ Dr.\ Jens Eisert, Freie Universität Berlin\\
    Zweitgutachter:\> Prof.\ Dr.\ Felix von Oppen, Freie Universität Berlin\\
    Tag der Disputation:  \> 2014-07-11\\
  \end{tabbing}

\cleardoublepage


%%%%Quotation%%%%%%%%%%%%%%%%%%%%%%%%%%%%%%%%%%%%%%%%%%
\thispagestyle{empty}
\vspace*{6cm}
\begin{quotation}
  ``Even things that are true can be proved.''
\end{quotation}
\begin{flushright}
--- Oscar Wilde, The Picture of Dorian Gray
\end{flushright}

\cleardoublepage


% %%%%Dedication%%%%%%%%%%%%%%%%%%%%%%%%%%%%%%%%%%%%%%%%%
% \thispagestyle{empty}
% \vspace*{6cm}
% \begin{center}
%   dedication
% \end{center}
% \cleardoublepage


%%%%Table of contents%%%%%%%%%%%%%%%%%%%%%%%%%%%%%%%%%%
\tableofcontents \label{toc}

% 1 Remarks on the foundations of statistical mechanics
%   1.1  Canonical approaches
%        1.1.1   Boltzmann and the H-Theorem
%        1.1.2   Gibbs' ensemble approach
%        1.1.3   (Quasi-)ergodicity
%        1.1.4   Jaynes' maximum entropy approach
%   1.2  Closing remarks
% 2 Pure state quantum statistical mechanics
%   2.1  Preliminaries and notation
%        2.1.1   Hilbert space and state vectors
%        2.1.2   Observables and states
%        2.1.3   Measurements and completely positive maps
%        2.1.4   Norms, distance measures and distinguishability
%        2.1.5   Entropy
%        2.1.6   Time evolution
%        2.1.7   Time averages and dephasing
%        2.1.8   Composite quantum systems and reduced states
%        2.1.9   Correlations and entanglement
%        2.1.10 Gibbs states
%        2.1.11 Microcanonical states
%   2.2  Equilibration
%        2.2.1   Notions of equilibration
%         2.2.2 Equilibration on average
%         2.2.3 Equilibration during intervals
%         2.2.4 A conjecture concerning equilibration
%         2.2.5 Other notions of equilibration
%    2.3 A quantum maximum entropy principle
%    2.4 Decoherence
%    2.5 Typicality
%    2.6 Time scales for equilibration on average
%    2.7 Thermalization
%         2.7.1 What is thermalization?
%         2.7.2 Thermalization under assumptions on the eigenstates
%         2.7.3 Thermalization under assumptions on the initial state
%         2.7.4 Hybrid approaches and other notions of thermalization
%    2.8 Absence of thermalization
%         2.8.1 Violation of subsystem initial state independence
%         2.8.2 A numerical investigation of the violation of initial state independence
%    2.9 Integrability
%         2.9.1 In classical mechanics
%         2.9.2 In quantum mechanics
%    2.10 Decay of correlations and stability of thermal states
% 3 Conclusions
% Bibliography
% A Back matter
%    A.1 Acknowledgements
%    A.2 Abstract
%    A.3 Zusammenfassung
%    A.4 Eigenständigkeitserklärung
%    A.5 Liste der Publikationen des Verfassers
%    A.6 Lebenslauf

%\listoftodos
\cleardoublepage
\pagenumbering{arabic}

%%%%Notation guide%%%%%%%%%%%%%%%%%%%%%%%%%%%%%%%%%%%%%
%\manualmark
\automark[chapter]{section}
\markboth{Notation guide and definitions}{Notation guide and definitions}
\markright{Notation guide and definitions}
\chapter*{Notation guide}
\label{sec:notationguideandsomedefinitions}
\addcontentsline{toc}{chapter}{Notation guide and definitions}
%
All notation will be carefully introduced in Section~\ref{sec:prelimiaries}.
Below is a list of the most used symbols as an aid to memory for the readers convenience.
\newcommand{\notationoverviewitem}[2][]{\item[{#2}] #1}
%\newcommand{\notationoverviewitem}{\item}
\begin{notationoverview}
\notationoverviewitem[Floor and ceiling]{$\floor\argdot$, $\ceiling\argdot$}
\notationoverviewitem[Range]{$ [n]\coloneqq \{1,\dots,n\} $}
 \notationoverviewitem[Union, intersection]{$\union$, $\intersection$ }
\notationoverviewitem[Disjoint union]{$\dunion$}
\notationoverviewitem[Complement]{$\compl X$}
\notationoverviewitem[(Bachmann-)Landau symbols]{$\landauO$, $\landauOmega$, and $\landauTheta$}
\notationoverviewitem[Hilbert spaces]{$\mcH$}
\notationoverviewitem[Dimension]{$d \coloneqq \dim(\mcH)$}
\notationoverviewitem[Bounded operators]{$A \in \Bop(\mcH)$}
\notationoverviewitem[(Hermitian) adjoint]{$A\ad$}
\notationoverviewitem[Commutator and anti-commutator]{$[\argdot,\argdot]$, $\{\argdot,\argdot\}$}
\notationoverviewitem[Observables]{$\Obs(\mcH)$}
\notationoverviewitem[Trace]{$\Tr$}
\notationoverviewitem[Trace class operators]{$A \in \Tcl(\mcH)$}
\notationoverviewitem[(Quantum) states]{$\rho \in \Qst(\mcH)$}
\notationoverviewitem[Expectation value]{$\ex A \rho$}
\notationoverviewitem[Quantum channels]{$\Qch(\mcH)$}
\notationoverviewitem[(Schatten) $p$-norms]{$\norm[p]\argdot$}
\notationoverviewitem[Trace distance]{$\tracedistance\argdot\argdot$}
\notationoverviewitem[Sets of POVMs]{$\POVMs$}
\notationoverviewitem[Distinguishability]{$\tracedistance[\POVMs]\argdot\argdot$}
\notationoverviewitem[Von~Neumann entropy]{$\Svn(\argdot)$}
\notationoverviewitem[Hamiltonians]{$\H$}
\notationoverviewitem[Energy eigenvalues, eigenstates]{$E_k$, $\ket{E_k}$}
\notationoverviewitem[Time averages]{$\taverage[T]{\argdot}$, $\taverage{\argdot}$}
\notationoverviewitem[Dephasing map]{$\$_{\H}(\argdot)$}
\notationoverviewitem[Interaction (hyper)graph]{$\mcG$}
\notationoverviewitem[Vertex set]{$\Vset$}
\notationoverviewitem[Edge set]{$\Eset$}
\notationoverviewitem[Tensor product]{$\otimes$}
\notationoverviewitem[Annihilation and creation operators]{$f_x,b_x$, $f\ad_x,b\ad_x$}
\notationoverviewitem[Subsystems]{$X,Y,S,B \subset \Vset$}
\notationoverviewitem[Subsystem Hilbert spaces]{$\mcH_X$}
\notationoverviewitem[Subsystem Hilbert spaces dimension]{$d_X$}
\notationoverviewitem[Restricted Hamiltonians]{$\H_X$}
\notationoverviewitem[Uncoupled Hamiltonians]{$\H_0$}
\notationoverviewitem[Interaction Hamiltonians]{$\H_I$}
\notationoverviewitem[Truncated operators]{$\trunc A X$}
\notationoverviewitem[Reduced states]{$\rho^X$}
\notationoverviewitem[Covariance in state $\rho$]{$\cov_\rho(\argdot,\argdot)$}
\notationoverviewitem[Geometric measure of entanglement]{$\geometricentanglement_{X|Y}(\argdot)$}
\notationoverviewitem[Inverse temperature]{$\beta$}
\notationoverviewitem[Gibbs states]{$\rhog[\H](\beta)$}
\notationoverviewitem[Energy intervals]{$[E,E+\Delta]$}
\notationoverviewitem[Microcanonical states]{$\rhomc[\H]([E,E+\Delta])$}
\notationoverviewitem[Effective dimension]{$\deff(\omega)$}
\notationoverviewitem[Time averaged/dephased state]{$\omega = \$_{\H}(\rho(0)) = \taverage{\rho}$}
\notationoverviewitem[Energy level populations]{$p_k$}
\notationoverviewitem[Group of unitary operators]{$U(d)$}
\notationoverviewitem[Subspaces]{$\mcH_R$}
\notationoverviewitem[Haar measure]{$\muhaar$}
\notationoverviewitem[Number of energy levels]{$\#_\Delta[\H](E)$}
\notationoverviewitem[Effective entanglement]{$R_{S|B}(\psi)$}
\notationoverviewitem[Equilibration coefficient]{$\eqcoef$}
\notationoverviewitem[Local interaction strength]{$J$}
\notationoverviewitem[Graph distance]{$\dist(\argdot,\argdot)$}
\end{notationoverview}

\cleardoublepage

\newcommand{\firstchaptername}{Preface}

\automark[chapter]{section}
\markboth{\firstchaptername}{\firstchaptername}
\markright{\firstchaptername}
\chapter*{\firstchaptername}
\label{chap:introduction}
\addcontentsline{toc}{chapter}{\firstchaptername}
%
The subject of this thesis is the interplay between quantum mechanics and statistical mechanics, a subject that has been the topic of an ongoing scientific debate since the late 1920s.

When reviewing such an old and extensive field a selection of the existing material is unavoidable.
Such a selection is necessarily to some extent subjective and in the present work a focus has been put on the contributions of the author, in particular \cite{PhysRevE.81.05-1,Kliesch2013a,Gogolin10-masterthesis,Riera2012,PhysRevLett.10-6} and other recent works with a similar mindset.
However, with almost 300 references, a large portion of which are at least summarized and many of which are discussed in detail, the current work arguably constitutes the most comprehensive review of the literature on equilibration and thermalization in closed quantum systems.

This thesis is divided into two chapters and a conclusion chapter.
Chapter~\ref{chap:remarksonthefoundationsofstatmech} sets the scene with a brief review of the canonical foundations of (classical) statistical mechanics and thermodynamics.
Chapter~\ref{sec:quantumstatisticalmechanics} is the main chapter of this work.
It starts with a careful introduction of the notation and mathematical concepts and then gradually builds up the theory of the ``pure state quantum statistical mechanics'' approach, by addressing topics such as equilibration in closed quantum system, decoherence, typicality, time scales of equilibration, thermalization and the absence thereof, the role of integrability, and finally correlations in thermal states.
Chapter~\ref{sec:conclusions} contains concluding remarks and provides an outlook.

In the main chapter (Chapter~\ref{sec:quantumstatisticalmechanics}) an effort has been made to minimize the amount of ``slang terminology'' and to carefully introduce and define all terms that are not absolutely standard.
This is reflected in the relatively long ``preliminaries and notation'' section (Section~\ref{sec:prelimiaries}).
That this potentially makes the reading experience for the experts slightly doggerel, is more than made up for by the increased readability for people from related fields who would like to get into the subject.

Most sections can be read independently of each other.
Whenever this is not the case, or a section builds upon material that was covered in previous sections, corresponding cross-references are provided.

Many sections in Chapter~\ref{sec:quantumstatisticalmechanics} finish with a discussion section that contains more speculative assertions, subjective opinions of the author, brief descriptions of related lines of research that could not be covered in full depth in this work, and discussions of weak points and contrary points of view.
The style of these sections will tend to be more colloquial than the main text.
The motivation for adding these sections is to give a more complete picture of the ongoing scientific debate concerning the subject of this thesis, while at the same time not interrupting the general narrative implied by the results covered in the main text.
The discussion sections are not essential to understand the following material, but it is the hope that they will help to make it easier to understand the big picture.

There exist at least four other works that are related to this thesis which should not remain unmentioned.
They offer a wealth of information complementing the content of this work and should be considered by the interested reader:

First, the book by \textcite{Gemmer09} entitled ``Quantum Thermodynamics''.
Especially Part~II of this book advertises an approach towards the foundations of thermodynamics that is in spirit very close to the approach of this work.
The focus is however more on \emph{typicality}, which we will discuss in Section~\ref{sec:typicality}, but which is not the central topic of this work.
Moreover, the first edition of the book is from 2004, and even though it has been expanded in the second edition from 2009, much of the newer material that takes the center stage in this work is not covered.

Second, the editorial of a New Journal of Physics focus issue on ``dynamics and thermalization in isolated quantum many-body systems'' by \textcite{Cazalilla2010}.
The editorial not only explains the significance of the individual articles published in the focus issue (some of which also play a prominent role in the current work) to the more general endeavor of developing a better understanding of the coherent dynamics of quantum many body systems, but on top of that gives an overview of many of the currently pursued research directions and many additional references.
This renders this editorial an excellent entry point into the more recent literature on the subject.
On the other hand it provides only very little background information, almost no historical context, and assumes that the reader is already familiar with the jargon of the field.

Third, a colloquium in Reviews in Modern Physics by \textcite{Polkovnikov11} entitled ``Nonequilibrium dynamics of closed interacting quantum systems''.
This work gives an overview of recent theoretical and experimental insights concerning such systems, but focuses mainly on the dynamics following so-called \emph{quenches}, i.e., rapid changes in the Hamiltonian of a system and the \emph{eigenstate thermalization hypothesis} (ETH).
We will discuss the ETH in Section~\ref{sec:thermalizationunderassumptionsontheeigenstates}, but the scope of the present work is considerably broader and we will also take a slightly different, quantum information theory inspired, point of view and put the focus more on analytical results.

Forth, a review entitled ``Equilibration and thermalization in finite quantum systems'' by \textcite{Yukalov2011}.
It contains a review of the history of both the experimental realization of coherently evolving, well controlled quantum systems and the observation and numerical investigation of equilibration and thermalization in such systems.
In addition it contains results on equilibration in closed systems with a continuous density of states and in systems undergoing so-called \emph{non-destructive measurements}.


\chapter{Remarks on the foundations of statistical mechanics}
\label{chap:remarksonthefoundationsofstatmech}
%
\begin{quotation}
  ``Statistical physics [...] has not yet developed a set of generally accepted formal axioms, and consequently we have no choice but to dwell on its history.''
\end{quotation}
\begin{flushright}
  --- Jos \textcite{UffinkFinal}
\end{flushright}
%
We begin the discussion with a sketch of the most influential canonical approaches towards the foundations of thermodynamics and statistical mechanics.
We will roughly follow the historical development, but emphasize more the problems of the respective approaches then their undeniable success and ingenuity.

Contrary to the rest of this work, this introductory chapter is rather superficial.
The main justification for the brevity is the existence of several comprehensive works on the topic, in particular the review by \textcite{UffinkFinal} and the book by \textcite{Sklar1995}, but also Refs.~\cite{RevModPhys.27.289,Ehrenfest2002,Penrose1979} and Chapter~4 in Ref.~\cite{Gemmer09}.
Adding yet another work to this list simply seems superfluous and a detailed review of the history of statistical mechanics is beyond the scope of this thesis.
Also, we will not address the more subtle issues of the classical approaches, such as the interpretation of probability and the problem of comparing discrete and continuous measures.

This chapter also differs from the following material in that we will not build the theory from the ground up, but assume that the reader is already familiar with the basic concepts of thermodynamics and statistical mechanics.
This seems eligible as this chapter is not essential for an understanding of the main part of this thesis and can be skipped safely.

This chapter is intended to partially answer the legitimate question of a person already familiar with thermodynamics and statistical physics: ``Why should I care about pure state quantum statistical mechanics? Weren't all the foundational questions already solved in the works from the 19th and early 20th century?''
As we will see, despite the numerous attempts and the great amount of work that has been put into establishing a convincing justification for the methods of statistical mechanics, a commonly accepted foundation for thermodynamics and statistical mechanics is still missing.
As \textcite{Jaynes} puts it: ``There is no line of argument proceding from the laws of microscopic mechanics to macroscopic phenomena that is generally regarded by physicists as convincing in all respects.''


\section{Canonical approaches}
\label{sec:canonicalapproaches}
%
Thermodynamics was originally developed as a purely phenomenological theory.
Prototypical for this era are the laws of \emph{Boyle--Mariotte} and \emph{Gay--Lussac} that state empirically observed relations between the volume, pressure, and temperature of gases.

The more widespread acceptance of the \emph{atomistic hypothesis} in the 18th century opened up the way for a microscopic understanding of such empirical facts.
The works of Clausius \cite{Clausius1857}, Maxwell \cite{Maxwell1860,Maxwell1860a}, Boltzmann \cite{Boltzmann1872}, and Gibbs \cite{Gibbs1902} in the second half of the 19th and the beginning of the 20th century are often perceived as the inception of statistical mechanics (see also Refs.~\cite{Boltzmann1896,UffinkFinal,Sklar1995}).
In this section we review some of these early attempts to develop a deeper understanding of thermodynamics based on microscopic considerations.


\subsection{Boltzmann and the H-Theorem}
\label{sec:boltzmannshtheorem}
%
One of Boltzmann's arguably most important contributions to the development of statistical mechanics is his derivation of what is known today as the \emph{Boltzmann equation} and his \emph{H-theorem} \cite{Boltzmann1872} (see also the first chapter of Boltzmann's book ``Vorlesungen über Gastheorie. Bd. 1.'' \cite{Boltzmann1896} as well as Ref.~\cite[Chapter~4]{Gemmer09} and Ref.~\cite{Sklar1995}).

\begin{figure}[bt]
  \centering
  \begin{itemize}  \setlength{\itemsep}{0.8cm}
  \item[(a)] Time reversal objection (Loschmidt)
    \begin{center}
      \begin{tikzpicture}[scale=0.8,baseline=(current bounding box.base)]
        \node (p1) at (-0.9, 0.65) {};
        \node (p2) at (-0.3, 0.4 ) {};
        \node (p3) at (-0.2, 0.1 ) {};
        \node (p4) at (-0.8,-0.3 ) {};
        \node (p5) at (-0.4,-0.7 ) {};
        \node (q1) at (-0.4, 0.5 ) {};
        \node (q2) at (-0.6, 0.2 ) {};
        \node (q3) at (-0.4, 0.3 ) {};
        \node (q4) at (-0.2,-0.0 ) {};
        \node (q5) at (-0.3,-0.2 ) {};
        \draw[black,fill=white,rounded corners] (-1,-1) rectangle (1,1);
        \foreach \x in {1,2,...,5} {\draw[->,fill=gray] (p\x) circle (2pt);};
        \draw[dashed,gray] (0,-1) -- (0,1);
      \end{tikzpicture}
      $\overset{t}{\longrightarrow}$
      \begin{tikzpicture}[scale=0.8,baseline=(current bounding box.base)]
        \node (p1) at (-0.7, 0.7) {};
        \node (p2) at ( 0.1, 0.4 ) {};
        \node (p3) at (-0.2, 0.1 ) {};
        \node (p4) at ( 0.6,-0.3 ) {};
        \node (p5) at (-0.4,-0.7 ) {};
        \node (q1) at (-0.4, 0.5 ) {};
        \node (q2) at (-0.6, 0.2 ) {};
        \node (q3) at (-0.4, 0.3 ) {};
        \node (q4) at ( 0.2,-0.0 ) {};
        \node (q5) at (-0.3,-0.3 ) {};
        \draw[black,fill=white,rounded corners] (-1,-1) rectangle (1,1);
        \foreach \x in {1,2,...,5} {\draw[->,fill=gray] (p\x) circle (2pt) -- ++(q\x);};
      \end{tikzpicture}
      \quad $\vec{v} \rightarrow -\vec{v}$ \quad
      \begin{tikzpicture}[scale=0.8,baseline=(current bounding box.base)]
        \node (p1) at (-0.7, 0.7) {};
        \node (p2) at ( 0.1, 0.4 ) {};
        \node (p3) at (-0.2, 0.1 ) {};
        \node (p4) at ( 0.6,-0.3 ) {};
        \node (p5) at (-0.4,-0.7 ) {};
        \node (q1) at (-1.0, 0.9 ) {};
        \node (q2) at ( 0.7, 0.6 ) {};
        \node (q3) at (-0.0,-0.1 ) {};
        \node (q4) at ( 1.0,-0.6 ) {};
        \node (q5) at (-0.5,-1.1 ) {};
        \draw[black,fill=white,rounded corners] (-1,-1) rectangle (1,1);
        \foreach \x in {1,2,...,5} {\draw[->,fill=gray] (p\x) circle (2pt) -- ++(q\x);};
      \end{tikzpicture}
      $\overset{t}{\longrightarrow}$
      \begin{tikzpicture}[scale=0.8,baseline=(current bounding box.base)]
        \node (p1) at (-0.9, 0.65) {};
        \node (p2) at (-0.3, 0.4 ) {};
        \node (p3) at (-0.2, 0.1 ) {};
        \node (p4) at (-0.8,-0.3 ) {};
        \node (p5) at (-0.4,-0.7 ) {};
        \draw[black,fill=white,rounded corners] (-1,-1) rectangle (1,1);
        \foreach \x in {1,2,...,5} {\draw[->,fill=gray] (p\x) circle (2pt);};
      \end{tikzpicture}
    \end{center}
  \item[(b)] Recurrence objection (Poincar\'{e}, Zermelo)
    \begin{center}
      \begin{tikzpicture}[scale=0.8,baseline=(current bounding box.base)]
        \node (p1) at (-0.9, 0.65) {};
        \node (p2) at (-0.3, 0.4 ) {};
        \node (p3) at (-0.2, 0.1 ) {};
        \node (p4) at (-0.8,-0.3 ) {};
        \node (p5) at (-0.4,-0.7 ) {};
        \node (q1) at (-0.4, 0.5 ) {};
        \node (q2) at (-0.6, 0.2 ) {};
        \node (q3) at (-0.4, 0.3 ) {};
        \node (q4) at (-0.2,-0.0 ) {};
        \node (q5) at (-0.3,-0.2 ) {};
        \draw[black,fill=white,rounded corners] (-1,-1) rectangle (1,1);
        \foreach \x in {1,2,...,5} {\draw[->,fill=gray] (p\x) circle (2pt);};
        \draw[dashed,gray] (0,-1) -- (0,1);
      \end{tikzpicture}
      $\overset{t}{\longrightarrow} \dots \overset{t}{\longrightarrow}$
      \begin{tikzpicture}[scale=0.8,baseline=(current bounding box.base)]
        \node (p1) at (-0.7, 0.7) {};
        \node (p2) at ( 0.1, 0.4 ) {};
        \node (p3) at (-0.2, 0.1 ) {};
        \node (p4) at ( 0.6,-0.3 ) {};
        \node (p5) at (-0.4,-0.7 ) {};
        \draw[black,fill=white,rounded corners] (-1,-1) rectangle (1,1);
        \foreach \x in {1,2,...,5} {\draw[->,fill=gray] (p\x) circle (2pt);};
      \end{tikzpicture}
      $\overset{t}{\longrightarrow} \dots \overset{t}{\longrightarrow}$
      \begin{tikzpicture}[scale=0.8,baseline=(current bounding box.base)]
        \node (p1) at (-0.9, 0.65) {};
        \node (p2) at (-0.3, 0.4 ) {};
        \node (p3) at (-0.2, 0.1 ) {};
        \node (p4) at (-0.8,-0.3 ) {};
        \node (p5) at (-0.4,-0.7 ) {};
        \node (q1) at (-0.4, 0.5 ) {};
        \node (q2) at (-0.6, 0.2 ) {};
        \node (q3) at (-0.4, 0.3 ) {};
        \node (q4) at (-0.2,-0.0 ) {};
        \node (q5) at (-0.3,-0.2 ) {};
        \draw[black,fill=white,rounded corners] (-1,-1) rectangle (1,1);
        \foreach \x in {1,2,...,5} {\draw[->,fill=gray] (p\x) circle (2pt);};
      \end{tikzpicture}
    \end{center}
  \end{itemize}
  \caption{The \emph{time reversal objection}, also known as \emph{Loschmidt's paradox} \cite{Loschmidt1877}, but actually first published by William Thomson \cite{ThomsonLordKelvin1874}, states that it should not be possible to deduce time reversal asymmetric statements like the H-theorem, implied by the Boltzmann equation, from an underlying time reversal invariant theory. More explicitly, it argues that for any process that brings a system into an equilibrium state starting from a non-equilibrium situation, there exists an equally physically allowed reverse process that takes the system out of equilibrium. The initial state for that process is obtained from the equilibrium state by reversing all velocities (see Panel~(a)).
The \emph{recurrence objection}, which is based on the \emph{Poincaré recurrence theorem} but was made explicit by \textcite{Zermelo1896}, states that Boltzmann's H-theorem is in conflict with Hamiltonian dynamics, because it can be proven on very general grounds that all finite systems are recurrent, i.e., return arbitrarily close to their initial state after possibly very long times (see Panel~(b)).}
  \label{fig:timereversalandrecurrenceobjection}
\end{figure}

In his 1872 article \cite{Boltzmann1872} Boltzmann aims at showing that the \emph{Maxwell-Boltzmann distribution} is the equilibrium distribution of the speed of gas particles and that a gas with an initially different distribution must inevitably approach it.
He tries to do this on the grounds of microscopic considerations and starts off from the prototypical model of the hard sphere gas.
He takes for granted that in equilibrium the distribution of the particles should be ``uniform'' and that their speed distribution should be independent of the direction of movement.
He assumes that the number of particles is large and introduces a continuously differentiable function called ``distribution of state''\footnote{German original \cite{Boltzmann1872}: \foreignlanguage{ngerman}{``Zustandsverteilung''}}, which is meant to approximate the (discrete) distribution of the speed of the particles.
He then derives a differential equation for the temporal evolution of this function, known today as the \emph{Boltzmann equation}.
He also defines an entropy for the ``distribution of state'' and shows that it increases monotonically in time under the dynamics given by the Boltzmann equation, a statement he calls \emph{H-Theorem}, after the letter $H$ used for denoting the entropy.

During the derivation he makes several approximations.
Essential is his ``Stoßzahl Ansatz'', later dubbed the``hypothesis of molecular disorder'' in Ref.~\cite{Boltzmann1896}, which explicitly breaks the time reversal invariance of classical mechanics.
This breaking of the time reversal symmetry is responsible for the  temporal increase of entropy reminiscent of the second law of thermodynamics.
Naturally this assumption has been much criticized.
Famous are the \emph{time reversal objection} of William Thomson and Loschmidt and the \emph{recurrence objection} due to Poincar\'{e} and Zermelo \cite{Sklar1995} (see Fig.~\ref{fig:timereversalandrecurrenceobjection}).

The bottom line of this debate, also later acknowledged by Boltzmann \cite{Boltzmann1896a}, is that any statement that implies the convergence of a finite system to a fixed equilibrium state/distribution in the limit of time going to infinity is incompatible with a time reversal invariant or recurrent microscopic theory.
This will be important for the notions of equilibration we will discuss later in Section~\ref{sec:equilibration}.


\subsection{Gibbs'  ensemble approach}
\label{sec:gibbsensembleapproach}
%
For many, Gibbs' book ``Elementary principles in statistical mechanics'' \cite{Gibbs1902} from 1902 marks the birth of modern statistical mechanics \cite{UffinkFinal}.
Central in Gibbs' approach is the concept of an \emph{ensemble}, which he describes as follows:
``We may imagine a great number of systems of the same nature, but differing in the configurations and velocities which they have at a given instant [\dots] we may set the problem, not to follow a particular system through its succession of configurations, but to determine how the whole number of systems will be distributed among the various conceivable configurations and velocities at any required time [\dots]''

In fact, the book then is not so much concerned with (non-equilibrium) dynamics, but rather with the calculation of statistical equilibrium averages.
Gibbs considers systems whose phase space is, as in Hamiltonian mechanics, spanned by canonical coordinates and introduces the \emph{microcanonical}, \emph{canonical}, and \emph{grand canonical} ensemble for such systems.
He assumes that the number of states is high enough such that a description with a, as he calls it, ``structure function'', a kind of density of states, is possible.
He shows how various thermodynamic relations for quantities such as temperature and entropy can be reproduced from his ensembles, if these quantities are properly defined in terms of the structure function.

Gibbs is mostly concerned with defining recipes for the description of systems in equilibrium.
He gives little insight into why the ensembles he proposes capture the physics of thermodynamic equilibrium or how and why systems equilibrate in the first place \cite{UffinkFinal}.
Instead of addressing such foundational questions he is ``contented with the more modest aim of deducing some of the more obvious propositions relating to the statistical branch of mechanics''\cite{Gibbs1902}.


\subsection{(Quasi-)ergodicity}
\label{sec:ergodicity}
%
The \emph{ergodicity hypothesis} was essentially born out of the incoherent use of different interpretations of probability by Boltzmann in his early work \cite{Boltzmann1868} and was formulated by him in Ref.~\cite{Boltzmann1871} as follows:
``The great irregularity of the thermal motion and the multitude of forces that act on a body make it probable that its atoms, due to the motion we call heat, traverse all positions and velocities which are compatible with the principle of [conservation of] energy.''\footnote{The English translation is taken from Ref.~\cite{UffinkFinal}.}
The concept of ergodicity was made prominent by P.\ and T.\ Ehrenfest in Ref.~\cite{Ehrenfest2002}, who proposed the \emph{ergodic foundations of statistical mechanics} \cite{UffinkFinal}.

Roughly speaking, a system is called \emph{(quasi-)ergodic} if it explores its phase space uniformly in the course of time for most initial states.
Making precise what ``uniformly'', ``most'', and ``in the course of time'' mean in this context already constitutes a mayor challenge \cite{UffinkFinal}.
However, if one is willing to believe that a systems at hand is ergodic in an appropriate sense then it readily follows that (infinite time) temporal averages of physical quantities in that system are (approximately and/or with ``high probability'') equal to certain phase space averages, such as for example that given by the microcanonical ensemble.

The ergodic foundations of statistical mechanics are then roughly based on arguments along the following lines:
Any physical measurement must be carried out during a finite time interval.
What one actually observes is not an instantaneous value, but an average over this time span.
The relevant time spans might seem short on a human time scale, but can at the same time be ``close to infinite'' compared to the microscopic time scales.
Think for example of the process of measuring the pressure in a gas container with a membrane.
The moment of inertia of the membrane is much too large to observe the spikes in the force due to hits by individual particles.
It is thus reasonable to assume that observations are well described by (infinite time) averages of the corresponding quantities, which, if the system is quasi-ergodic, can be calculated by averaging in an appropriate way over phase space.

The arguably most striking objection against such reasoning is the following \cite{Sklar1995}:
If it were in fact true that all realistic measurements could legitimately be described as infinite time averages, then the observation of any non-equilibrium dynamics, including the approach to equilibrium, would simply be impossible.
The latter is manifestly not the case.

Besides this issue of the ``infinite time'' averages and the other problems mentioned above it is extraordinarily difficult to show that a given system is \emph{(quasi-)ergodic}.
Despite the ground breaking works of Birkhoff and von Neumann on the concept of \emph{metric transitivity}, Sinai's work on \emph{dynamical billiards}, and more recent approaches such as \emph{Khinchin's ergodic theorem}, the full problem still awaits solution \cite{UffinkFinal}.


\subsection{Jaynes' maximum entropy approach}
\label{sec:jaynesmaximumentropyapproach}
%
Conceptually very different from the three previously discussed approaches is the work of \textcite{Jaynes}.
He fully embraces a subjective interpretation of probability and proposes to regard statistical physics as a ``form of statistical inference rather than a physical theory''.
He then introduces a \emph{maximum entropy principle}.
In short, the maximum entropy principle states that in situations where the existing knowledge is insufficient to make definite predictions the best possible predictions can be reached by finding the distribution of the state space of the system that maximizes the (Shannon) entropy and is compatible with the knowledge.
The principle is inspired by the work of \textcite{Shannon1949} who, as Jaynes claims, had shown that the maximum entropy distribution is the one with the least bias towards the missing information \cite{Jaynes}: ``[The] maximum entropy distribution may be asserted for the positive reason that it is uniquely determined as the one which is maximally noncommittal with respect to missing information.''

Moreover, in Ref.~\cite{Jaynes}, Jaynes shows in quite some generality that the ``usual computational rules [as presented in Gibbs' book \cite{Gibbs1902}] are an immediate consequence of the maximum entropy principle''.
In addition, he points out various other advantages of his subjective approach.
For example that it makes predictions ``only if the available information is sufficient to justify fairly strong opinions'', and that it can account for new information in a natural way.

While Jaynes principle can be used to justify the methods of statistical mechanics it gives little insight into why and under which conditions these methods yield results that agree with experiments.
In other words: The maximum entropy principle ensures that making predictions based on statistical mechanics is ``best practice'', but does not explain why this ``best practice'' is good enough.
The question ``Why does statistical mechanics work?'' hence remains partially unanswered.

A last point of criticism is that Ref.~\cite{Jaynes} works in a classical setting.
While an extension to quantum mechanics is possible \cite{PhysRev.108.17} the subjective interpretation of probability advertised by Jaynes is arguably less convincing or at least debatable in this setting, although this is of course to some extend a matter of taste \cite{Fuchs,Timpson2008}.
Problems arise because mixed quantum states can be written as convex combinations of pure states in more than one way so that more complicated arguments are needed to identify the von~Neumann entropy as the right entropy measure to be maximized.


\section{Closing remarks}
\label{sec:somegeneralcriticalremarks}
%
Except for Jaynes subjective maximum entropy principle, all approaches we have discussed in this chapter differ in one important point from that advertised in the main part of this thesis:
They are based on classical mechanics.
The applicability of classical models to systems that behave thermodynamically is however questionable.

Consider for example two of the most prominently used models in statistical mechanics:
The hard sphere model for gases and the Ising model for ferromagnetism.
The atoms and molecules of a gas, as well as the interactions between them, in principle require a quantum mechanical description.
It is however often claimed that in the so-called \emph{Ehrenfest limit}, i.e., if the spread of the quantum mechanical wave packets of the individual particles is small compared to the ``radius'' of the particles, the classical hard sphere approximation is eligible.
It can however be shown that under reasonable conditions systems typically leave the Ehrenfest limit on timescales much shorter than those of usual thermodynamic processes \cite[Chapter~4]{Gemmer09}.
Moreover, whether the Ehrenfest limit constitutes a sufficient condition for the applicability of (semi)classical approximations in the first place is debatable \cite{Ballentine1994}.
Similarly, the relevant elementary magnetic moments of a piece of iron, namely the electronic spins, are intrinsically quantum.
In fact, it is known that classical physics alone cannot explain the phenomenon of ferromagnetism in a satisfactory way --- a statement known as \emph{Bohr--van Leeuwen theorem} \cite{Bohr1911,Aharoni2000,Nolting2009}.
The extremely simplified description employed in the Ising model can thus, despite its pedagogical value, arguably not capture all the relevant physics.

In addition to this, there are many situations where thermodynamic behavior cannot be understood in a purely classical framework \cite{greiner}:
For example, black-body radiation cannot be understood without postulating a quantization of energy to avoid the \emph{ultraviolet catastrophe}.
Further prime example for this are gases of indistinguishable particles.
An application of classical physics leads to \emph{Gibbs' paradox} for the mixing entropy and the statistics of Bose and Fermi gases at low temperatures cannot be explained classically.
Last but not least, the ``freezing out'' of certain internal degrees of freedom of molecular gases, which impacts their heat capacities, cannot be understood in a convincing way from classical physics alone.

In the light of the above discussion it appears reasonable to try to ``derive'' statistical mechanics and thermodynamics from quantum mechanics.
In the following we will thus specifically use quantum mechanics as the underlying microscopic theory and exploit quantum effects.
It is not claimed that this can solve all the problems of the canonical approaches, but, as we will see, it does help in explaining thermodynamic behavior (for a comparison of the difficulties of that arise when one tries to justify the methods of statistical mechanics starting either from classical or quantum mechanics see also Ref.~\cite{Reimann2013}).

\cleardoublepage

\chapter{Pure state quantum statistical mechanics}
\label{sec:quantumstatisticalmechanics}
%
\begin{quotation}
  ``Whenever a theory appears to you as the only possible one, take this as a sign that you have neither understood the theory nor the problem which it was intended to solve.''
\end{quotation}
\begin{flushright}
  --- Karl Popper
\end{flushright}

As we have seen in the introductory chapter (Chapter~\ref{chap:remarksonthefoundationsofstatmech}) the canonical approaches cannot convincingly explain the emergence of thermodynamic behavior from a microscopic theory.
Can this situation be improved by explicitly taking quantum effects into account?
In the remainder of this thesis we will investigate to which extend this is possible.

Especially Refs.~\cite{slloydthesis,PhysRevA.43.20,PhysRevE.50.88,Popescu05,Popescu06,Gemmer09} argue for a new interpretation of the foundations of statistical mechanics based on quantum theory.
Following Refs.~\cite{slloydthesis,lloyed13}, we shall call this approach \emph{pure state quantum statistical mechanics}.
During the last few years there has been enormous progress and the field has attracted a significant amount of attention.

This development has partly been fueled by revolutionary improvements in the experimental techniques that have made it possible to observe the coherent, quantum mechanical non-equilibrium evolution of large systems, in particular in clouds of ultracold atoms and ions \cite{Haffner2005,Greiner2002a,Greiner2002,Sadler2006,Regal2006,Tuchman2006,Aidelsburger2011,Bloch2005,Kinoshita2006,Hofferberth2007,Weller2008,Strohmaier2007} (see also the reviews Refs.~\cite{Bloch2008,Yukalov2011}) and the experimental observation of equilibration and thermalization in such systems \cite{1101.2659v1,1111.0776v1,Langen2013,1112.0013v1,Langen2013a}.

The great interest in the topic is reflected by an enormous amount of theoretical studies, including many (mostly) numerical works on equilibration and thermalization in closed quantum systems and related topics \cite{Sirker2013,1109.5904v1,1110.4690v1,Rigol07,Zhuang2013,PhysRevE.81.06,Lesanovsky10,Beugeling2013,1112.3424v1.pd,Beugeling2013,Yukalov2011,Ikeda2013a,Fine2013,Jensen1985}, often with a focus on so-called \emph{quenches}, i.e., rapid changes of the Hamiltonian \cite{Moeckel2008,Kollath07,Rigol08,Rigol09,1011.0781v1,Venuti09,1108.2703v1,1109.5904v1,Rigol11,Rigol2006,Cassidy11,1102.0528v1,Torres-Herrera2013,1103.0787v1,1102.3651v1,1108.0928v1,Mazza2013}.
In addition, there exists a large number of partly or entirely analytical works that study these and related phenomena in concrete systems or classes of models (often integrable ones) \cite{Sengupta2004,Flesch08,Ates2011,Fagotti2012,Calabrese2007,Cazalilla2006,Fioretto2010,1104.0154v1,Cazalilla11,Eckstein2008,Campos10,1206.2408v1,Karzig2010,Iucci2010,Iucci2010,Iucci2009,Geiger2013,Queisser2013,Kehrein2013}.
The above list is grossly incomplete.

The focus of this work lies on the fundamental aspects of the interplay between quantum mechanics, statistical physics and thermodynamics.
At the heart of the approach advertised in this thesis lies the attempt to use standard quantum mechanics only to explain the emergence of thermodynamic behavior, and to do this in a mathematically rigorous and general way.
It is an invitation to explore how much of statistical mechanics and thermodynamics can be \emph{derived} from quantum mechanics.
\emph{Derived} here means to justify the well established methods and postulates of equilibrium and non-equilibrium statistical mechanics by means of the microscopic picture provided by quantum mechanics.
It is by no means the intention to overthrow statistical mechanics, but rather to install a sound and solid foundation that can serve as a basis for this extremely well corroborated theory.

We will follow three main guiding principles:

The first principle is to strictly stay in the well-defined setting of finite dimensional closed system quantum mechanics with unitary time evolution.
No additional postulates and no uncontrolled approximations are to be used.
In particular, we will \emph{not} break the time reversal invariance of quantum mechanics, e.g., by making a \emph{Born-Mark off approximation}.
In addition, we will try to avoid ``putting probabilities by hand'' by, for example, assuming the applicability of a description relying on ensembles.

The second principle is to start off considering very general scenarios and to gradually add more structure, and finally consider more concrete systems only, once this seems indispensable to show the specific effect one is interested in.
This is motivated by the fact that thermodynamic behavior is ubiquitous in nature and that it hence seems plausible that it is the consequence of very general mechanisms independent of specific (toy) models.
Moreover, this approach has the advantage of yielding insights into precisely which properties of a specific model are necessary or sufficient to yield a certain behavior.

The third principle is that of mathematical rigor.
It is the intention of the author to reach a level of mathematical rigor that is above the standard of an average physics article.
Rigorous results will be organized in \emph{lemmas}, \emph{theorems}, and \emph{corollaries}.
Only when stating a fully rigorous result would be too cumbersome, or giving all the necessary conditions would obfuscate the physically relevant statement too much, we will resort to making semi rigorous statements and call them \emph{observation}.

On the downside, the elevated level of rigor will make it necessary to spend quite some time to carefully introduce the basic concepts.
On the upside, this will make it easier to see exactly which assumptions are needed to prove a particular statement and allow us to present the results in a form that makes them easily reusable in future work.

The main reason why an elevated level of rigor seems necessary is that on an intuitive, non-rigorous level we already know that statistical mechanics and thermodynamics work and in which situations we expect them to be applicable.
Arguing in favor of the methods of statistical mechanics in a hand-wavy way thus seems superfluous.

For a reader with a good physical intuition many of the results discussed in this text will be not very surprising.
What is remarkable is the extent to which vague physical intuition can actually be turned into rigorous theorems and what can be learned in the process of doing this.


\section{Preliminaries and notation}
\label{sec:prelimiaries}
%
In this section we will fix the notation and introduce the concepts that form the mathematical foundation of the theory of (mostly) finite dimensional, non-relativistic quantum mechanics.
The presentation will be limited to the minimum necessary to make the following statements well-defined.
An effort has been made to make this introduction self-contained.
However, a basic knowledge of analysis, linear algebra, group theory and related subjects is assumed.
Technical terms are typeset in italics on their first occurrence and when a definition is given.
For some facts and terms that can safely be assumed to be standard knowledge of readers familiar with quantum mechanics we will not provide specific references.
These facts and additional background information can be found in introductory textbook such as Refs.~\cite{FeynmanV03,Sakurai1995,nielsenchuang}, in the more rigorous works Refs.~\cite{Teschl,thirringquantu,Galindo1990,Galindo1991}, as well as in more specialized textbooks, such as Refs.~\cite{Reed1980,Reed1975,bhatia2,bhatia}.

To begin with, we fix some general notation.
We denote the logical \emph{and} by $\land$ and the logical \emph{or} by $\lor$.
Given a number $r \in \R$ we denote by $\floor r$ the greatest integer that is less than or equal to $r$ and by $\ceiling r$ the least integer that is greater than or equal to $r$.
Given a positive integer $n \in \Z^+$ we use the short hand notation $[n] \coloneqq \{1,\dots,n\}$ for the \emph{range} of numbers from $1$ to $n$ and set $[\infty] \coloneqq \Z^+$.
Given a complex number $c \in \C$ we write $|c|$ for its \emph{modulus} and $\arg(c)$ for its \emph{argument}, i.e., $c = |c|\,\e^{\i\,\arg(c)}$.

Given a \emph{set} $X$ we denote its \emph{cardinality} by $|X|$.
If $X$ has a \emph{universal superset} $\Vset \supset X$, we write $\compl X \coloneqq \Vset \setminus X$ for its complement.
It will always be clear from the context what the universal superset is.
We denote the \emph{empty set} by $\emptyset$.
Given two sets $X,Y$ we write $X \union Y$ and $X \intersection Y$ for their \emph{union} and \emph{intersection}.
To stress that a set $\Vset$ is the union of two \emph{disjoint} sets $X,Y$, i.e., $X \cap Y = \emptyset$ we write
$\Vset = X \dunion Y$.
Given a set $X$ of sets we write $\union X \coloneqq \bigcup_{x \in X} x$ for the union of the sets in $X$.
For \emph{sequences} $S$, $|S|$ denotes the length of the sequence.
When we define sets or sequences in terms of their elements we use curly $\{\argdot\}$ or round $(\argdot)$ brackets respectively.

Given two functions $f,g$ with suitable \emph{domain} and \emph{image} we denote by $g \circ f$ their \emph{composition}, i.e., the function that is equivalent to applying $g$ to the outcome of an application of $f$.

When we write $\log$ we will always mean the logarithm to base 2 and denote the natural logarithm by $\ln$.
We denote by $\delta_{x,y}$ the \emph{Kronecker delta}, which for $x,y \in \C$ is equal to $1$ if $x=y$ and zero otherwise.

We use the \emph{(Bachmann-)Landau} symbols $\landauO$, $\landauOmega$ and $\landauTheta$ to denote asymptotic growth rates of real functions $f,g\oftype\R\to\R$.
In particular
\begin{align}
  f(x) \in \landauO(g(x)) &\iff \limsup_{x\to\infty}|f(x)/g(x)| < \infty ,\\
\intertext{and for $\landauOmega$ we adopt the convention from complexity theory that}
  f(x) \in \landauOmega(g(x)) &\iff g(x) \in \landauO(f(x))
\end{align}
and write $f(x) \in \landauTheta(g(x))$ if both $f(x) \in \landauO(g(x))$ and $f(x) \in \landauOmega(g(x))$.

To simplify the notation we work with \emph{natural}, or \emph{Planck units} such that in particular the Planck constant $\hbar$ and the Boltzmann constant $k_B$ are equal to $1$.


\subsection{Hilbert space and state vectors}
\label{sec:hilbertspaceandstatevectos}
%
Let $\mcH$ be a \emph{separable Hilbert space} over $\C$.
We use \emph{Dirac-notation}, i.e., we denote by $\braket\varphi\psi$ the inner product of $\ket\varphi,\ket\psi\in\mcH$, write $\ket\psi \in \mcH$ for elements of $\mcH$ and denote linear functionals on $\mcH$ by $\bra\psi\oftype\mcH\to\C$.
The inner product induces the norm $\norm{\ket\argdot} \coloneqq \sqrt{\braket{\argdot}{\argdot}}$ on vectors from $\mcH$, which we will refer to as the \emph{Hilbert space norm}.
We call the normalized elements $\ket\psi\in\mcH$, i.e., those with $\norm{\ket\psi} = 1$, \emph{state vectors}.
A countable subset $\Basis \subset \mcH$ of linearly independent vectors is called a \emph{basis} of $\mcH$ if each element in $\mcH$ can be arbitrarily well approximated by a linear combination of vectors from $\Basis$.
The number of elements in a basis of a given Hilbert space $\mcH$ is independent of the particular choice of the basis and is called the \emph{dimension} $\dim(\mcH) \in \Z^+ \union \{\infty\}$ of $\mcH$.
A basis $\Basis = (\ket k)_{k=1}^{\dim(\mcH)}$ is called \emph{orthonormal} if $\braket k l = \delta_{k,l}$ for all $k,l\in [\dim(\mcH)]$.

Throughout most of this thesis we will work in the framework of finite dimensional quantum mechanics.
That is, if not explicitly stated otherwise, we consider systems that are described by a Hilbert space $\mcH$ over $\C$ whose dimension $d\coloneqq\dim(\mcH)$ is finite.
In Section~\ref{sec:equlibrationinthestrongsense} however, we will consider systems of bosons, whose Hilbert space is infinite dimensional.
As we have to make this exception, we will keep this introduction more general than would be necessary in the exclusively finite dimensional setting and be a bit more careful when introducing the fundamental concepts, without, however, delving into the details of functional analysis \cite{Reed1980,Reed1975,BratteliRobinson1,BratteliRobinson2}.


\subsection{Observables and states}
\label{sec:observablesandstates}
%
Let $\Bop(\mcH)$ be the Banach space of \emph{bounded (linear) operators} on the (for now not necessarily finite dimensional) Hilbert space $\mcH$, i.e.,
\begin{equation} \label{eq:operatornorm}
  \forall A \in \Bop(\mcH) \itholds \norm[\infty]A \coloneqq \sup_{\ket\psi\in\mcH\suchthat\norm{\ket\psi}=1} \norm{A \ket\psi} < \infty.
\end{equation}
We call $\norm[\infty]\argdot$ the \emph{operator norm} and denote the \emph{identity operator} on $\mcH$ by $\1 \in \Bop(\mcH)$.
Each $A \in \Bop(\mcH)$ has a unique \emph{(Hermitian) adjoint} $A\ad \in \Bop(\mcH)$ with the property that
\begin{equation} \label{eq:defadjoint}
  \forall \ket\psi,\ket\phi\in\mcH\itholds \bra\phi A \ket\psi^* = \bra\psi A\ad \ket\phi.
\end{equation}
We call an operator $A\in \Bop(\mcH)$ \emph{self-adjoint} if $A = A\ad$.
The space of bounded linear operators $\Bop(\mcH)$, together with the usual operator multiplication (concatenation) and the involution $\ad$ is a \emph{$C^*$-algebra} \cite{Arveson76}.

For any two operators $A,B \in \Bop(\mcH)$ we define their \emph{commutator} $[A,B] \coloneqq A\,B - B\,A$ and their \emph{anti-commutator} $\{A,B\} \coloneqq A\,B + B\,A$.
We say that $A,B$ \emph{commute} or \emph{anti-commute} if $[A,B] = 0$ or $\{A,B\}=0$ respectively.
The \emph{rank} of an operator $A \in \Bop(\mcH)$, denoted by $\rank A$, is the dimension of its image.

An operator $\Pi \in \Bop(\mcH)$ is a \emph{projector} if $\Pi\,\Pi = \Pi$.
A projector $\Pi \in \Bop(\mcH)$ is self-adjoint if and only if it is an \emph{orthogonal projector}, i.e., it acts like the identity on a subspace of $\mcH$ and maps the orthogonal subspace to zero.
%The subspace on which an orthogonal projector $\Pi$ acts like the identity is called its \emph{image} and we denote it by $\img(\Pi)$.
An operator $U \in \Bop(\mcH)$ is called \emph{unitary} if $U\ad\,U = U\,U\ad = \1$.
The unitary operators on a Hilbert space of dimension $d$ form a group, which we denote by $U(d)$.
The elements of the subspace $\Obs(\mcH) \subset \Bop(\mcH)$ of self-adjoint operators are called \emph{observables}.

Denoting by $(\ket j)_{j=1}^{\dim(\mcH)}$ some orthonormal basis of $\mcH$, we define the \emph{trace}
\begin{equation} \label{eq:trace}
  \Tr(A) \coloneqq \sum_{j=1}^{\dim(\mcH)} \bra j A \ket j ,
\end{equation}
for any operator $A \in \Bop(\mcH)$ for which the series in \texteqref{eq:trace} is absolutely convergent.
Any such operator is said to be \emph{trace class} and the trace $\Tr$ is a linear functional on the space of trace class operators $\Tcl(\mcH)$.
Note that the definition is independent of the choice of the orthonormal basis.
If $A \in \Tcl(\mcH)$ and $B \in \Bop(\mcH)$, then both $A\,B \in \Tcl(\mcH)$ and $B\,A \in \Tcl(\mcH)$, i.e., the space of trace class operators is an \emph{ideal}.

To keep the rest of this section as simple as possible we will from now on assume that $\mcH$ is finite dimensional.

The \emph{spectral theorem} tells us that any self-adjoint operator $A \in \Obs(\mcH)$ can be written in the form
\begin{equation} \label{eq:spectraldecomposition}
  A = \sum_{k=1}^{|\spec(A)|} a_k\,\Pi_k ,
\end{equation}
which is called the \emph{spectral decomposition} of $A$.
The sequence $\spec(A) \coloneqq (a_k)_{k=1}^{|\spec(A)|}$ of distinct and ordered, i.e., $k<l \implies a_k < a_l$, \emph{eigenvalues} $a_k \in \R$ of $A$ is called its \emph{spectrum}, $\Pi_k \in \Obs(\mcH)$ are the orthogonal \emph{spectral projectors} onto the \emph{eigenspaces} of $A$, i.e.,
\begin{equation}
  \forall k \in [|\spec(A)|]\itholds A\,\Pi_k = a_k\,\Pi_k .
\end{equation}
\texteqref{eq:spectraldecomposition} moreover implies that for any self-adjoint operator there exists at least one orthonormal basis $(\ket{\tilde{a}_k})_{k=1}^d$ of \emph{eigenstates} in which the operator $A$ is \emph{diagonal}, i.e., $\bra{\tilde{a}_k} A \ket{\tilde{a}_l} \propto \delta_{k,l}$.

Given a measurable function $f\oftype\R\to\R$, the \emph{(Borel) functional calculus} allows to assign for any $A \in \Obs(\mcH)$ (with spectral decomposition as in \texteqref{eq:spectraldecomposition}) in a natural way an operator to the expression $f(A)$ that satisfies
\begin{equation}
  f(A) = \sum_{k=1}^{|\spec(A)|} f(a_k)\,\Pi_k .
\end{equation}
We will make use of this, for example, for the exponential function $\exp$.

We call an operator $A \in \Bop(\mcH)$ \emph{positive} and write $A>0$ if $\forall\+\ket\psi\in\mcH\itholds\bra \psi A \ket \psi > 0$ and accordingly for \emph{negative}.
All positive and all negative operators are self-adjoint.
We call a self-adjoint operator $A \in \Bop(\mcH)$ \emph{non-negative} and write $A\geq0$ if $\spec(A) \in \R_0^+$ and accordingly for \emph{non-positive}.
We generalize this notion to pairs of operators $A,B\in\Obs(\mcH)$ and write $A \geq B$ if $A-B \geq 0$ and accordingly for $\leq$, $>$, and $<$.

Together with the \emph{Hilbert-Schmidt inner product}
\begin{equation}
  \forall A,B \in \Bop(\mcH)\itholds \langle A,B\rangle \coloneqq \Tr(A\ad B),
\end{equation}
$\Bop(\mcH)$ is itself a Hilbert space with the induced norm
\begin{equation} \label{eq:twonorm}
  \forall A \in \Bop(\mcH)\itholds \norm[2]A \coloneqq \sqrt{\Tr(A\ad A)} ,
\end{equation}
called the \emph{(Schatten) 2-norm}.

Consider the dual space $\Bop^*(\mcH)$ of the space of bounded linear operators $\Bop(\mcH)$, i.e., $\Bop^*(\mcH)$ is the space of linear functionals that map $\Bop(\mcH)$ to $\C$.
\emph{Riesz representation theorem} ensures that all elements of $\Bop^*(\mcH)$ can be uniquely written in the form $\langle A,\argdot\rangle$ for some trance class operator $A \in \Tcl(\mcH)$.
The natural norm on the space of trace class operators $\Tcl(\mcH)$ is the \emph{(Schatten) $1$-norm} defined by
\begin{equation} \label{eq:onenorm}
  \forall A \in \Tcl(\mcH) \itholds \norm[1]A \coloneqq \sum_{j=1}^{\dim(\mcH)} | \bra j A \ket j | < \infty ,
\end{equation}
where $(\ket j)_{j=1}^{\dim(\mcH)}$ is some orthonormal basis of $\mcH$.
The definition is independent of the choice of the orthonormal basis.
The normalized trace class operators $\rho \in \Tcl(\mcH), \norm[1]\rho = 1$ whose associated linear functional $\langle \rho,\argdot\rangle$ is non-negative, i.e., $\forall A \geq 0\itholds \langle \rho,A\rangle \geq 0$, form the convex set $\Qst(\mcH)$ of \emph{(quantum) states} or \emph{density operators}.
As we will see later in Section~\ref{sec:measurementsandcompletelypositivmaps}, the quantum state of a system encodes all the information about a system that is necessary to predict the probabilities of the outcomes of all measurements that can be performed on it.

It turns out that in the finite dimensional setting considered here $\Qst(\mcH) \subset \Obs(\mcH)$ is the convex set of self-adjoint, non-negative operators with unit trace.
We call the state $\1 / d \in \Qst(\mcH)$ the \emph{maximally mixed state}.
The extreme points of $\Qst(\mcH)$ are rank one projectors of the form $\ketbra\psi\psi$ and are called \emph{pure states}.
Up to a complex phase they are in one to one correspondence with state vectors.
Given a state vector $\ket\psi\in\mcH$ we will sometimes use the short hand notation $\psi \coloneqq \ketbra\psi\psi$ for the associated quantum state.

Given a bounded operator $A \in \Bop(\mcH)$ and a state $\rho \in \Qst(\mcH)$, the \emph{expectation value} of $A$ in state $\rho$ is defined as
\begin{equation} \label{eq:expectationvalue}
 \ex A \rho \coloneqq \Tr(A\,\rho).
\end{equation}
This expression is most useful in the case where $A$ is an observable, i.e., $A \in \Obs(\mcH)$, to express the expectation value of the observable in successive measurements on independently and identically prepared quantum systems.
We will see what measuring an observable means in the next section.


\subsection{Measurements and completely positive maps}
\label{sec:measurementsandcompletelypositivmaps}
%
Quantum mechanics is an \emph{operational} theory, i.e., its formalism includes mathematical objects that describe what can be done with a quantum system prepared in a certain state.
The actions on a system can be either \emph{measurements} or so-called \emph{(quantum) operations}.

The most general \emph{measurements} possible in quantum mechanics are so-called \emph{positive operator valued measurements} (POVMs) \cite{nielsenchuang}.
A POVM with $K$ measurement outcomes is a sequence $M = (M_k)_{k=1}^K$ of operators $M_k \in \Bop(\mcH)$, called \emph{POVM elements}, with the property that
\begin{equation}
  \sum_{k=1}^K M_k = \1 .
\end{equation}
Upon measuring a system in state $\rho \in \Qst(\mcH)$ with the POVM $M$, outcome number $k$ is obtained with probability $\Tr(M_k\,\rho)$.
When we say that an observable $A \in \Obs(\mcH)$, with spectral decomposition $A = \sum_{k=1}^{d'} a_k\,\Pi_k$, is \emph{measured}, we mean that the POVM $M = (\Pi_k)_{k=1}^{d'}$ is measured and the measurement device outputs the value $a_k$ when outcome $k$ is obtained.
The average value output by the device in measurements of identically prepared systems is then indeed given by \texteqref{eq:expectationvalue}.
A measurement of a POVM where all the POVM elements are projectors is called a \emph{projective measurement}.
The \emph{measurement statistic} of a POVM in a state $\rho$ is the vector of probabilities $\Tr(M_k\,\rho)$.
% In turn every POVM can be interpreted as a projective measurement, i.e., of an ordinary observable, on a composite system formed by the system of interest and an ancilla system \cite{nielsenchuang}.

The most general \emph{(quantum) operations} in quantum mechanics are captured by so-called \emph{completely positive trace preserving maps}, also-called \emph{quantum channels} \cite{nielsenchuang}.
We call maps $\Bop(\mcH)\to\Bop(\mcH)$ \emph{superoperators}.
We denote the identity superoperator by $\id\oftype\Bop(\mcH)\to\Bop(\mcH)$.
A linear map $\Chann\oftype\Obs(\mcH)\to\Obs(\mcH)$ is then called \emph{completely positive trace preserving} if for all separable Hilbert spaces $\mcH'$ it holds that
\begin{equation} \label{eq:conditionforcompletepositivity}
  \forall \rho \in \Qst(\mcH \otimes \mcH')\itholds (\Chann \otimes \id)\,\rho \in \Qst(\mcH \otimes \mcH') .
\end{equation}
In the finite dimensional setting considered here, it turns out that fixing $\mcH' = \mcH$ in \texteqref{eq:conditionforcompletepositivity} already gives a necessary and sufficient condition for a map $\Chann\oftype\Obs(\mcH)\to\Obs(\mcH)$ to be completely positive trace preserving \cite{nielsenchuang}.
We denote the set of all completely positive trace preserving maps on $\Qst(\mcH)$ by $\Qch(\mcH)$.

A particularly important kind of quantum operation is \emph{time evolution} under a Hamiltonian (more on that in Section~\ref{sec:timeevolution}).


\subsection{Norms, distance measures and distinguishability}
\label{sec:normsanddistancemeasures}
%
The natural norm for observables is the operator norm $\norm[\infty]\argdot$.
A useful family of further norms are the \emph{unitary invariant norms}, i.e., norms $\norm[uinv]\argdot$ with the property that
\begin{equation}
  \forall U \in U(d)\itholds \norm[uinv]{\argdot} = \norm[uinv]{U\,\argdot U\ad} ,
\end{equation}
and a particularly useful subclass thereof are the \emph{(Schatten) $p$-norms}.
For every $1\leq p<\infty$ the \emph{Schatten $p$-norm} of an operator $A \in \Bop(\mcH)$ is defined as \cite{bhatia}
\begin{equation} \label{eq:schattenpnorms}
  \norm[p]A \coloneqq \left[\sum_{j=1}^d (s_j(A))^p \right]^{1/p} ,
\end{equation}
where $(s_j(A))_{j=1}^d$ is the ordered, i.e., $s_1(A) \geq \dots \geq s_d(A)$, sequence of non-negative, real \emph{singular values} of $A$.
If the operator $A$ is self-adjoint and non-degenerate, then the sequence of its singular values is equal to the sequence of the moduli of its eigenvalues.
For $p=1$, $p=2$, and $p\to\infty$ definition \texteqref{eq:schattenpnorms} is consistent with \texteqref{eq:onenorm}, \texteqref{eq:twonorm}, and \texteqref{eq:operatornorm}.
The Schatten $p$-norms are ordered in the sense that \cite{bhatia}
\begin{equation}
  \forall A\in\Bop(\mcH)\colon \norm[p]A \leq \norm[p']A \iff p\geq p'
\end{equation}
and in the converse direction the following inequalities hold \cite{bhatia}
\begin{equation}
  \norm[1]\argdot \leq \sqrt{d} \norm[2]\argdot \leq d \norm[\infty]\argdot .
\end{equation}

For quantum states a natural and frequently used distance measure is the \emph{trace distance} \cite{nielsenchuang}
\begin{equation}
  \forall \rho,\sigma\in\Qst(\mcH)\itholds \tracedistance\rho\sigma \coloneqq \frac{1}{2}\,\norm[1]{\rho-\sigma} .
\end{equation}
It is, up to the factor of $1/2$, the metric induced by the Schatten 1-norm $\norm[1]\argdot$ (see \texteqref{eq:onenorm}).
Its relevance stems from the fact that it is equal to the maximal difference between the expectation values of all normalized observables in the states $\rho$ and $\sigma$, i.e., \cite{nielsenchuang}
\begin{equation} \label{eq:tracedistanceasmaxoverobservables}
  \tracedistance\rho\sigma = \max_{A\in\Obs(\mcH)\colon 0\leq A\leq\1} \Tr(A\,\rho) - \Tr(A\,\sigma) .
\end{equation}
Moreover, if one is given an unknown quantum system and is promised that with probability $1/2$ it is either in state $\rho$ or state $\sigma$, then the maximal achievable probability $p_{\max}$ for correctly identifying the state after a single measurement of the optimal observable from \texteqref{eq:tracedistanceasmaxoverobservables} is given by \cite{1012.4622v1,Aubrun2013}
\begin{equation} \label{eq:maxsucessprobabilityfordistinguishinginsingelshotmeasurement}
  p_{\max} = \frac{1+\tracedistance\rho\sigma}{2} .
\end{equation}
Inspired by this, one can define the \emph{distinguishability} of two quantum states under a restricted set $\POVMs$ of POVMs.
The optimal success probability for single shot state discrimination is then again given by an expression of the form \eqref{eq:maxsucessprobabilityfordistinguishinginsingelshotmeasurement}, but with $\tracedistance\rho\sigma$ replaced by \cite{1110.5759v1}
\begin{equation} \label{eq:distinguishabilityunderrestrictedsetsofpovms}
  \tracedistance[\POVMs]\rho\sigma \coloneqq \sup_{M \in \POVMs} \frac{1}{2}\,\sum_{k=1}^{|M|} |\Tr(M_k\,\rho) - \Tr(M_k\,\sigma)| ,
\end{equation}
and it holds that
\begin{equation} \label{eq:tracedistanceboundsrestrictedtracedistance}
 \tracedistance[\POVMs]\rho\sigma \leq \tracedistance\rho\sigma .
\end{equation}
with equality for all $\rho,\sigma \in \Qst(\mcH)$ if and only if $\POVMs$ is a dense subset of the set of all POVMs \cite{1110.5759v1}.
It is worth noting that $\tracedistance[\POVMs]\argdot\argdot$ is a pseudometric on $\Qst(\mcH)$, i.e., it is a symmetric, positive semidefinite bilinear form, but $\tracedistance[\POVMs]\rho\sigma = 0 \notimplies \rho = \sigma$.
For further properties of the distinguishability $\tracedistance[\POVMs]{}{}$ see for example Ref.~\cite{Aubrun2013}.


\subsection{Entropy}
\label{sec:entropy}
%
An important quantity in quantum information theory is the \emph{von~Neumann entropy}
\begin{equation} \label{eq:vonneumannentropy}
  \Svn(\rho) \coloneqq - \Tr(\rho \log_2 \rho ) .
\end{equation}
It is a generalization of the \emph{Shannon entropy} for classical probability distributions and it has many desirable features \cite{thirringquantu}.
The von~Neumann entropy has many applications in quantum information theory \cite{nielsenchuang}, but there is no a priori reason to believe that it is the, or a, thermodynamic entropy.
Nevertheless, it will feature in some of the results that we will discuss later.


\subsection{Time evolution}
\label{sec:timeevolution}
%
A physically particularly important type of quantum operation is time evolution of a \emph{closed system} under a \emph{Hamiltonian}.
This type of time evolution is often described in either of two equivalent formulations known as the \emph{Schrödinger picture} and the \emph{Heisenberg picture}.
In the former, the state of a quantum system is considered to be time dependent and the observables are time independent.
In the latter, the observables are evolved backwards in time from the time of the measurement to the beginning of the time evolution (for more details see for example Ref.~\cite{Kliesch2013}).
We will mostly work in the Schrödinger picture.

The (time independent) \emph{Hamiltonian} $\H \in \Obs(\mcH)$ of a finite dimensional quantum system has the spectral decomposition
\begin{equation} \label{eq:hamiltonianspectraldecomposition}
  \H = \sum_{k=1}^{d'} E_k\,\Pi_k
\end{equation}
where the $\Pi_k \in \Obs(\mcH)$ are its orthogonal (and mutually orthogonal) \emph{spectral projectors} and $d' \coloneqq |\spec(\H)| \leq d = \dim(\mcH)$ is the number of distinct, ordered \emph{(energy) eigenvalues} $E_k \in \R$ of $\H$, i.e., $k<l \implies E_k < E_l$.
The subspaces on which the $\Pi_k$ project are called \emph{(energy) eigenspaces} or \emph{energy levels}.
If $\H$ is non-degenerate it holds that $\Pi_k = \ketbra{E_k}{E_k}$ with $(\ket{E_k})_{k=1}^d$ a sequence of orthonormal \emph{energy eigenstates} of $\H$ and $d \coloneqq \dim(\mcH)$ the dimension of $\mcH$.
If $\H$ has degeneracies the energy eigenstates are not unique (not even up to a phase), but a basis of orthonormal energy eigenstates can still be constructed by choosing an arbitrary orthonormal basis in each energy eigenspace.

The Hamiltonian $\H$ governs the \emph{time evolution} $\rho \colon \R \to \Qst(\mcH)$ of the state of a quantum system via the \emph{(Schrödinger-)von-Neumann-equation}, which in the Schrödinger picture reads
\begin{equation}
  \frac{\del}{\del t} \rho(t) = - \i [\H,\rho(t)] .
\end{equation}
Its formal solution can be given in terms of the \emph{time evolution operator}, which in the case of time independent Hamiltonian dynamics is given by the operator exponential
\begin{equation} \label{eq:timeevolution}
  \forall t\in\R\itholds U(t) \coloneqq \e^{-\i\,\H\,t} \in \Bop(\mcH) .
\end{equation}
The time evolved quantum state at time $t$ is then
\begin{equation}
  \rho(t) \coloneqq U^\dagger(t)\,\rho(0)\,U(t) ,
\end{equation}
with $\rho(0)$ the \emph{initial state} at time $t=0$.

The temporal evolution of the expectation value of an observable $A \in \Obs(\mcH)$ then solves
\begin{equation} \label{eq:timeevolutionofexpectationvalueinschroedingerpicture}
  \ex{A}{\rho(t)} = \Tr[A\,U^\dagger(t)\,\rho(0)\,U(t)] = \Tr[U(t)\,A\,U^\dagger(t)\,\rho(0)] .
\end{equation}
One can thus equally well-define the time evolution of an observable $A\oftype \R \to \Obs(\mcH)$, with the initial value $A(0)$ given by the operator $A$ from \texteqref{eq:timeevolutionofexpectationvalueinschroedingerpicture}, by setting $A(t) \coloneqq U(t)\,A(0)\,U^\dagger(t)$, and consider a fixed quantum state $\rho \in \Qst(\mcH)$, equal to the initial state $\rho(0)$ in \texteqref{eq:timeevolutionofexpectationvalueinschroedingerpicture}.
Then $\ex{A(t)}{\rho}$ is equal to $\ex{A}{\rho(t)}$ from \texteqref{eq:timeevolutionofexpectationvalueinschroedingerpicture} for all $t \in \R$.

The time evolution $A\oftype \R \to \Obs(\mcH)$ of an observable in the Heisenberg picture solves the differential equation
\begin{equation}\label{eq:schroedingervonneumannequationinheisenbergpicture}
  \frac{\del}{\del t} A(t) = \i [\H,A(t)] .
\end{equation}

We call all observables $A \in \Obs(\mcH)$ that commute with the Hamiltonian, i.e., for which $[\H,A] = 0$, \emph{conserved quantities}.
It follows directly from \texteqref{eq:schroedingervonneumannequationinheisenbergpicture} that the expectation value of all conserved quantities is independent of time, irrespective of the initial state, which justifies the name.
If the Hamiltonian $\H$ is non-degenerate, then exactly the observables that are diagonal in the same basis as $\H$ are conserved quantities.
In the presence of degeneracies exactly the observables $A \in \Obs(\mcH)$ for which some basis exists in which both $A$ and $\H$ are diagonal are conserved quantities.

% In more general scenarios, for example if the Hamiltonian is itself time dependent, or the system under consideration is not closed, the backward time evolution in the Heisenberg picture is more involved \cite{Kliesch2013}.

% The temporal evolution of all physical, i.e., in principle measurable, quantities is invariant under a shift of all the eigenvalues of the Hamiltonian.
% We can thus, without loss of generality, assume that the Hamiltonian under consideration is either traceless, or non-negative, depending on what is more convenient.
% This freedom can be used to tune some of the bounds of the following theorems.


\subsection{Time averages and dephasing}
\label{sec:timeaverages}
%
Given a function $f$ depending on time, we define its \emph{finite time average}
\begin{equation} \label{eq:generalfinitetimeaverage}
  \taverage[T]{f} \coloneqq \frac{1}{T}\,\int_0^T f(t) ,
\end{equation}
and its \emph{(infinite) time average}
\begin{equation} \label{eq:generalinfinitetimeaverage}
  \taverage{f} \coloneqq \lim_{T\to\infty} \taverage[T]{f} ,
\end{equation}
whenever the limit exists.
In all cases we will be interested in, the existence of the limit in \texteqref{eq:generalinfinitetimeaverage} is guaranteed by the theory of \emph{(Besicovitch) almost-periodic} functions \cite{Besicovitch1926}.

In particular we will encounter the \emph{time averaged state} $\omega \coloneqq \taverage{\rho}$, which is, in the finite dimensional case considered here, equal to the initial state $\rho(0)$ \emph{dephased} with respect to the Hamiltonian $\H$, i.e., $\omega = \$_\H(\rho(0))$, with the \emph{dephasing map}
\begin{equation}
  \$_\H(\argdot) \coloneqq \sum_{k=1}^{d'} \Pi_k \argdot \Pi_k
\end{equation}
and $(\Pi_k)_{k=1}^{d'}$ the sequence of orthogonal spectral projectors of $\H$ (see also Section~\ref{sec:aquantummaximumentropyprinciple}).


\subsection{Composite quantum systems and reduced states}
\label{sec:localquantumsystems}
%
We will encounter systems consisting of smaller subsystems.
Often their Hamiltonian can be written as a sum of Hamiltonians that each act non-trivially only on certain subsets of the whole system.
We will refer to such systems as \emph{composite (quantum) systems} or as \emph{locally interacting (quantum) systems}, depending on whether we want to stress that they consist of multiple parts or that the interaction between the parts has a special structure.
If a composite system consists of two or three subsystems, then we will also call it a \emph{bipartite} or \emph{tripartite} system.
Of course this does not exclude that those subsystems are themselves composite systems.

The notion of locally interacting quantum systems can be formalized by means of an \emph{interaction (hyper)graph} $\mcG \coloneqq (\Vset, \Eset)$, which is a pair of a \emph{vertex set} $\Vset$ and an \emph{edge set} $\Eset$.
In the following we will explain in detail what the vertex and edge sets are and how the Hilbert spaces of a composite system can be constructed.

The \emph{vertex set} $\Vset$ is the set of indices labeling the sites of the system and we will work under the assumption that $|\Vset| < \infty$.
The Hilbert space $\mcH$ of such a system is either, in the case of spin systems, the \emph{tensor product} $\bigotimes_{x\in\Vset} \mcH_{\{x\}}$ of the Hilbert spaces $\mcH_{\{x\}}$ of the individual sites $x \in \Vset$, or, in the case of fermionic or bosonic systems, the \emph{Fock space}, or a subspace of the latter.

We will encounter bosons, which usually need to be described using infinite dimensional Hilbert spaces, only in Section~\ref{sec:equlibrationinthestrongsense}, hence we want to avoid the technicalities of a proper treatment of infinite dimensional Hilbert spaces and unbounded operators in the framework of functional analysis.
We will thus only introduce the minimal notation necessary to formulate the statements we will discuss in Section~\ref{sec:equlibrationinthestrongsense}.

The sites $x \in \Vset$ of fermionic and bosonic composite systems are often called \emph{modes}.
In the case of fermions each mode is equipped with the Hilbert space $\mcH^f_{\{x\}} = \C^2$ with orthonormal basis $((\ket n_f)_{n=0}^1$, and in the case of bosons with the Hilbert space $\mcH^b_{\{x\}} = \ltwo$ of square summable sequences with orthonormal basis $(\ket n_b)_{n=0}^\infty$.

Bosons and fermions are two kinds of \emph{indistinguishable} particles that occur in nature.
The indistinguishability has profound implications for their statistical properties and their mathematical description.
The Hilbert space of a composite system of fermions/bosons is not the tensor product of the Hilbert spaces of the modes.
For composite systems with exactly $N$ fermions or bosons in $M$ modes, i.e., $\Vset = [M]$, the Hilbert space is given by a so-called \emph{Fock layer}.
The Fock layer to particle number $N$ is the complex span of the orthonormal \emph{Fock (basis) states} $\ket{n_1,\dots,n_M}_f$ or $\ket{n_1,\dots,n_M}_b$ respectively, where for each $x \in \Vset$, $n_x$ is the number of particles in mode $x$ and thus $\sum_{x \in \Vset} n_x = N$ with $n_x \in \{0,1\}$ in the case of fermions, and $n_x \in [N]$ in the case of bosons.

The full \emph{Fock space} of a system of fermions or bosons is the Hilbert space completion of the direct sum of the Fock layers for each possible total particle number.
For fermions it holds that $N \leq M$ due to the \emph{Pauli exclusion principle}, and the resulting Hilbert space is hence finite dimensional.
In the case of bosons $N$ is independent of $M$ and the Fock space is thus infinite dimensional already for a finite number of modes.

We define the fermionic and bosonic \emph{annihilation operators} $f_x$ and $b_x$ on site $x$ and the corresponding \emph{creation operators} $f\ad_x$ and $b\ad_x$ (collectively often referred to as simply the \emph{fermionic/bosonic operators}) via their action on the Fock basis states given by
\begin{align}
  f_x \ket{n_1,\dots,n_M}_f &=  n_x (-1)^{\sum_{y=1}^{x-1} n_y} \ket{\dots,n_{x_1},n_x - 1,n_{x+1},\dots}_f \label{eq:fermionicannihilationoperator}\\
  f\ad_x \ket{n_1,\dots,n_M}_f &=  (1-n_x) (-1)^{\sum_{y=1}^{x-1} n_y} \ket{\dots,n_{x_1},n_x + 1,n_{x+1},\dots}_f \\
  \intertext{and}
  b_x \ket{n_1,\dots,n_M}_b &=  \sqrt{n_x} \ket{n_1,\dots,n_{x_1},n_x - 1,n_{x+1},\dots,n_M}_b \label{eq:bosonicannihilationoperator}\\
  b\ad_x \ket{n_1,\dots,n_M}_b &=  \sqrt{n_x+1} \ket{n_1,\dots,n_{x_1},n_x + 1,n_{x+1},\dots,n_M}_b .
\end{align}
They satisfy the \emph{(anti) commutation relations}
\begin{align}
  \{f_x,f_y\} = \{f\ad_x,f\ad_y\} &= 0 & \{f_x,f\ad_y\} &= \delta_{x,y} \\
  [b_x,b_y] = [b\ad_x,b\ad_y] &= 0 & [b_x,b\ad_y] &= \delta_{x,y} . \label{eq:bosonicommutationrelations}
\end{align}
The products $f\ad_x\,f_x$ and $b\ad_x\,b_x$ are called \emph{particle number operators} as
\begin{equation}
  f\ad_x\,f_x \ket{n_1,\dots,n_M}_f = n_x \ket{n_1,\dots,n_M}_f
\end{equation}
and respectively for bosons.

Any operator that commutes with the \emph{total particle number operator} $\sum_{x\in\Vset} f\ad_x\,f_x$ or $\sum_{x\in\Vset}  b\ad_x\,b_x$ respectively is called \emph{particle number preserving}.
In systems with particle number preserving Hamiltonians a constraint on the particle number can be used to make the description of bosonic systems with finite dimensional Hilbert spaces possible.
The Hilbert space is then a finite direct sum of Fock layers.
% The Hilbert space of a fermionic or bosonic system with finitely many modes and an upper bound on the maximal particle number fermions, or bosons respectively.
We say that a state has a \emph{finite particle number} if it is completely contained in such a finite direct sum of Fock layers.

Two comments concerning the mathematical structure are in order:
First, for fermions $f\ad_x$ is indeed the adjoint of $f_x$ in the sense of \texteqref{eq:defadjoint}.
Second, strictly speaking, for bosons expressions such as those in \texteqref{eq:bosonicommutationrelations} are in the context of this work not rigorously well-defined.
The operators $b_x$ and $b\ad_x$ are \emph{unbounded operators} and we have not introduced the mathematical machinery necessary for a proper treatment of such operators.
Equations such as those in \texteqref{eq:bosonicommutationrelations} in the current work are to be seen as \emph{formal expressions}.
For example, the action of $[b_x,b_y]$ on a Fock basis state can be calculated using the relations in \texteqref{eq:bosonicannihilationoperator}.
To avoid the difficulties of a proper treatment of unbounded operators we will in the following only consider the identity operator and operators that can be written as polynomials in the bosonic operators, whenever we talk about systems of bosons.

In systems of fermions, all operators can be written as polynomials of the fermionic operators.
A polynomial of fermionic operators is called \emph{even/odd} if it can be written as a linear combination of monomials that are each a product of an even/odd number of creation and annihilation operators.
According to the \emph{fermion number parity superselection rule} \cite{Banuls2009}, only observables that are even polynomials in the fermionic operators can occur in nature.
The same holds for the Hamiltonians and density matrices of such systems.
Consequently, whenever we make statements about systems of fermions we assume that all observables, states and the Hamiltonian are even.

We refer to subsets of the vertex set $\Vset$ as \emph{subsystems}.
Generalizing the notation introduced for the Hilbert spaces of the individual sites we denote the Hilbert spaces associated with a subsystem $X \subseteq \Vset$ by $\mcH_X$ and its dimension by $d_X\coloneqq\dim(\mcH_X)$.
In the case of composite systems of fermions or bosons it is understood that if an upper bound on the total number of particles has been imposed, then $\mcH_X$ is taken to be the direct sum of Fock layers corresponding to the sites in $X$ up to the total number of particles.
The \emph{size} of a (sub)system $X \subseteq \Vset$ is given by the number of sites or modes $|X|$, not the dimension of the corresponding Hilbert space.

For spin systems we define the \emph{support} $\supp(A)$ of an operator $A \in \Bop(\mcH)$ as the smallest subset of $\Vset$ such that $A$ acts like the identity outside of $X$.
For systems of fermions or bosons we define the support of an operator via its representation as a polynomial in the respective creation and annihilation operators.
The \emph{support} is then the set of all site indices $x\in\Vset$ for which the polynomial contains a fermionic or bosonic operator acting on site $x$, e.g., $b_x\ad$ or $f_x$.
The support of a POVMs is simply the union of the supports of its POVM elements.
Similarly, we define the \emph{support} $\supp(\Chann)$ of a superoperator $\Chann\oftype\Bop(\mcH)\to\Bop(\mcH)$ as the smallest subset of $\Vset$ such that
\begin{equation}
  \forall A \in \Bop(\mcH) \itholds \supp(A) \subseteq \compl{\supp(\Chann)} \implies \Chann(A) = A .
\end{equation}
We say that an observable, POVM, or superoperator is \emph{local} if the size of its support is small compared to and/or independent of the system size.

In order to fully exploit the notion of a subsystem we need to understand how the description of a joint system fits together with the description of a subsystem as an isolated system, i.e., how systems can be combined and decomposed.
For every subsystem $X \subseteq \Vset$ there is a \emph{canonical embedding} of $\Bop(\mcH_X)$ into $\Bop(\mcH)$ that bijectively maps $\Bop(\mcH_X)$ onto the subalgebra of bounded linear operators $A \in \Bop(\mcH)$ with $\supp(A) \subseteq X$, and similarly for all operators that are polynomials of bosonic operators.
In the case of spin systems the embedding is simply the natural embedding $A \in \Bop(\mcH_X) \mapsto A \otimes \1_{\compl X} \in \Bop(\mcH)$, where $\1_{\compl X}$ denotes the identity operator on $\mcH_{\compl X}$.
In systems of fermions or bosons we associate to each operator on $\mcH_X$ the operator on $\mcH$ that has the same representation as a polynomial in the fermionic/bosonic operators, but, of course, in terms of the fermionic/bosonic operators of the fully system with Fock space $\mcH$ rather than the fermionic/bosonic operators that act on $\mcH_X$.
For systems of fermions, because of the phase in \texteqref{eq:fermionicannihilationoperator} that depends non-locally on the state, this embedding depends on the exact position the sites in $X$ have in the vertex set $\Vset$.
The vertex set should hence rather be called \emph{vertex sequence}, but for even operators the phases cancel out, which is why we ignore this subtlety.

Conversely, for any $A \in \Bop(\mcH)$ and any subsystem $X \subseteq \Vset\colon X \supseteq \supp(A)$ that contains $\supp(A)$ we define the \emph{truncation} $\trunc A X \in \Bop(\mcH_X)$ of $A$ as the operator that acts on the sites/modes in the subsystem $X$ ``in the same way'' as $A$, in the sense that a truncation followed by a canonical embedding gives back the original operator.
In particular, for spin systems any $A \in \Bop(\mcH)$ is of the form $A = \trunc A {\supp(A)} \otimes \1_{\compl{\supp(A)}}$.
For general systems, the identity operator $\1$ of course satisfies $\1_X = \trunc \1 X$ for any $X \subset \Vset$.

We now turn to the edge set.
The \emph{edge set} $\Eset$ is the set of all subsystems $X \subset \Vset$ for which a non-trivial Hamiltonian term $\H_X$ with $\supp(\H_X) = X$ exists that couples the sites in $X$.
%Without loss of generality we can assume that $\Eset$ does not contain subsets that are completely contained in another subset in $\Eset$, i.e., that $\forall X \in \Eset \itholds \nexists\+ Y \in \Eset \suchthat Y \neq X \land X \subset Y$.

The Hamiltonian of a locally interacting quantum system with edge set $\Eset$ is of the form
\begin{equation} \label{eq:localhamiltonian}
  \H = \sum_{X \in \Eset} \H_X ,
\end{equation}
with $\supp(\H_X) = X$ for all $X \in \Eset$.
Generalizing this notation to subsystems $X \subset \Vset$ that are not in $\Eset$ we define for any subsystem $X \subset \Vset$ the \emph{restricted Hamiltonian}
\begin{equation} \label{eq:restrictedhamiltonian}
  \H_X \coloneqq \sum_{Y \in \Eset\colon Y\subseteq X} \H_Y  \in \Obs(\mcH) ,
\end{equation}
which obviously fulfills $\supp(\H_X) \subseteq X$.
Note that we adopt the convention that $\H_X$ is an element of $\Obs(\mcH)$ and not of $\Obs(\mcH_X)$.

We will also need the \emph{graph distance}.
In order to define it, we first need to give a precise meaning to a couple of intuitive terms:
We say that two subsystems $X,Y \subset \Vset$ \emph{overlap} if $X \cap Y \neq \emptyset$,
a set $X \subset \Vset$ and a set $F\subset \Eset$ \emph{overlap} if $F$ contains an edge that overlaps with $X$, and two sets $F,F'\subset \Eset$ \emph{overlap} if $F$ overlaps with any of the edges in $F'$.
A subset $F \subset \Eset$ of the edge set \emph{connects} $X$ and $Y$ if $F$ contains all elements of some sequence of pairwise overlapping edges such that the first overlaps with $X$ and the last overlaps with $Y$ and similarly for sites $x,y \in V$.

The \emph{(graph) distance} $\dist(X,Y)$ of two subsets $X,Y\subset \Vset$ with respect to the (hyper)graph $(\Vset,\Eset)$ is zero if $X$ and $Y$ overlap and otherwise equal to the size of the smallest subset of $\Eset$ that connects $X$ and $Y$.
The \emph{diameter} of a set $F \subset \Eset$ is the largest graph distance between any two sets $X,Y \in F$.
We extend the definition of the graph distance to operators $A,B \in \Bop$ and set $\dist(A,B) \coloneqq \dist(\supp(A), \supp(B))$.

A short side remark:
We have now introduced for any subsystem $X \subset \Vset$ the notation $\mcH_X$ for the subsystem Hilbert space, $\H_X$ for the restricted Hamiltonian, and $\trunc A X$ for the truncation of an operator.
Slightly abusing this notation we will later also give special meaning to symbols such as $\H_0$, $\H_I$ or $\mcH_R$ where the subscript is not a subset of the vertex set $\Vset$.

We will also make use of the notion of \emph{reduced states}, or \emph{marginals}.
Given a quantum state $\rho \in \Qst(\mcH)$ of a composite system with subsystem $X \subset \Vset$ we write $\rho^X$ for the \emph{reduced state} on $X$, which is defined as the unique quantum state $\rho^X \in \Qst(\mcH_X)$ with the property that for any observable $A \in \Obs(\mcH)$ with $\supp(A) \subseteq X$
\begin{equation} \label{eq:reducedstate}
  \Tr(\trunc A X \,\rho^X) = \Tr(A\,\rho) .
\end{equation}
Defining the reduced state in systems of fermions in this way is important to avoid ambiguities \cite{Friis2013}.
We will denote the map linear $\rho \mapsto \rho^X$ by $\Tr_{\compl{X}}$.
As $\Tr_{\compl{X}}$ is linear can we naturally extend its domain to all of $\Bop(\mcH)$ so that
\begin{equation} \label{eq:partialtrace}
  \Tr_{\compl{X}} \colon \Bop(\mcH) \to \Bop(\mcH_X) .
\end{equation}
In the case of spin systems $\Tr_{\compl{X}}$ is indeed the \emph{partial trace} over $\compl X = \Vset\setminus X$ as defined for example in Ref.~\cite{nielsenchuang}.
For time evolutions $\rho \colon \R \to \Qst(\mcH)$ we use the natural generalization of the superscript notation, i.e., $\rho^X = \Tr_{\compl{X}} \circ \mathop\rho \colon \R \to \Qst(\mcH_X)$.


\subsection{Correlations and entanglement}
\label{sec:correlationsandmomemts}
%
Correlations play a central role in the description of composite systems and hence in condensed matter physics and statistical mechanics.
It is beyond the scope of this work to give a comprehensive overview of the different types and measures of correlations (see for example Refs.~\cite{Kastoryano2011,Kastoryano2013,nielsenchuang,Plenio07}).

An important measure of correlation is the \emph{covariance}, which for a quantum state $\rho \in \Qst(\mcH)$ and two operators $A,B \in \Bop(\mcH)$ is defined to be
\begin{equation} \label{eq:covariance}
 \cov_\rho(A,B) \coloneqq \Tr(\rho\,A\,B) - \Tr(\rho\, A) \Tr(\rho \, B) .
\end{equation}
It satisfies
\begin{equation}
 |\cov_\rho(A,B)| \leq \sqrt{\ex{A^2}\rho \, \ex{B^2}\rho}
\end{equation}
and hence one defines the \emph{correlation coefficient} as $\cov_\rho(A,B) / \sqrt{\ex{A^2}\rho \, \ex{B^2}\rho}$.
We will encounter a slightly generalized version of the covariance in Section~\ref{sec:propertiesofthermalstatesofcompositesystems}.

For a given state $\rho \in \Qst(\mcH)$, if the outcomes of a measurement of two observables $A,B \in \Obs(\mcH)$ are considered as random variables, then $\cov_\rho(A,B) = 0$ if the random variables are independent.
The converse, however, is not necessarily true.

The covariance is most interesting as a correlation measure if $A$ and $B$ act on disjoint subsystems, i.e., $\supp(A) \cap \supp(B) = \emptyset$.
If for a given state $\rho \in \Qst(\mcH)$ of a bipartite system with $\Vset = X \dunion Y$ and any two observables $A,B \in \Obs(\mcH)$ with $\supp A \subseteq X$ and $\supp B \subseteq Y$ it holds that $\cov_\rho(A,B) = 0$, then we say that $\rho$ is \emph{uncorrelated} with respect to the bipartition $\Vset = X \dunion Y$.

Uncorrelated states of spin systems are \emph{product states}.
Consider a bipartite spin system with Hilbert space $\H$ and vertex set $\Vset = X \dunion Y$.
A quantum state $\rho \in \Qst(\mcH)$ is said to be \emph{product} with respect to this bipartition if $\rho = \rho^X \otimes \rho^Y$.
The claimed equivalence can be seen as follows:
Given a state $\rho \in \Qst(\mcH)$, $\cov_\rho(A,B) = 0$ implies that $\Tr(A\,B\, (\rho^X \otimes \rho^Y) ) = \Tr(A\,B\,\rho).$ If this holds for all $A,B \in \Obs(\mcH)$, then necessarily $\rho^X \otimes \rho^Y = \rho$.
We call a basis that consists entirely of product states a \emph{product basis}.

Still in the setting of a bipartite spin system with Hilbert space $\H$ and vertex set $\Vset = X \dunion Y$, all quantum states of the form
\begin{equation}
  \rho = \sum_j p_j\, \rho_j^X \otimes \rho_j^Y
\end{equation}
with $(p_j)_j$ a \emph{probability vector}, i.e., $\sum_j p_j = 1$ and $p_j \geq 0$ for all $j$, and $\rho_j^X \in \Qst(\mcH_X)$ and $\rho_j^Y \in \Qst(\mcH_Y)$ for all $j$, are called \emph{separable} or \emph{classically correlated} with respect to the bipartition $\Vset = X \dunion Y$.

In spin systems all states that are not separable are \emph{entangled}.
There is a plethora of measures of entanglement \cite{Plenio07} and in multipartite scenarios even multiple inequivalent forms of entanglement exist \cite{Horodecki2009}.
In fermionic systems the situation is even less clear \cite{Banuls2007}.

One particular measure of entanglement that we will encounter in Section~\ref{sec:absenceofthermalization} is the \emph{geometric measure of entanglement}.
For a state vector $\ket\psi \in \mcH$ of a bipartite spin system with $\Vset = X \dunion Y$ the geometric measure of entanglement is defined as \cite{Shimony95,Barnum2001,Wei2003}
\begin{equation} \label{eq:geometricmeasureofentanglementdefinition}
  \geometricentanglement_{X|Y}(\ket\psi) \coloneqq 1 - \sup_{\ket{\phi_X} \in \mcH_X,\ket{\psi_Y} \in \mcH_Y} |\bra{\psi} (\ket{\phi_X} \otimes \ket{\phi_Y})|^2
\end{equation}
As for pure state vectors $\ket\psi,\ket\phi \in \mcH$ it holds that $1 - |\braket\psi\phi|^2 = \tracedistance{\psi}{\phi}^2$ \cite[Section~9.2.2 and 9.2.3]{nielsenchuang} the geometric measure of entanglement is a measure for how far a given pure state is from the closest pure product state in trace distance, hence the name.


\subsection{Gibbs states}
\label{sec:gibbsstates}
%
In quantum statistical mechanics a particularly important class of states are so-called \emph{thermal}, \emph{canonical}, or \emph{Gibbs states}.
Essentially they are the quantum version of the \emph{canonical ensemble}.
The Gibbs state of a system with Hilbert space $\mcH$ and Hamiltonian $\H \in \Obs(\mcH)$ at inverse temperature $\beta \in \R$ is defined as
\begin{equation}
  \rhog[\H](\beta) \coloneqq \frac{\e^{-\beta\,\H}}{Z[\H](\beta)} \in \Qst(\mcH) ,
\end{equation}
where $Z[\H]$ is the \emph{(canonical) partition function} defined as
\begin{equation}
  Z[\H](\beta) \coloneqq \Tr(\e^{-\beta\,\H}) .
\end{equation}

The Gibbs state has the important property that it is the unique quantum state that maximizes the von~Neumann entropy \eqref{eq:vonneumannentropy} given the expectation value of the Hamiltonian \cite{thirringquantu}.
This is a direct consequence of Schur's lemma \cite{bhatia} and the fact that the same statement holds in classical statistical mechanics, as can be seen from a straight forward application of the Lagrange multiplier technique.
In fact, the inverse temperature $\beta$ is nothing but the Lagrange parameter associated with the energy expectation value.

For locally interacting quantum systems with a Hamiltonian $\H \in \Obs(\mcH)$ of the form given in \eqref{eq:localhamiltonian} we adopt the convention that for any subsystem $X \subset \Vset$
\begin{equation}
  \rhog^X[\H](\beta) = \Tr_{\compl{X}}(\rhog[\H](\beta)) \in \Qst(\mcH_X)
\end{equation}
denotes the reduction of the Gibbs state of the full system to the subsystem $X$ (compare \texteqref{eq:reducedstate}), while we write
\begin{equation}
  \rhog_X[\H](\beta) \coloneqq \Tr_{\compl{X}}(\rhog[\H_X](\beta)) = \rhog[\trunc{\H_X} X](\beta) \in \Qst(\mcH_X)
\end{equation}
for the reduced state on $X$ of the Gibbs state of the restricted Hamiltonian $\H_X$, or equivalently the Gibbs state of $\trunc{\H_X} X$  (compare \texteqref{eq:restrictedhamiltonian}).
%For the sake of readability we will sometimes omit arguments specifying the Hamiltonian and simply write for example $\rhog_X(\beta)$ if it is unambiguously clear  which Hamiltonian $\H$ is meant.


\subsection{Microcanonical states}
\label{sec:microcanonicalstates}
%
The microcanonical ensemble in quantum statistical mechanics takes the form of the \emph{microcanonical state}.
Usually one defines the microcanonical ensemble and state with respect to an energy interval $[E,E+\Delta]$.
Here we make the slightly more general definition that will be useful later:
The microcanonical state to any subset $R \subseteq \R$ of the real numbers of a system with Hilbert space $\mcH$ and Hamiltonian $\H \in \Obs(\mcH)$ with spectral decomposition $\H = \sum_{k=1}^{d'} E_k \Pi_k$ is defined as
\begin{equation}
  \rhomc[\H](R) \coloneqq \frac{\sum_{k:E_k \in R} \Pi_k}{Z_{mc}[\H](R)} \in \Qst(\mcH) ,
\end{equation}
where $Z_{mc}[\H]$ is the \emph{microcanonical partition function} defined as
\begin{equation}
  Z_{mc}[\H](R) \coloneqq \Tr(\sum_{k:E_k \in R} \Pi_k) .
\end{equation}


\section{Equilibration}
\label{sec:equilibration}
%
The dynamics of finite dimensional quantum system, as described in the previous section, is recurrent \cite{JChemPhys843,Schulman1978,PhysRevLett.49,PhysRev.107.33,Wallace2013} and time reversal invariant and hence equilibration in the sense of an \emph{H-Theorem} \cite{RevModPhys.27.289} (see Section~\ref{sec:boltzmannshtheorem}) is impossible.
This apparent contradiction between the microscopic theory of quantum mechanics and the thermodynamics behavior observed in nature is one of the big puzzles of physics.

We will see in this section that the unitary time evolution of pure states of such systems does imply in a surprisingly general and natural way that certain time dependent properties of quantum systems do dynamically equilibrate and that hence this apparent contradiction can be resolved to a large extend.

We will concentrate on two notions of equilibration that we will call \emph{equilibration on average} and \emph{equilibration during intervals}.
After an introduction of these two notions in Section~\ref{sec:notionsofequilibration} we will discuss them in detail in Sections~\ref{sec:equlibrationintheweaksense} and \ref{sec:equlibrationinthestrongsense}.
We conclude in Section~\ref{sec:conjectureconcerningequilibration} with a conjecture concerning the relation of transport and equilibration.


\subsection{Notions of equilibration}
\label{sec:notionsofequilibration}
%
In this section we define and compare two notions of equilibration compatible with the recurrent and time reversal invariant nature of unitary quantum dynamics in finite dimensional systems.
These notions will capture the intuition that equilibration means that a quantity, after having been initialized at a non-equilibrium value, evolves towards some value and then stays close to it for an extended amount of time.
At the same time, what we will call \emph{equilibration} is less than what one usually associates with the evolution towards \emph{thermal equilibrium}.
We will define a quantum analog of the latter, call it \emph{thermalization}, and discuss it in detail in Section~\ref{sec:thermalization}.

To keep the definition of equilibration as general as possible we will refer abstractly to \emph{time dependent properties} of quantum systems, by which we mean functions $f\oftype \R\to M$ that map time to some metric space $M$, for example $\R$ or $\mcS(\mcH)$.
The metric will allow us to quantify how close the value of such functions is for different times and in particular how close it is to the time average and ``equilibrium values'' of the function.

Properties that we will be interested in include for example the time evolution of expectation values of individual observables.
We will also encounter \emph{subsystem equilibration}.
In this case the property is the time evolution of the state of the subsystem and the metric the trace distance.
It will also be convenient to speak more generally of the \emph{apparent equilibration of the whole system} with the metric then being the distinguishability under a restricted set of POVMs introduced in \ref{sec:normsanddistancemeasures}.

We will discuss the following two notions of equilibration in more detail:
\begin{description}[font=\normalfont\itshape]
\item[Equilibration on average:]
  We say that a time dependent property \emph{equilibrates on average} if its value is for \emph{most times during the evolution} close to some \emph{equilibrium value}.
\item[Equilibration during intervals:]
We say that a time dependent property \emph{equilibrates during an interval} if its value is close to some \emph{equilibrium value} for \emph{all times during that interval}.
\end{description}

The use of the notion of equilibration on average in the quantum setting goes back to at least the work of von \textcite{vonneumann1929} and has recently been developed further, in particular in Refs.~\cite{tasaki98,Reimann08,Linden09,1110.5759v1,1012.4622v1,Reimann2012,Reimann12}.
We will see that equilibration on average, especially for expectation values of observables as well as for reduced states of small subsystems of large quantum systems, is an extremely generic feature.
In contrast, equilibration during intervals can to date only be proven for few systems with very specific properties \cite{cramer10_1,PhysRevLett.10-5}.

Equilibration on average implies that the equilibrating property spends most of the time during the evolution close to its time average.
This allows for a reasonable definition of an \emph{equilibrium state}, which is then the time averaged or dephased state.
As we will see later in Section~\ref{sec:thermalization}, this makes it possible to tackle the question of \emph{thermalization} in unitarily evolving quantum systems and to make statements about decoherence in Section~\ref{sec:decoherence}.

On the down side, a proof of equilibration on average alone does not immediately imply much about the time scale on which the equilibrium value is reached after a system is started in an out of equilibrium situation.
We will see that even though it is possible to bound these time scales, the bounds obtainable in the general settings considered here are only of very limited physical relevance (see Section~\ref{sec:timescales}).

As we will see in the following, the statements on equilibration during intervals are much more powerful in this respect.
They imply bounds on the time it takes to equilibrate that scale reasonably with the size of the system and hence prove equilibration on experimentally relevant time scales.
On the other hand, in the few settings in which equilibration during intervals of reduced states of subsystems has been proved, it is known that the equilibrium states are not close to thermal states of suitably restricted Hamiltonians.
In particulate, no proof of \emph{thermalization} (in the sens of the word we will defined later in Section~\ref{sec:whatisthermalization}) based on a result on equilibration during intervals is known to date.

We discuss both notions of equilibration in detail in the following two sections.


\subsection{Equilibration on average}
\label{sec:equlibrationintheweaksense}
%
In this section we discuss equilibration on average.
The outline is as follows:
After giving some historic perspective we will go through the main ingredients that feature in the known results on equilibration on average and discuss their roll in the arguments and to what extent they are physically reasonable and mathematically necessary.
After this preparation we will state, prove and interpret the arguably strongest result on equilibration on average known to date.

Already the founding fathers of quantum mechanics realized that the unitary evolution of large, closed quantum systems, together with the immensely high dimension of their Hilbert space and quantum mechanical uncertainty, could possibly explain the phenomenon of equilibration.
Most notable is an article of von~\textcite{vonneumann1929} from 1929, which already contains a lot of the ideas and even variants of some of the results that can be found in the modern literature on the subject.
The renewed interest in the topic of equilibration was to a large extent a consequence of two independent theoretical works by \textcite{Reimann08} and \textcite{Linden09}.
The approach outlined there was then more recently refined and the results gradually strengthened.
Important contributions are in particular Refs.~\cite{Reimann12,1110.5759v1,1012.4622v1}.
Also very noteworthy is the often overlooked earlier work \cite{tasaki98}.

The first fact that plays a prominent roll in the proofs of equilibration on average is the immensely high dimension of the Hilbert space of most many body systems.
As we have seen in Section~\ref{sec:localquantumsystems}, the dimension of the Hilbert space of composite systems grows exponentially with the number of constituents.
What actually matters, of course, is the \emph{number of significantly occupied energy levels}, rather than the number of levels that are in principle available but not populated.
For each $k \in [d']$ we define the occupation $p_k \coloneqq \Tr(\Pi_k\,\rho(0))$ of the $k$-th energy level.
Refs.~\cite{tasaki98,Reimann08} use $\max_k p_k$, the occupation of the most occupied level, to quantify the number of significantly occupied energy levels.
Ref.~\cite{Linden09} uses a quantity called \emph{effective dimension}, denoted by $\deff(\omega)$, which in our notation can be defined as
\begin{equation} \label{eq:effectivedimension}
  \deff(\omega) \coloneqq \frac{1}{\sum_{k=1}^{d'} p_k^2} \geq \frac{1}{\max_k p_k} .
\end{equation}
If the initial state is taken to be an energy eigenstate, the resulting effective dimension is one, while that resulting from a uniform coherent superposition of $\tilde{d}$ energy eigenstates to different energies is $\tilde{d}$.
This justifies the interpretation of $\deff(\omega)$ as a measure of the number of significantly occupied states.
It is also reciprocal to a quantity that is known mostly in the condensed matter literature as \emph{inverse participation ration} \cite{Neuenhahn10} and related to the time average of the \emph{Loschmidt echo} \cite{Levstein98,Campos10}.

There are a number of different ways to argue why it is acceptable to restrict oneself to initial states that populate a large number of energy levels when trying to prove the emergence of thermodynamic behavior from the unitary dynamics of closed systems.
First, one can argue that initial states that only occupy a small subspace of the Hilbert space of a large system behave essentially like small quantum systems and such systems are anyway not expected to behave thermodynamically, but rather show genuine quantum behavior.
Second, one can invoke the inevitable limits to the resolution and precision of experimental equipment to conclude that preparing states that overlap only with a handful of the roughly $2^{10^{23}}$ energy levels of a macroscopic system is impossible, even if we had apparatuses that were many orders of magnitude more precise than the equipment available today \cite{Reimann08,Reimann12}.
Finally, one can also take a more mathematical point of view and use results based on a phenomenon called \emph{measure concentration} \cite{ledoux01,CHATTERJEE07} that guarantees that uniformly random pure states drawn from sufficiently large subspaces of a Hilbert space have, with extremely high probability, an effective dimension with respect to any fixed, sufficiently non-degenerate Hamiltonian that is comparable to the dimension of that subspace \cite{Linden09,Popescu06,Popescu05,Gogolin10-masterthesis} (more on such typicality arguments in Section~\ref{sec:typicality}).
If one is willing to assume that such states are physically natural initial states, this can justify the assumption of a large effective dimension.
We will come back to this in Section~\ref{sec:typicality} where we discuss \emph{typicality}.
For an earlier work that directly mingles typicality arguments and dephasing to derive an equilibration result see also Ref.~\cite{Bocchieri1959}.

As we will see below, it is actually sufficient for equilibration that $\maxprime_k p_k$, the second largest of the energy level occupations, is small.
Note that in the physically relevant situation of a system that is cooled close to its ground state $\maxprime_k p_k$ can be orders of magnitude smaller than $\max_k p_k$ or $1/\deff(\omega)$.
Although the proof of this extension of previous results is not trivial \cite{Reimann12}, the physical intuition behind it is clear:
The expectation values of all observables of a system that is initialized in an energy eigenstate are already in equilibrium.
What can prevent equilibration on average are not macroscopic populations of one energy level, but rather initial states that are coherent superpositions of a small number of energy eigenstates.
Such states can show a behavior reminiscent of Rabi-Oscillations and not exhibit equilibration.

\begin{figure}[bt]
  \centering
  \begin{tikzpicture}[scale=1.2]
    \draw[thick,->] (-1,0) -- (6,0) node[at end,below] {$E$} node[at start,left] {$\spec(\H):$};
    % \foreach \i/\x in {1/0.1,2/0.3,3/1,4/1.5,5/2.5,6/3.1,7/3.2,8/4.5,9/4.8} {\node (n\i) at (\x,-0.1) {}; \draw[thick] (\x,-0.1) -- (\x,0.1);};
    \foreach \i/\x in {1/0.1,2/0.3,3/1,4/1.5,5/2.5,6/3.1,7/3.2,8/4.5,9/4.8} {\draw[thick] (\x,-0.1) -- (\x,0.1) node[at start,scale=0.01] (n\i) {};};
    \draw[decorate,decoration={brace}] (n4) -- (n3) node[midway,below=0.5] (gap1) {} node[at start,above=6] (el) {$E_l$} node[at end,above=6] (ek) {$E_k$};
    \draw[decorate,decoration={brace}] (n8) -- (n7) node[midway,below=0.5] (gap2) {} node[at start,above=6] (en) {$E_n$} node[at end,above=6] (em) {$E_m$};
    \node[anchor=north] (neq) at ($ (gap1) !.5! (gap2) $) {$\neq$};
    \draw[->] (neq) -- (gap1);
    \draw[->] (neq) -- (gap2);
  \end{tikzpicture}
  \caption{Illustration of the non-degenerate energy gaps condition. No gap between two energy levels may occur more than once in the spectrum, but the individual levels may well be degenerate.}
  \label{fig:nondegenerateenergygaps}
\end{figure}

The second main ingredient to the proofs of Refs.~\cite{vonneumann1929,tasaki98,Reimann08,Linden09} is the condition of \emph{non-degenerate energy gaps} originally called the \emph{non-resonance} condition.
We say that a Hamiltonian $\H$ has \emph{non-degenerate energy gaps}, if for every $k,l,m,n \in [d']$
\begin{equation}
  E_k - E_l = E_m - E_n \implies (k = l \land m = n) \lor (k=m \land l=n) ,
\end{equation}
i.e., if every energy gap $E_k - E_l$ appears exactly once in the spectrum of $\H$.
The original condition used in Refs.~\cite{vonneumann1929,tasaki98,Reimann08,Linden09} is stronger and excludes in addition all Hamiltonians with degeneracies, i.e., requires that $d' = d$.
Although the non-degenerate energy gaps condition appears pretty technical at first sight, the motivation for imposing it can be made apparent by the following consideration:
The main concern of Ref.~\cite{Linden09} is the equilibration on average of the reduced state $\rho^S(t)$ of a small subsystem $S$ of a bipartite system with $\Vset = S \dunion B$.
If the Hamiltonian of the composite system is of the form
\begin{equation} \label{eq:hamiltonianwithoutcouplingtodemonstratenondegenerategaps}
  \H = \H_S + H_B ,
\end{equation}
i.e., $S$ and $B$ are not coupled (remember the definition of the restricted Hamiltonian in \texteqref{eq:restrictedhamiltonian}), then $\rho^S(t)$ will simply evolve unitarily and equilibration of $\rho^S(t)$ is clearly impossible.
Hence, one needs a condition that excludes such non-interacting Hamiltonians.
Imposing the condition of non-degenerate energy gaps is a mathematically elegant, simple, and natural way to do this.
It is easy to see that Hamiltonians of the form given in \texteqref{eq:hamiltonianwithoutcouplingtodemonstratenondegenerategaps} have many degenerate gaps, as their eigenvalues are simply sums of the eigenvalues of $\H_S$ and $\H_B$.

In the more recent literature, the condition of non-degenerate energy gaps has been gradually weakened.
Ref.~\cite{1110.5759v1} defines the maximal number of energy gaps in any energy interval of width $\epsilon$
\begin{equation}\label{eq:numeberofalmostdegenerategaps}
  N(\epsilon) \coloneqq \sup_{E \in \R} |\{(k,l) \in [d']^2\suchthat k\neq l \land E_k - E_l \in [E,E+\epsilon] \}| .
\end{equation}
Note that $N(0)$ is the number of degenerate energy gaps and a Hamiltonian $\H$ satisfies the \emph{non-degenerate energy gaps} condition if and only if $N(0) = 1$.
The above definition allows to prove an equilibration theorem that still works if a system has a small number of degenerate energy gaps.
Moreover, it has the advantage that it allows to make statements about the equilibration time.
As we will see in the next theorem, equilibration on average can be guaranteed to happen on a time scale $T$ that is large enough such that $T\,\epsilon \gg 1$ where $\epsilon$ must be chosen small enough such that $N(\epsilon)$ is small compared to the number of significantly populated energy levels.

The arguably strongest and most general result concerning equilibration on average in quantum systems can be obtained by combining the two recent works Refs.~\cite{1110.5759v1,Reimann12}.
In fact, we will see that it even goes slightly beyond a mere proof of equilibration on average, as it does have non-trivial implications for the time scales on which equilibration happens.

\begin{theorem}[Equilibration on average] \label{thm:equilibrationonaverage}
  Given a system with Hilbert space $\mcH$ and Hamiltonian $\H \in \Obs(\mcH)$ with spectral decomposition $\H = \sum_{k=1}^{d'} E_k\,\Pi_k$.
  For $\rho(0) \in \Qst(\mcH)$ the initial state of the system, let $\omega = \$_H(\rho(0))$ be the dephased state and define the energy level occupations $p_k \coloneqq \Tr(\Pi_k\,\rho(0))$.
  Then, for every $\epsilon,T>0$ it holds that (i) for any operator $A \in \Bop(\mcH)$
  \begin{equation} \label{eq:equilibrationonaverageforexpectationvalues}
    \taverage[T]{( \ex A {\rho(t)} - \ex A \omega )^2} \leq \norm[\infty]{A}^2\,N(\epsilon)\,f(\epsilon\,T)\,g((p_k)_{k=1}^{d'})  ,
  \end{equation}
  and (ii) for every set $\POVMs$ of POVMs
  \begin{equation} \label{eq:equilibrationonaverageforrestricedpovms}
    \taverage[T]{\tracedistance[\POVMs]{\rho(t)}{\omega}} \leq h(\POVMs)\,\sqrt{N(\epsilon)\,f(\epsilon\,T)\,g((p_k)_{k=1}^{d'}) } ,
  \end{equation}
  where $N(\epsilon)$ is defined in \texteqref{eq:numeberofalmostdegenerategaps}, $f(\epsilon\,T) \coloneqq 1+8 \log_2(d')/(\epsilon\,T)$,
  \begin{align}
    && g((p_k)_{k=1}^{d'}) &\coloneqq \min(\sum_{k=1}^{d'} p_k^2, 3  \maxprime_k p_k ) ,\label{eq:equilibrationonaveraggeneralizedeffectivedimenson}\\
    \text{and}&& h(\POVMs) &\coloneqq \min(|{\union \POVMs}|/4, \dim(\mcH_{\supp(\POVMs)})/2 ) \label{eq:equilibrationonaveragnumberofmeasurements},
  \end{align}
  with $\maxprime_k p_k$ the second largest element in $(p_k)_{k=1}^{d'}$, $\union \POVMs$ the set of all distinct POVM elements in $\POVMs$, and $\supp(\POVMs) \coloneqq \bigcup_{M \in \union \POVMs} \supp(M)$.
\end{theorem}

\begin{proof}
  \texteqref{eq:equilibrationonaverageforexpectationvalues} for $g((p_k)_{k=1}^{d'})$ equal to the first argument of the $\min$ in \texteqref{eq:equilibrationonaveraggeneralizedeffectivedimenson} is Theorem~1 in Ref.~\cite{1110.5759v1}.
  The same statement, but with $g((p_k)_{k=1}^{d'})$ equal to the second argument in the $\min$, follows from Eqs.~(44), (50), (61), and (63) in Ref.~\cite{Reimann12}.
  With $|{\union U}|$ in \texteqref{eq:equilibrationonaveragnumberofmeasurements} replaced by the total number of all measurement outcomes, i.e., $\sum_{M \in \POVMs} |M|$, \texteqref{eq:equilibrationonaverageforrestricedpovms}, for $g((p_k)_{k=1}^{d'})$ equal to the first argument of the $\min$ in \texteqref{eq:equilibrationonaveraggeneralizedeffectivedimenson}, is implied by Theorems~2 and 3 from Ref.~\cite{1110.5759v1}.
  A careful inspection of Eq.~(B.1) in Ref.~\cite{1110.5759v1} however reveals that the slightly stronger result holds.
  In particular one can first use the bound $\max_{M(t) \in \mathcal{M}} D_{M(t)}(\rho(t),\omega) \leq \sum_{M_a \in \union \mathcal{M}} |\mathop{tr}(M_a\,\rho(t)) - \mathop{tr}(M_a\,\omega)|$ for the argument of the time average in the right hand side of the first line of Eq.~(B.1) and then use the triangle inequality to pull the time average into the sum.
  For $g((p_k)_{k=1}^{d'})$ equal to the second argument the result follows using \texteqref{eq:equilibrationonaverageforexpectationvalues} instead of Theorem~1 from Ref.~\cite{1110.5759v1} in the proofs of Theorems~2 and 3 from Ref.~\cite{1110.5759v1}.
\end{proof}


\begin{figure}[bt]
  \centering
  \begin{tikzpicture}[scale=0.7]
    \draw[-,thick] (-0.5,0) -- (7,0);
    \draw[-,dotted,thick] (7,0) -- (8,0) ;
    \draw[-,dotted,thick] (9,0) -- (10,0) ;
    \draw[->,thick] (10,0) -- (14,0) node[near end,below] {$t$} ;
    \draw[->,thick] (-0.5,0) -- (-0.5,5) node[midway,above,rotate=90] {$( \ex A {\rho(t)} - \ex A \omega )^2$} ;
    \draw[color=structure] plot file {distanceevolution.dat};
    \draw[color=structure] plot file {distanceevolution2.dat};
    \draw[-,dotted,thick] (0,0) -- (0,5) node[at start, below] {$0$};
  \end{tikzpicture}
  \caption{Equilibration on average is compatible with the time reversal invariant and recurrent nature of the time evolution of finite dimensional quantum systems. The figure shows a prototypical example of equilibration on average. Started in a non-equilibrium initial condition at time $0$ the expectation value of some observable $A$ quickly relaxes towards the equilibrium value $\ex A \omega$ and then fluctuates around it, with far excursions from equilibrium being rare. After very long times the system returns (close to) its initial state and so does the expectation value of the observable. A similar behavior is observed when the initial state is evolved backwards in time.}
  \label{fig:equilibrationonaverage}
\end{figure}

\begin{filecontents}{distanceevolution.dat}
-0.5	0.001
-0.49	0.001
-0.48	0.002
-0.47	0.002
-0.46	0.003
-0.45	0.004
-0.44	0.006
-0.43	0.007
-0.42	0.01
-0.41	0.013
-0.4	0.017
-0.39	0.022
-0.38	0.028
-0.37	0.036
-0.36	0.047
-0.35	0.059
-0.34	0.074
-0.33	0.093
-0.32	0.115
-0.31	0.142
-0.3	0.174
-0.29	0.212
-0.28	0.257
-0.27	0.308
-0.26	0.368
-0.25	0.436
-0.24	0.514
-0.23	0.602
-0.22	0.702
-0.21	0.816
-0.2	0.946
-0.19	1.095
-0.18	1.259
-0.17	1.434
-0.16	1.61
-0.15	1.779
-0.14	1.936
-0.13	2.084
-0.12	2.229
-0.11	2.377
-0.1	2.529
-0.09	2.685
-0.08	2.842
-0.07	2.996
-0.06	3.143
-0.05	3.278
-0.04	3.393
-0.03	3.479
-0.02	3.532
-0.01	3.551
  0.	3.543
0.01	3.551
0.02	3.532
0.03	3.479
0.04	3.393
0.05	3.278
0.06	3.143
0.07	2.996
0.08	2.842
0.09	2.685
0.1	2.529
0.11	2.377
0.12	2.229
0.13	2.084
0.14	1.936
0.15	1.779
0.16	1.61
0.17	1.434
0.18	1.259
0.19	1.095
0.2	0.946
0.21	0.816
0.22	0.702
0.23	0.602
0.24	0.514
0.25	0.436
0.26	0.368
0.27	0.308
0.28	0.257
0.29	0.212
0.3	0.174
0.31	0.142
0.32	0.115
0.33	0.093
0.34	0.074
0.35	0.059
0.36	0.047
0.37	0.036
0.38	0.028
0.39	0.022
0.4	0.017
0.41	0.013
0.42	0.01
0.43	0.007
0.44	0.006
0.45	0.004
0.46	0.003
0.47	0.002
0.48	0.002
0.49	0.001
0.5	0.001
0.51	0.001
0.52	0.
0.53	0.001
0.54	0.001
0.55	0.003
0.56	0.008
0.57	0.018
0.58	0.037
0.59	0.064
0.6	0.099
0.61	0.131
0.62	0.153
0.63	0.155
0.64	0.137
0.65	0.105
0.66	0.071
0.67	0.041
0.68	0.021
0.69	0.009
0.7	0.004
0.71	0.001
0.72	0.
0.73	0.
0.74	0.
0.75	0.
0.76	0.
0.77	0.
0.78	0.
0.79	0.
0.8	0.
0.81	0.
0.82	0.
0.83	0.001
0.84	0.003
0.85	0.008
0.86	0.016
0.87	0.026
0.88	0.036
0.89	0.04
0.9	0.038
0.91	0.03
0.92	0.02
0.93	0.011
0.94	0.006
0.95	0.004
0.96	0.006
0.97	0.012
0.98	0.022
0.99	0.035
1.	0.051
1.01	0.069
1.02	0.088
1.03	0.105
1.04	0.111
1.05	0.098
1.06	0.07
1.07	0.039
1.08	0.017
1.09	0.006
1.1	0.001
1.11	0.
1.12	0.
1.13	0.
1.14	0.
1.15	0.
1.16	0.
1.17	0.
1.18	0.
1.19	0.
1.2	0.
1.21	0.
1.22	0.001
1.23	0.005
1.24	0.015
1.25	0.038
1.26	0.073
1.27	0.113
1.28	0.138
1.29	0.134
1.3	0.104
1.31	0.064
1.32	0.031
1.33	0.012
1.34	0.004
1.35	0.001
1.36	0.
1.37	0.
1.38	0.
1.39	0.
1.4	0.
1.41	0.
1.42	0.
1.43	0.
1.44	0.
1.45	0.
1.46	0.
1.47	0.
1.48	0.
1.49	0.
1.5	0.
1.51	0.
1.52	0.
1.53	0.
1.54	0.
1.55	0.
1.56	0.
1.57	0.
1.58	0.
1.59	0.
1.6	0.
1.61	0.
1.62	0.
1.63	0.
1.64	0.
1.65	0.
1.66	0.
1.67	0.
1.68	0.
1.69	0.
1.7	0.
1.71	0.
1.72	0.
1.73	0.001
1.74	0.004
1.75	0.012
1.76	0.03
1.77	0.063
1.78	0.107
1.79	0.149
1.8	0.17
1.81	0.16
1.82	0.126
1.83	0.087
1.84	0.064
1.85	0.067
1.86	0.095
1.87	0.137
1.88	0.176
1.89	0.196
1.9	0.187
1.91	0.153
1.92	0.107
1.93	0.064
1.94	0.033
1.95	0.015
1.96	0.005
1.97	0.002
1.98	0.
1.99	0.
2.	0.
2.01	0.
2.02	0.001
2.03	0.004
2.04	0.012
2.05	0.028
2.06	0.053
2.07	0.085
2.08	0.11
2.09	0.118
2.1	0.105
2.11	0.081
2.12	0.061
2.13	0.057
2.14	0.072
2.15	0.098
2.16	0.121
2.17	0.129
2.18	0.116
2.19	0.089
2.2	0.058
2.21	0.034
2.22	0.021
2.23	0.019
2.24	0.026
2.25	0.039
2.26	0.055
2.27	0.071
2.28	0.085
2.29	0.1
2.3	0.113
2.31	0.122
2.32	0.121
2.33	0.111
2.34	0.104
2.35	0.11
2.36	0.129
2.37	0.15
2.38	0.155
2.39	0.138
2.4	0.104
2.41	0.067
2.42	0.037
2.43	0.021
2.44	0.017
2.45	0.028
2.46	0.051
2.47	0.086
2.48	0.128
2.49	0.164
2.5	0.184
2.51	0.178
2.52	0.149
2.53	0.109
2.54	0.069
2.55	0.038
2.56	0.018
2.57	0.007
2.58	0.003
2.59	0.001
2.6	0.
2.61	0.
2.62	0.
2.63	0.
2.64	0.
2.65	0.
2.66	0.
2.67	0.
2.68	0.
2.69	0.
2.7	0.
2.71	0.
2.72	0.
2.73	0.
2.74	0.
2.75	0.
2.76	0.
2.77	0.
2.78	0.
2.79	0.
2.8	0.
2.81	0.
2.82	0.
2.83	0.
2.84	0.
2.85	0.
2.86	0.
2.87	0.
2.88	0.001
2.89	0.002
2.9	0.004
2.91	0.008
2.92	0.012
2.93	0.016
2.94	0.025
2.95	0.049
2.96	0.107
2.97	0.211
2.98	0.355
2.99	0.501
3.	0.597
3.01	0.61
3.02	0.547
3.03	0.438
3.04	0.319
3.05	0.213
3.06	0.129
3.07	0.071
3.08	0.034
3.09	0.014
3.1	0.005
3.11	0.002
3.12	0.002
3.13	0.005
3.14	0.014
3.15	0.03
3.16	0.051
3.17	0.067
3.18	0.067
3.19	0.051
3.2	0.03
3.21	0.014
3.22	0.005
3.23	0.001
3.24	0.
3.25	0.
3.26	0.
3.27	0.
3.28	0.
3.29	0.
3.3	0.
3.31	0.
3.32	0.
3.33	0.
3.34	0.
3.35	0.
3.36	0.
3.37	0.001
3.38	0.002
3.39	0.003
3.4	0.005
3.41	0.008
3.42	0.009
3.43	0.008
3.44	0.007
3.45	0.004
3.46	0.002
3.47	0.001
3.48	0.
3.49	0.
3.5	0.
3.51	0.
3.52	0.
3.53	0.
3.54	0.
3.55	0.
3.56	0.
3.57	0.
3.58	0.
3.59	0.
3.6	0.
3.61	0.
3.62	0.002
3.63	0.009
3.64	0.029
3.65	0.066
3.66	0.104
3.67	0.116
3.68	0.091
3.69	0.051
3.7	0.02
3.71	0.005
3.72	0.001
3.73	0.
3.74	0.
3.75	0.
3.76	0.
3.77	0.
3.78	0.
3.79	0.
3.8	0.
3.81	0.
3.82	0.
3.83	0.
3.84	0.
3.85	0.
3.86	0.
3.87	0.
3.88	0.
3.89	0.
3.9	0.
3.91	0.
3.92	0.
3.93	0.
3.94	0.
3.95	0.
3.96	0.
3.97	0.
3.98	0.
3.99	0.
4.	0.
4.01	0.
4.02	0.
4.03	0.
4.04	0.
4.05	0.
4.06	0.
4.07	0.
4.08	0.
4.09	0.
4.1	0.
4.11	0.
4.12	0.
4.13	0.
4.14	0.
4.15	0.
4.16	0.
4.17	0.
4.18	0.
4.19	0.
4.2	0.001
4.21	0.002
4.22	0.006
4.23	0.015
4.24	0.029
4.25	0.047
4.26	0.066
4.27	0.078
4.28	0.076
4.29	0.063
4.3	0.044
4.31	0.026
4.32	0.013
4.33	0.005
4.34	0.002
4.35	0.001
4.36	0.
4.37	0.
4.38	0.
4.39	0.
4.4	0.
4.41	0.
4.42	0.
4.43	0.
4.44	0.
4.45	0.
4.46	0.
4.47	0.
4.48	0.
4.49	0.
4.5	0.
4.51	0.
4.52	0.
4.53	0.
4.54	0.
4.55	0.
4.56	0.
4.57	0.
4.58	0.
4.59	0.
4.6	0.
4.61	0.
4.62	0.
4.63	0.
4.64	0.
4.65	0.
4.66	0.
4.67	0.
4.68	0.
4.69	0.
4.7	0.
4.71	0.
4.72	0.
4.73	0.001
4.74	0.002
4.75	0.006
4.76	0.014
4.77	0.028
4.78	0.049
4.79	0.071
4.8	0.088
4.81	0.094
4.82	0.086
4.83	0.067
4.84	0.044
4.85	0.025
4.86	0.012
4.87	0.005
4.88	0.003
4.89	0.005
4.9	0.011
4.91	0.025
4.92	0.05
4.93	0.085
4.94	0.125
4.95	0.159
4.96	0.176
4.97	0.168
4.98	0.139
4.99	0.099
5.	0.061
5.01	0.033
5.02	0.015
5.03	0.007
5.04	0.004
5.05	0.004
5.06	0.007
5.07	0.012
5.08	0.016
5.09	0.017
5.1	0.016
5.11	0.013
5.12	0.008
5.13	0.005
5.14	0.002
5.15	0.002
5.16	0.003
5.17	0.006
5.18	0.011
5.19	0.018
5.2	0.025
5.21	0.032
5.22	0.034
5.23	0.031
5.24	0.025
5.25	0.017
5.26	0.01
5.27	0.006
5.28	0.004
5.29	0.006
5.3	0.013
5.31	0.027
5.32	0.051
5.33	0.082
5.34	0.116
5.35	0.142
5.36	0.151
5.37	0.141
5.38	0.115
5.39	0.083
5.4	0.053
5.41	0.03
5.42	0.017
5.43	0.014
5.44	0.03
5.45	0.083
5.46	0.202
5.47	0.407
5.48	0.67
5.49	0.905
5.5	1.
5.51	0.905
5.52	0.67
5.53	0.407
5.54	0.202
5.55	0.082
5.56	0.027
5.57	0.007
5.58	0.002
5.59	0.
5.6	0.
5.61	0.
5.62	0.
5.63	0.
5.64	0.
5.65	0.
5.66	0.
5.67	0.001
5.68	0.004
5.69	0.011
5.7	0.02
5.71	0.028
5.72	0.029
5.73	0.03
5.74	0.036
5.75	0.052
5.76	0.071
5.77	0.088
5.78	0.099
5.79	0.107
5.8	0.12
5.81	0.136
5.82	0.143
5.83	0.131
5.84	0.099
5.85	0.06
5.86	0.029
5.87	0.011
5.88	0.004
5.89	0.001
5.9	0.
5.91	0.
5.92	0.
5.93	0.
5.94	0.
5.95	0.
5.96	0.
5.97	0.
5.98	0.
5.99	0.
6.	0.
6.01	0.
6.02	0.
6.03	0.
6.04	0.
6.05	0.
6.06	0.
6.07	0.
6.08	0.
6.09	0.
6.1	0.
6.11	0.
6.12	0.
6.13	0.
6.14	0.
6.15	0.
6.16	0.
6.17	0.
6.18	0.
6.19	0.
6.2	0.
6.21	0.002
6.22	0.007
6.23	0.021
6.24	0.04
6.25	0.055
6.26	0.052
6.27	0.034
6.28	0.016
6.29	0.005
6.3	0.001
6.31	0.
6.32	0.
6.33	0.
6.34	0.
6.35	0.
6.36	0.
6.37	0.
6.38	0.
6.39	0.
6.4	0.
6.41	0.
6.42	0.
6.43	0.
6.44	0.
6.45	0.
6.46	0.
6.47	0.
6.48	0.001
6.49	0.003
6.5	0.013
6.51	0.038
6.52	0.092
6.53	0.177
6.54	0.265
6.55	0.308
6.56	0.276
6.57	0.19
6.58	0.1
6.59	0.042
6.6	0.019
6.61	0.021
6.62	0.035
6.63	0.049
6.64	0.057
6.65	0.066
6.66	0.08
6.67	0.099
6.68	0.121
6.69	0.142
6.7	0.152
6.71	0.139
6.72	0.102
6.73	0.059
6.74	0.026
6.75	0.009
6.76	0.002
6.77	0.
6.78	0.
6.79	0.
6.8	0.
6.81	0.
6.82	0.
6.83	0.
6.84	0.
6.85	0.
6.86	0.
6.87	0.001
6.88	0.002
6.89	0.006
6.9	0.014
6.91	0.03
6.92	0.056
6.93	0.089
6.94	0.121
6.95	0.144
6.96	0.152
6.97	0.144
6.98	0.119
6.99	0.081
7.	0.045
\end{filecontents}


\begin{filecontents}{distanceevolution2.dat}
14.	0.
13.99	0.
13.98	0.
13.97	0.
13.96	0.
13.95	0.
13.94	0.
13.93	0.
13.92	0.
13.91	0.
13.9	0.
13.89	0.
13.88	0.
13.87	0.
13.86	0.
13.85	0.
13.84	0.
13.83	0.
13.82	0.
13.81	0.
13.8	0.
13.79	0.
13.78	0.
13.77	0.
13.76	0.
13.75	0.
13.74	0.
13.73	0.
13.72	0.
13.71	0.
13.7	0.
13.69	0.
13.68	0.
13.67	0.
13.66	0.
13.65	0.
13.64	0.
13.63	0.
13.62	0.
13.61	0.
13.6	0.
13.59	0.
13.58	0.
13.57	0.
13.56	0.
13.55	0.
13.54	0.
13.53	0.
13.52	0.
13.51	0.001
13.5	0.001
13.49	0.001
13.48	0.002
13.47	0.002
13.46	0.003
13.45	0.004
13.44	0.006
13.43	0.007
13.42	0.01
13.41	0.013
13.4	0.017
13.39	0.022
13.38	0.028
13.37	0.036
13.36	0.047
13.35	0.059
13.34	0.074
13.33	0.093
13.32	0.115
13.31	0.142
13.3	0.174
13.29	0.212
13.28	0.257
13.27	0.308
13.26	0.368
13.25	0.436
13.24	0.513
13.23	0.6
13.22	0.697
13.21	0.805
13.2	0.923
13.19	1.051
13.18	1.189
13.17	1.336
13.16	1.491
13.15	1.653
13.14	1.821
13.13	1.993
13.12	2.166
13.11	2.338
13.1	2.508
13.09	2.672
13.08	2.828
13.07	2.973
13.06	3.104
13.05	3.22
13.04	3.318
13.03	3.397
13.02	3.454
13.01	3.488
13.	3.5
12.99	3.488
12.98	3.454
12.97	3.397
12.96	3.318
12.95	3.22
12.94	3.104
12.93	2.973
12.92	2.828
12.91	2.672
12.9	2.508
12.89	2.338
12.88	2.166
12.87	1.993
12.86	1.821
12.85	1.654
12.84	1.492
12.83	1.338
12.82	1.194
12.81	1.061
12.8	0.941
12.79	0.834
12.78	0.736
12.77	0.646
12.76	0.561
12.75	0.485
12.74	0.432
12.73	0.426
12.72	0.471
12.71	0.522
12.7	0.506
12.69	0.405
12.68	0.28
12.67	0.195
12.66	0.155
12.65	0.134
12.64	0.109
12.63	0.08
12.62	0.052
12.61	0.032
12.6	0.021
12.59	0.014
12.58	0.011
12.57	0.009
12.56	0.011
12.55	0.018
12.54	0.034
12.53	0.058
12.52	0.09
12.51	0.12
12.5	0.139
12.49	0.14
12.48	0.124
12.47	0.102
12.46	0.083
12.45	0.072
12.44	0.066
12.43	0.061
12.42	0.052
12.41	0.039
12.4	0.026
12.39	0.015
12.38	0.008
12.37	0.007
12.36	0.01
12.35	0.021
12.34	0.046
12.33	0.092
12.32	0.159
12.31	0.221
12.3	0.25
12.29	0.245
12.28	0.224
12.27	0.194
12.26	0.154
12.25	0.11
12.24	0.074
12.23	0.05
12.22	0.036
12.21	0.026
12.2	0.017
12.19	0.01
12.18	0.005
12.17	0.002
12.16	0.001
12.15	0.
12.14	0.
12.13	0.001
12.12	0.005
12.11	0.016
12.1	0.039
12.09	0.063
12.08	0.069
12.07	0.051
12.06	0.026
12.05	0.009
12.04	0.002
12.03	0.
12.02	0.
12.01	0.
12.	0.
11.99	0.
11.98	0.
11.97	0.
11.96	0.
11.95	0.
11.94	0.
11.93	0.002
11.92	0.009
11.91	0.028
11.9	0.067
11.89	0.124
11.88	0.173
11.87	0.186
11.86	0.158
11.85	0.113
11.84	0.076
11.83	0.048
11.82	0.027
11.81	0.011
11.8	0.003
11.79	0.001
11.78	0.
11.77	0.
11.76	0.
11.75	0.
11.74	0.
11.73	0.
11.72	0.
11.71	0.
11.7	0.
11.69	0.
11.68	0.
11.67	0.
11.66	0.
11.65	0.
11.64	0.
11.63	0.
11.62	0.
11.61	0.
11.6	0.
11.59	0.
11.58	0.
11.57	0.
11.56	0.
11.55	0.
11.54	0.
11.53	0.
11.52	0.
11.51	0.
11.5	0.001
11.49	0.003
11.48	0.009
11.47	0.026
11.46	0.059
11.45	0.107
11.44	0.156
11.43	0.183
11.42	0.173
11.41	0.132
11.4	0.081
11.39	0.04
11.38	0.016
11.37	0.005
11.36	0.001
11.35	0.
11.34	0.
11.33	0.
11.32	0.
11.31	0.
11.3	0.
11.29	0.
11.28	0.
11.27	0.
11.26	0.
11.25	0.
11.24	0.
11.23	0.
11.22	0.
11.21	0.
11.2	0.
11.19	0.
11.18	0.
11.17	0.
11.16	0.
11.15	0.
11.14	0.
11.13	0.
11.12	0.
11.11	0.
11.1	0.
11.09	0.
11.08	0.
11.07	0.
11.06	0.
11.05	0.
11.04	0.
11.03	0.
11.02	0.
11.01	0.
11.	0.001
\end{filecontents}


What is the physical meaning of the theorem?
The quantity $g((p_k)_{k=1}^{d'})$ is small, except if the initial state assigns large populations to few (but more than one) energy levels.
For initial states with a reasonable energy uncertainty and large enough systems it can be expected to be of the order of $\landauO(1/d')$, i.e., reciprocal to the total number of distinct energy levels.
The quantity $h(\POVMs)$ on the other hand can be thought of as a measure of the experimental capabilities in distinguishing quantum states and can reasonably be assumed to be much smaller than $d'$.
In particular, when all measurements in $\POVMs$ have a support contained inside of a small subsystem $S \subset \Vset$ it is bounded by $d_S/2$.
Because of the conditions for equality in \texteqref{eq:tracedistanceboundsrestrictedtracedistance}, the theorem then also implies an upper bound on $\taverage[T]{\tracedistance{\rho^S(t)}{\omega^S}}$ and hence proves \emph{subsystem equilibration on average}.

For fixed $\H$ and $\epsilon>0$ we have $\lim_{T\to\infty} f(\epsilon\,T) = 1$, hence the theorem proves, for a wide class of reasonable initial states, equilibration on average of all sufficiently small subsystems and \emph{apparent equilibration on average} of the state of the full system under realistic restrictions on the number of different measurements that can be performed.
In this sense it improves and generalizes the results of Refs.~\cite{Reimann08,Linden09}.

On what time scales is equilibrium reached?
The product $N(\epsilon)\,f(\epsilon\,T)$, which is lower bounded by one, will typically be close to one only if $T$ is comparable to ${d'}^2$, i.e., to the total number of energy gaps, and will otherwise be roughly of the order of $\landauOmega({d'}^2/T)$ for smaller $T$.
So, even under the favorable assumption that $g((p_k)_{k=1}^{d'})$ is of the order of $\landauO(1/d')$, equilibration of a subsystem $S$ can only be guaranteed after a time $T$ that is roughly of the order of $\landauOmega(d_S^2\,d')$.


\subsubsection*{Discussion}
%
A brief comment concerning equilibration time scales is in order, even though we will discuss this in more detail in Section~\ref{sec:timescales}.
As we have seen, Theorem~\ref{thm:equilibrationonaverage} can guarantee equilibration of subsystems only over time scales on the order of $\landauOmega(d_S^2\,d')$.
Both $d'$ and $d_S$ typically grow exponentially with the size of the composite system and the subsystem $S$ respectively.
Hence, times of the order of $\landauOmega(d_S^2\,d')$ are unphysical already for systems of moderate size.
This weakness of theorems such as Theorem~\ref{thm:equilibrationonaverage} has been rightfully criticized in Ref.~\cite{1109.4696v1}.

There are at least two possible replies to this criticism:
First, it is known that there are systems in which equilibration does indeed take extremely long (see Section~\ref{sec:timescales}) and thus, being a very general statement, Theorem~\ref{thm:equilibrationonaverage} is probably close to optimal.
Proofs of shorter equilibration times will need further assumptions, such as locality or translation invariance of the Hamiltonian, and restrictions on the allowed measurements \cite{1110.5759v1,Linden09}.
Second, almost all systems in which equilibration has been studied and in which equilibration of some property on reasonable time scales could be demonstrated were found to exhibit equilibration towards the time average (see for example \cite{Gemmer09,Campos10,Fagotti2012,Rigol11,1110.4690v1,Rigol07,Rigol2006}), so in these cases the upper bound on the equilibration time implied by Theorem~\ref{thm:equilibrationonaverage} is not tight, but the theorem still captures the relevant physics.
Transient equilibration to metastable states that precedes equilibration to the time average seems to require special structure in the Hamiltonian.
That the physics of such special systems is not captured by a result as general as Theorem~\ref{thm:equilibrationonaverage} is not too surprising.

An interesting variant of the subsystem equilibration setting is investigated in Ref.~\cite{MasterThesisHutter}, in which the subsystem $S$ can initially be correlated (either classically or even quantum mechanically) with a reference system $R$.
The ``knowledge'' about the initial state of $S$ stored in the reference $R$ can in principle help to distinguish the state $\rho^S(t)$ from $\omega^S$.
Still, by using \emph{decoupling theorems} \cite{Dupuis2010,1109.4348v1,Szehr2012} and properties of \emph{smooth min and max entropies} \cite{Koenig08,Ciganovic2013} it is possible to show subsystem equilibration on average under conditions similar to those of Theorem~\ref{thm:equilibrationonaverage}, in the sense that the combined state of $S$ and $R$ is on average almost indistinguishable from $\omega^{SR} = \taverage{\rho^{SR}}$.

In the above disquisition on equilibration we have put a focus on the more recent literature.
From a historical perspective this is highly unjust.
Many of the ideas behind the results mentioned above can already be found in the work of von~\textcite{vonneumann1929}.
I encourage the interested reader to consider the English translation \cite{Tumulka2010} of this article and the discussion of von~Neumann's results in Ref.~\cite{0907.0108v1} and the brief summary of parts of this article in Section~\ref{sec:typicality} of this work.

Further statements concerning equilibration towards the dephased state, which are related to those discussed above, can also be found in Refs.~\cite{1108.2985v3,1108.0374,1112.5295v1,1107.6035v1}.
We will discuss these works in more detail in Section~\ref{sec:timescales}.


\subsection{Equilibration during intervals}
\label{sec:equlibrationinthestrongsense}
%
In this section we will investigate under which conditions equilibration during intervals can be guaranteed.
After a brief overview of the literature on the topic we will concentrate on the results presented in Ref.~\cite{cramer10_1}.
Instead of reproducing the full proof we will only give the intuition behind it and describe the general structure.
One reason for this is that Ref.~\cite{cramer10_1} is concerned with a special class of bosonic Hamiltonians, so-called \emph{quadratic bosonic Hamiltonians}, i.e., Hamiltonians that are quadratic polynomials in the bosonic creation and annihilation operators.
For these Hamiltonians there exists a special formalism based on so-called \emph{covariance matrices} that allows, for example, to calculate for a special class of initial states, namely \emph{Gaussian states}, the time evolution of the expectation values of certain observables in a computationally efficient way.
A full introduction of this formalism is beyond the scope of this thesis.
More details can be found for example in Refs.~\cite{Anders2003,Braunstein2005,Adesso2007}.

Equilibration during intervals of non-Gaussian initial states under certain quadratic Hamiltonians was proved in Ref.~\cite{PhysRevLett.10-5} and the results have later been generalized and improved in Ref.~\cite{cramer10_1}.
The techniques are inspired by earlier works \cite{Dudnikova2003} on classical harmonic crystals, i.e., systems of coupled classical harmonic oscillators, and can be seen as bounds on the pre-asymptotic behavior and an extension to finite system sizes of the results on equilibration of Ref.~\cite{Lanford1972}.
See also Refs.~\cite{Lanford1972,Tegmark1994,Barthel08} for related results on equilibration starting from Gaussian initial states.

More precisely, the results on equilibration during intervals concern systems evolving under certain quadratic Hamiltonians of the form
\begin{equation} \label{eq:quadratichamiltonian}
  \H = \frac{1}{2}\,\sum_{x,y\in\Vset} \left( b_x^\dagger\,A_{x,y}\,b_y + b_x\,A_{x,y}\,b_y^\dagger \right) ,
\end{equation}
where $b_x,b_x^\dagger$ are the bosonic annihilation/creation operators on site $x \in \Vset$ and $A \in \R^{|\Vset|\times|\Vset|}$.
The operator $\H$, as defined in \texteqref{eq:quadratichamiltonian}, is unbounded and hence, in principle, a careful treatment of the system with the methods of functional analysis \cite{Reed1980,Reed1975} would be necessary.
The Hamiltonian in \texteqref{eq:quadratichamiltonian} is however particle number preserving.
Thus, when we restrict to initial states with finite particle number the whole evolution happens in a finite dimensional subspace of the Fock space.
The Hamiltonian $\H$ and all relevant observables can then be represented by bounded operators on this subspace.
We are hence back in the framework of finite dimensional quantum mechanics as introduced in Section~\ref{sec:prelimiaries} (see in particular Section~\ref{sec:localquantumsystems}) and the following statement is well-defined:

\begin{theorem}[Equilibration during intervals] \label{thm:equilbrationduringintervals}
  Consider the class of systems with a finite number of bosons in $M$ modes on a ring with nearest neighbor interactions, i.e., $\Vset = [M]$ and $\Eset=\{(1,2),\allowbreak(2,3),\dots,(M,1)\}$, evolving under a Hamiltonian of the form given in \eqref{eq:quadratichamiltonian} with $A = -\delta_{|i-j| \mod M,1}$.
  Let $\mcH$ be the direct sum of Fock layers up to the maximal particle number.
  If the initial state $\rho(0) \in \Qst(\mcH)$ satisfies a form of decay of correlations (Assumptions~1--3 in Ref.~\cite{cramer10_1}), then for every $S \subset \Vset$ and every $\epsilon>0$ there exists a system size $M^\ast$, such that for all $M \geq M^\ast$ there exists a time $t_{\mathrm{relax}}$ independent of $M$ and a time $t_{\mathrm{rec}} \in \landauOmega(M^{6/7})$ such that there is a state $\tilde\omega^S \in\Qst(\mcH_S)$ such that
  \begin{equation}
    \forall t \in [t_{\mathrm{relax}},t_{\mathrm{rec}}]\itholds \tracedistance{\rho^S(t)}{\tilde\omega^S} \leq \epsilon .
  \end{equation}
\end{theorem}
%Note that $\tilde \omega^S$ is not necessarily the dephased state reduced to $S$.
\begin{proof}
  The theorem is essentially implied by Theorem~2 and Corollary~1 from Ref.~\cite{cramer10_1}, the scaling of the times $t_{\mathrm{relax}}$ and $t_{\mathrm{rec}}$ follows from Eq.~(61) and Lemma~4 in Ref.~\cite{cramer10_1}.
\end{proof}
The theorem proves equilibration during the interval $[t_{\mathrm{relax}},t_{\mathrm{rec}}]$ of all small subsystems of a sufficiently large system.
Important for the proof is the decay of correlations in the initial state and that the specific quadratic Hamiltonian under consideration leads to transport through the system.
The time $t_{\mathrm{relax}}$ depends on the size of the subsystem $S$ under consideration, but is independent of the size of the composite system.
It depends on the speed at which the Hamiltonian is able to transport correlations through the system and the length scale on which the correlations in the initial state decay.
The time $t_{\mathrm{rec}}$ is a lower bound on the recurrence time and is slightly smaller than the time it takes for a signal to travel around the ring of bosonic modes.

\subsubsection*{Discussion}
%
According to Ref.~\cite{cramer10_1} a similar theorem can also be proved for systems on other lattices evolving under more general quadratic Hamiltonians of the form given in \eqref{eq:quadratichamiltonian}, but even such more general results would suffer from the following limitation:
Quadratic Hamiltonians of the form given in \texteqref{eq:quadratichamiltonian}, whose $A$ matrix is banded, i.e., has non-zero entries only for pairs of sites that are geometrically close to each other, have \emph{local conserved quantities}, i.e., observables that commute with $\H$ but have support only on a small number of sites independent of the system size.
The time averaged state has the same expectation values for all these local observables as the initial state so that equilibration of subsystems to a state that is close in trace distance to a thermal/Gibbs state of the truncated Hamiltonian is impossible for most initial states \cite{Lanford1972,cramer10_1}.


\subsection{A conjecture concerning equilibration}
\label{sec:conjectureconcerningequilibration}
%
In this section we formulate the \emph{equilibration from transport conjecture}.
It claims the existence of a general equilibration mechanism based on the spreading of the support of local observables during the time evolution under a certain class of Hamiltonians.
The line of reasoning that we will use to justify this conjecture is based on the proofs of the results concerning equilibration during intervals that we discussed in the last section.

The proof of Theorem~\ref{thm:equilbrationduringintervals} exploits the special structure of quadratic Hamiltonians and the covariance matrix formalism.
However, taking a birds eye perspective reveals the general mechanism behind the equilibration process described in Ref.~\cite{cramer10_1}.

The four main ingredients are:
\begin{enumerate}[leftmargin=*]
\item \label{item:naturalnotionofdistance}\emph{Natural notion of distance}: The Hamiltonian of the system must contain only short range interactions, in the sense that a reasonable locality structure emerges that allows for the definition of a (pseudo)metric with the property that far apart regions do not interact directly or do so with sufficiently quickly decreasing strength.
\item \label{item:decayofcorrelations}\emph{Decay of correlations}: The initial state must satisfy a form of decay of correlations in this distance. That is, subsystems that are sufficiently far apart must be sufficiently uncorrelated initially.
\item \label{item:lrbound}\emph{Maximal speed of propagation}: The dynamics must obey some upper bound on the speed at which correlations can spread through the system, a so-called \emph{Lieb-Robinson} \cite{Lieb1972,Kliesch2013} bound.
\item \label{item:transport}\emph{Transport}: The system must exhibit a form of transport that makes initially uncorrelated parts of the system interact with each other sufficiently quickly.
\end{enumerate}
Properties~\ref{item:naturalnotionofdistance} and \ref{item:decayofcorrelations} make sense independently of the specific model studied in Ref.~\cite{cramer10_1} and both short range interactions and the assumption of some form of decay of correlations in the initial state seem to be physically appealing and plausible starting points for a proof of equilibration.
Likewise, a wide range of physically relevant composite systems share Property~\ref{item:lrbound}.
See for example Ref.~\cite{Kliesch2013} and the references therein for a review of the topic of \emph{Lieb-Robinson bounds}.

The most vague of the properties in the above list is Property~\ref{item:transport}.
This is also where the proof of Ref.~\cite{cramer10_1} makes most explicitly use of the particular structure of the system that is considered.
Hamiltonians of the form given in Theorem~\ref{thm:equilbrationduringintervals} exhibit \emph{ballistic transport}.
The elementary excitations of Hamiltonians of this form behave very similarly to freely moving particles, which is why systems with quadratic Hamiltonians are also sometimes referred to as \emph{free systems} \cite{Lanford1972}, and can travel arbitrary distances with a constant speed.

In order to prove statements similar to Theorem~\ref{thm:equilbrationduringintervals} for more general systems, a better understanding of the transport that is responsible for the mechanism behind Theorem~\ref{thm:equilbrationduringintervals} seems necessary.
We will now define a notion of transport that appears to be sufficient for equilibration during intervals.
It turns out to be more convenient to formulate it in the \emph{Heisenberg picture}, i.e., to consider observables as time dependent and conversely quantum states as static (see also Section~\ref{sec:timeevolution}).
\begin{definition}[Spreading transport] \label{def:mixingtransport}
  An infinite sequence of composite quantum systems indexed by their system size $|\Vset|$ with respective Hilbert spaces $\mcH(|\Vset|)$, evolving under Hamiltonians $\H(|\Vset|) \in \Obs(\mcH(|\Vset|))$ exhibits \emph{spreading transport} if there exists a $D \in \landauO(1)$ and a time $t_2 \in \landauOmega(|\Vset|^{1/D})$ such that up to $t_2$ the support of every initially observable $A\oftype \R \to  \Obs(\mcH(|\Vset|))$ which was initially local, i.e., $|\supp(A(0))| \in \landauO(1)$ grows with time in the sense that for every sufficiently small $0<\epsilon \in \landauO(1)$ and every $\tilde{A}\oftype \R \to \Obs(\mcH(|\Vset|))$ and all $0 \leq t_1 \leq t_2$ it holds that
  \begin{equation}
    \left( \forall t \in [t_1,t_2]\itholds \norm[\infty]{\tilde{A}(t) - A(t)} \leq \epsilon \right) \implies \inf_{t \in [t_1,t_2]} |\supp(\tilde{A}(t))| \in \landauOmega(t_1) .
  \end{equation}
\end{definition}

The intuition behind this definition is as follows:
If $A(0)$ is a local observable with support only on a small subsystem, then, if the system exhibits spreading transport, for all $t$ up to a time $t_2$, which grows with the system size, $A(t) = U(t)\ad\,A\,U(t)$ will act non-trivially on a subsystem of size at least proportional to $t$, and it will do so in a way that it cannot be approximated by any other observable $\tilde{A}(t)$ with a significantly smaller support.
Hence it will depend significantly on a large part of the initial state.
If the initial state exhibits some form of clustering of correlations with a correlation length that is small compared to the size of the region on which $A(t)$ acts sufficiently non-trivially, the expectation value $\ex{A(t)}{\rho}$ will be a sum of many small and uncorrelated contributions, and will hence look \emph{equilibrated}.
Thus it seems that spreading transport together with conditions \ref{item:naturalnotionofdistance}--\ref{item:lrbound} should be sufficient to guarantee that an ``equilibrium'' value is reached after a time independent of the system size.

The reason why we need to speak about a sequence of systems is that we must have a way to guarantee that the expectation value of $A(t)$ stays equilibrated for a long time.
To do this, we must be able to exclude that its support shrinks again due to a (quasi) recurrence in the system, which is why we introduce the time $t_2$.
The quantity $D$ is meant to play the role of a spacial dimension, such that $|\Vset|^{1/D}$ can be thought of as a measure of the linear size of the system.
It seems plausible that for a suitably chosen but fairly general class of Hamiltonians, Lieb-Robinson type bounds can guarantee the existence of such a time in a way similar to how this is achieved with a more specialized bound in Ref.~\cite{cramer10_1}.

The condition on the growth rate of the support of local observables given in Definition~\ref{def:mixingtransport} is actually weaker than what one would reasonably expect to find in natural locally interacting systems, but seems sufficient for equilibration on time scales independent of the system size.
If $D$ is the spacial dimension of the lattice of the system, one can expect that $\inf_{t \in [t_1,t_2]} |\supp(\tilde{A}(t))|$ scales more like $\landauOmega(t_1^D)$, the volume inside the ``light cone'' of the dynamics, or at least like $\landauOmega(t_1^{D-1})$, its surface area.

Based on the above considerations I make the following conjecture:
\begin{conjecture}[Equilibration from transport conjecture] \label{conj:equilibrationfromtransportconjecture}
  Any sufficiently large locally interacting system with sufficiently quickly decaying correlations in the initial state and a maximal speed of propagation of correlations that exhibits \emph{spreading transport} has the property that all sufficiently small subsystems equilibrate during a time interval with a system size independent lower bound and an upper bound that grows with the system size.
\end{conjecture}


\subsubsection*{Discussion}
%
Spreading transport excludes the existence of local conserved quantities and hence, in particular, requires that the system is not localized due to, for example, spacial disorder.
A proof of the above conjecture would therefore fit nicely into the folklore knowledge that disorder prevents systems from reaching thermodynamic equilibrium, and the often invoked picture of quasi particles propagating through a system, thereby leading to equilibration \cite{Calabrese2007}.

Whether the above conjecture can be turned into a theorem and whether Definition~\ref{def:mixingtransport} is the right notion of transport is the subject of further research.


\subsection{Other notions of equilibration}
\label{sec:othernotionsofequilibration}
%
In this section we briefly cover two other notions of equilibration for closed quantum systems.

The first alternative notion of equilibration we want to discuss was proposed in Ref.~\cite{0907.0108v1}.
This work is closely related to an article of von~\textcite{vonneumann1929}, which investigates why large quantum systems can be well described within the framework of statistical mechanics.
To that end it is postulated that on large systems only a set of so-called \emph{macroscopic observables} is accessible.
The macroscopic observables are required to commute, thus they divide the Hilbert space in subspaces, so-called \emph{phase cells}, each containing states that belong to the same sequence of eigenvalues for all the macroscopic observables (see also the more detailed discussion of Ref.~\cite{vonneumann1929} in Section~\ref{sec:typicality}).
If one of the phase cells is particularly large, Ref.~\cite{0907.0108v1} associates it with \emph{thermal equilibrium} and says that a system is in \emph{thermal equilibrium} if and only if its state is almost entirely contained in that cell.
Variants of the results from Ref.~\cite{vonneumann1929} can then be used to prove equilibration in this sense.

Reminding oneself that measurements of quantum systems are ultimately \emph{sampling experiments} opens up an entirely new vista on the problem of equilibration, which leads us the second alternative notion of equilibration.
Performing a measurement of an observable does neither provide the experimentalist with the measurement statistic nor does it yield the expectation value of the observable.
Both can only be approximately determined by repeatedly performing the same experiment many times.
How many repetitions are needed to distinguish whether the measurement statistic of a given observable is close or far from that predicted by equilibrium statistical mechanics?
Such questions have been posed and partially answered in the fields of \emph{sample complexity} \cite{Batu2001,Batu2000,Canonne2012} and \emph{state discrimination} \cite{Audenaert2012,Audenaert2012a}.
Using the complexity of the task of collecting information about a quantum system as a justification for a statistical description was recently proposed in Ref.~\cite{Ududec2012}, which defines the concept of \emph{information theoretic equilibration}.
Essentially the authors of Ref.~\cite{Ududec2012} are able to show that with the use of very fine grained observables pure quantum states are practically indistinguishable from states corresponding to statistical ensembles.


\section{A quantum maximum entropy principle}
\label{sec:aquantummaximumentropyprinciple}
%
We have seen in Section~\ref{sec:equlibrationintheweaksense} that if the expectation value of an observable or the reduced state of a subsystem equilibrates on average, then they necessarily equilibrate to their expectation value in, or the reduced state of, the time averaged/dephased state $\omega = \taverage{\rho} = \$_\H(\rho(0))$.
The state $\omega$ hence encodes the information necessary to describe the equilibrium properties of such a system.
Moreover, it has a peculiar property that implies the following \emph{quantum mechanical maximum entropy principle}:

\begin{theorem}[Maximum entropy principle \cite{PhysRevLett.10-6}] \label{thm:maximumentropyprinciple}
  Consider the time evolution $\rho\oftype\R\to\Qst(\mcH)$ of a quantum system with Hilbert space $\mcH$ and Hamiltonian $\H \in \Obs(\mcH)$.
  If the expectation value of an operator $A \in \Bop(\mcH)$ equilibrates on average, then it equilibrates towards its time average, given by
  \begin{equation}
    \taverage{\Tr(A\,\rho)} = \Tr(A\,\omega) ,
  \end{equation}
  where $\omega = \taverage{\rho}$ is the unique quantum state that maximizes the von~Neumann entropy $\Svn$, given all conserved quantities.
\end{theorem}
\begin{proof}
  That the equilibrium value of the expectation value of $A$ is given by $\Tr(A\,\omega)$ follows directly from the definition of equilibration on average.
  The time averaged state $\omega$ is equal to the dephased initial state
  \begin{equation}
    \$_H(\rho(0)) = \sum_{k=1}^{d'} \Pi_k\,\rho(0)\,\Pi_k .
  \end{equation}
  The dephasing map $\$_\H$ is a so-called \emph{pinching} and the von~Neumann entropy is non-decreasing under pinchings \cite[Problem II.5.5]{bhatia} (this is a generalization of \emph{Schur's theorem}).
  Furthermore, two states $\sigma_1,\sigma_2 \in \Qst(\mcH)$ yield the same expectation values for all conserved quantities, i.e., all $A \in\Obs(\mcH)$ that commute with the Hamiltonian $[A,\H] = 0$, if and only if $\$_\H(\sigma_1) = \$_\H(\sigma_2)$.
  This already shows that $\omega$ has the maximal achievable von~Neumann entropy given all conserved quantities.
  It remains to show uniqueness.
  Let $\Basis$ be a basis of the linear span of all $A \in\Obs(\mcH)$ with $[A,\H] = 0$.
  The objective function of the maximization problem, namely the von~Neumann entropy, is a strictly concave function $\Svn\oftype\Qst(\mcH) \to \R$ and it is optimized over all $\sigma \in \Qst(\mcH)$ under the finite number of affine equality constrains $\forall B \in \Basis \itholds \Tr(B\,\sigma) = \Tr(B\,\rho(0))$.
  Under these conditions uniqueness follows from a standard result from convex optimization \cite{Boyd2004}.
\end{proof}

\begin{figure}[h!]
  \centering
  \begin{subfigure}{0.3\textwidth}
    \begin{tikzpicture}[scale=0.45,transform shape]
      \path[use as bounding box] (-1,-0.3) rectangle (8.7,-8.7);
      \foreach \i/\ri/\phii/\omegai in {1/0.137264/-159.008/1,2/0.0919508/93.4659/1,3/0.204039/-110.476/1,4/0.14146/28.212/2,5/0.116946/145.35/4,6/0.128177/59.1885/4,7/0.113461/-153.794/5,8/0.0667032/-167.439/6}
      \foreach \j/\rj/\phij/\omegaj in {1/0.137264/-159.008/1,2/0.0919508/93.4659/1,3/0.204039/-110.476/1,4/0.14146/28.212/2,5/0.116946/145.35/4,6/0.128177/59.1885/4,7/0.113461/-153.794/5,8/0.0667032/-167.439/6}
      {
        \draw[very thin,-latex] (\i,-\j) -- +(\phii-\phij:\ri*\rj*12);
        % \draw<4>[thick,-latex] (\i,-\j) -- +({\phii-\phij+(\omegai-\omegaj)*60}:\ri*\rj*12);
        \ifnum \omegai=\omegaj
        % \draw<5->[thick,-latex] (\i,-\j) -- +(\phii-\phij:\ri*\rj*12);
        \fi
        % \draw<4>[thick] (\phii-\phij:\ri*\rj*12)+(\i,-\j) arc (\phii-\phij:{\phii-\phij+(\omegai-\omegaj)*60}:\ri*\rj*12)+(\i,-\j);
        \node[minimum size=1cm] at (\i,-\j) (n\i\j) {};
      }

      \draw[thick] (n11.north west) to[bend right=6] node[midway,anchor=base east] {\Large $\rho(0)=$} (n18.south west);
      \draw[thick] (n88.south east) to [bend right=6] (n81.north east);
      \fill[opacity=0.2,niceblue] (n11.north west) rectangle (n33.south east);
      \fill[opacity=0.2,niceblue] (n44.north west) rectangle (n44.south east);
      \fill[opacity=0.2,niceblue] (n55.north west) rectangle (n66.south east);
      \fill[opacity=0.2,niceblue] (n77.north west) rectangle (n77.south east);
      \fill[opacity=0.2,niceblue] (n88.north west) rectangle (n88.south east);
    \end{tikzpicture}
    \caption{}
    \label{fig:maximumentropyprinciplea}
  \end{subfigure}\hfill
  \begin{subfigure}{0.3\textwidth}
    \begin{tikzpicture}[scale=0.45,transform shape]
      \path[use as bounding box] (-1,-0.3) rectangle (8.7,-8.7);
      \foreach \i/\ri/\phii/\omegai in {1/0.137264/-159.008/1,2/0.0919508/93.4659/1,3/0.204039/-110.476/1,4/0.14146/28.212/2,5/0.116946/145.35/4,6/0.128177/59.1885/4,7/0.113461/-153.794/5,8/0.0667032/-167.439/6}
      \foreach \j/\rj/\phij/\omegaj in {1/0.137264/-159.008/1,2/0.0919508/93.4659/1,3/0.204039/-110.476/1,4/0.14146/28.212/2,5/0.116946/145.35/4,6/0.128177/59.1885/4,7/0.113461/-153.794/5,8/0.0667032/-167.439/6}
      {
        % \draw[very thin,-latex] (\i,-\j) -- +(\phii-\phij:\ri*\rj*12);
        \draw[very thin,-latex] (\i,-\j) -- +({\phii-\phij+(\omegai-\omegaj)*60}:\ri*\rj*12);
        \ifnum \omegai=\omegaj
        \draw[very thin,-latex] (\i,-\j) -- +(\phii-\phij:\ri*\rj*12);
        \fi
        \draw[very thin] (\phii-\phij:\ri*\rj*12)+(\i,-\j) arc (\phii-\phij:{\phii-\phij+(\omegai-\omegaj)*60}:\ri*\rj*12)+(\i,-\j);
        \node[minimum size=1cm] at (\i,-\j) (n\i\j) {};
      }
      \draw[thick] (n11.north west) to[bend right=6] node[midway,anchor=base east] {\Large $\rho(t)=$} (n18.south west);
      \draw[thick] (n88.south east) to [bend right=6] (n81.north east);
      \fill[opacity=0.2,niceblue] (n11.north west) rectangle (n33.south east);
      \fill[opacity=0.2,niceblue] (n44.north west) rectangle (n44.south east);
      \fill[opacity=0.2,niceblue] (n55.north west) rectangle (n66.south east);
      \fill[opacity=0.2,niceblue] (n77.north west) rectangle (n77.south east);
      \fill[opacity=0.2,niceblue] (n88.north west) rectangle (n88.south east);
    \end{tikzpicture}
    \caption{}
    \label{fig:maximumentropyprincipleb}
  \end{subfigure}\hfill
  \begin{subfigure}{0.3\textwidth}
    \begin{tikzpicture}[scale=0.45,transform shape]
      \path[use as bounding box] (-1,-0.3) rectangle (8.7,-8.7);
      \foreach \i/\ri/\phii/\omegai in {1/0.137264/-159.008/1,2/0.0919508/93.4659/1,3/0.204039/-110.476/1,4/0.14146/28.212/2,5/0.116946/145.35/4,6/0.128177/59.1885/4,7/0.113461/-153.794/5,8/0.0667032/-167.439/6}
      \foreach \j/\rj/\phij/\omegaj in {1/0.137264/-159.008/1,2/0.0919508/93.4659/1,3/0.204039/-110.476/1,4/0.14146/28.212/2,5/0.116946/145.35/4,6/0.128177/59.1885/4,7/0.113461/-153.794/5,8/0.0667032/-167.439/6}
      {
        % \draw[very thin,-latex] (\i,-\j) -- +(\phii-\phij:\ri*\rj*12);
        % \draw[very thin,-latex] (\i,-\j) -- +({\phii-\phij+(\omegai-\omegaj)*60}:\ri*\rj*12);
        \ifnum \omegai=\omegaj
        \draw[very thin,-latex] (\i,-\j) -- +(\phii-\phij:\ri*\rj*12);
        \fi
        % \draw[very thin] (\phii-\phij:\ri*\rj*12)+(\i,-\j) arc (\phii-\phij:{\phii-\phij+(\omegai-\omegaj)*60}:\ri*\rj*12)+(\i,-\j);
        \node[minimum size=1cm] at (\i,-\j) (n\i\j) {};
      }

      \draw[thick] (n11.north west) to[bend right=6] node[midway,anchor=base east] {\Large $\taverage{\rho}=$
      } (n18.south west);
      \draw[thick] (n88.south east) to [bend right=6] (n81.north east);
      \fill[opacity=0.2,niceblue] (n11.north west) rectangle (n33.south east);
      \fill[opacity=0.2,niceblue] (n44.north west) rectangle (n44.south east);
      \fill[opacity=0.2,niceblue] (n55.north west) rectangle (n66.south east);
      \fill[opacity=0.2,niceblue] (n77.north west) rectangle (n77.south east);
      \fill[opacity=0.2,niceblue] (n88.north west) rectangle (n88.south east);
    \end{tikzpicture}
    \caption{}
    \label{fig:maximumentropyprinciplec}
  \end{subfigure}
  \caption{Dephasing implies a maximum entropy principle. A quantum system started in an initial state $\rho(0)$ represented in panel (\subref{fig:maximumentropyprinciplea}) in an eigenbasis of its Hamiltonian $\H$ with degenerate subspaces corresponding to the squares, evolves (\subref{fig:maximumentropyprincipleb}) in a way such that time averaging its evolution (\subref{fig:maximumentropyprinciplec}) has the same effect as dephasing the initial state with respect to $\H$. The time averaged state $\taverage{\rho}$ is the state that maximizes the von Neumann entropy under the constraint that all conserved quantities give the same expectation value as in the initial state $\rho(0)$.}
  \label{fig:maximumentropyprinciple}
\end{figure}



\subsubsection*{Discussion}%
%
Theorem~\ref{thm:maximumentropyprinciple} is very reminiscent of Jaynes' maximum entropy principle.
It is however remarkable that it is not, as in Jaynes' approach, a postulate motivated by a subjective interpretation of probability that is taken as a starting point of a statistical theory, but a consequence of purely quantum mechanical considerations.
The unitary quantum dynamics of closed systems alone gives rise to a maximum entropy principle.

While this is remarkable, Theorem~\ref{thm:maximumentropyprinciple} is at the same time a bit disappointing, or at least not the whole story.
Note that it says that the equilibrium expectation values of all observables that equilibrate on average can be calculated from the state that maximizes the von~Neumann entropy given \emph{all} conserved quantities (compare also Ref.~\cite{a726f01}).
The number of \emph{all} linearly independent conserved quantities of a composite quantum system however increases exponentially with the number of constituents, and finding them usually again requires resources that scale exponentially with the system size.
The predictive power of Theorem~\ref{thm:maximumentropyprinciple} is hence rather limited.

The truly interesting question that arises from Theorem~\ref{thm:maximumentropyprinciple} is the following:
If one is interested in only, say, equilibrating local observables, how many, and which, conserved quantities does one need to keep fixed such that entropy maximization still yields a good approximation to their equilibrium expectation values?
In particular, one might expect that in this situation taking into account all, in some appropriate sense, \emph{local} conserved quantities should be sufficient.
One might even hope that in systems that do not have any such local conserved quantities, just fixing the expectation value of the energy and maybe the particle number could already suffice.
This has been investigated numerically in the context of the so-called \emph{generalized Gibbs ensemble} in Refs.~\cite{Sirker2013,1109.5904v1,Fagotti2012,Rigol08,Rigol11,1110.4690v1,Cazalilla11,Rigol07,Calabrese2007,Rigol2006,Cazalilla2006,Eckstein2008,Iucci2010,Fioretto2010,Cassidy11,1104.0154v1,Iucci2010,Fioretto2010,Iucci2009} (just to name a few references).
Unfortunately very little is known analytically \cite{1012.3587v1}, the results of Ref.~\cite{cramer10_1} being a noteworthy exception (see Section~\ref{sec:equlibrationinthestrongsense} for more details).
We will come back to this when we discuss thermalization in Section~\ref{sec:thermalization}.


\section{Decoherence}
\label{sec:decoherence}
%
The main aim of the \emph{decoherence program} is to explain the emergence of classical behavior in quantum systems as a consequence of a \emph{loss of coherence}, which is also called \emph{dephasing}, or just \emph{decoherence}.
A prime example of this is \emph{einselection}, which we will describe below.
It is the subject of an ongoing scientific debate to which extent decoherence provides a satisfactory explanation of the emergence of classicality and whether and to which extent it can solve the measurement problem \cite{Schlosshauer2005,0112095v3,0908.4069v1}.
It is beyond the scope of this thesis to provide a comprehensive review of decoherence theory.
More detailed information can for example be found in the reviews Ref.~\cite{RevModPhys.75.715,Schlosshauer2005}.

Decoherence theory is similar in spirit to the approach taken in this work in that it tries to justify a macroscopic theory based on purely microscopic, quantum mechanical considerations, thereby leading to a reconciliation of the two theories.
In fact, the connection exists not only on this meta level, but there also is a far more direct connection.
The results on equilibration of Ref.~\cite{Linden10} were used in Ref.~\cite{PhysRevE.81.05-1} to obtain a very general proof of decoherence under weak interactions.
In order to properly understand the meaning of this result we will first review some concepts from decoherence theory and then reproduce and discuss a bound on the speed of fluctuations around equilibrium from Ref.~\cite{Linden10}.

An important concept in decoherence theory is that of \emph{environment induced super selection (einselection)}, a term due to \textcite{RevModPhys.75.715}.
In short, einselection can be shown to occur in bipartite systems with $\Vset = S \dunion B$ that have a Hamiltonian $\H$ that commutes with the projectors onto an orthonormal basis of so-called \emph{pointer states} $(\ket p)_{p=1}^{d_S}$ for the subsystem $S$ \cite{PhysRevD.26.18,RevModPhys.75.715}.
The Hamiltonian and time evolution operator are then of the form
\begin{align}
  \H &= \sum_{p=1}^{d_S} \ketbra p p \otimes \H^{(p)} \label{eq:einselectionhamiltonian}\\
  \text{and}\quad U(t) &= \sum_{p=1}^{d_S} \ketbra p p \otimes U^{(p)}(t) ,
\end{align}
with $U^{(p)}(t) \coloneqq \exp(-\i\,\H^{(p)}\,t)$ for some Hamiltonians $\H^{(p)} \in \Obs(\mcH_B)$ on the subsystem $B$.
If the initial state is product with respect to the bipartition $\Vset = S \dunion B$, i.e., if
\begin{equation}
  \rho(0) = \rho^S(0) \otimes \rho^B(0) ,
\end{equation}
then a straight forward calculation shows that for all $t$ during the evolution
\begin{equation} \label{eq:timeevolvedreducedstatesundereinselectionhamiltonian}
  \rho^S(t) = \sum_{p,p'=1}^{d_S} \ketbra{p}{p} \rho^S(0) \ketbra{p'}{p'}\,\Tr(\rho^B(0)\, U^{(p)}(t)\ad\, U^{(p')}(t)) .
\end{equation}
This implies that the diagonal elements of $\rho^S(t)$ are conserved by the time evolution, i.e., $\bra{p} \rho^S(t) \ket{p} = \bra{p} \rho^S(0) \ket{p}$.
The off diagonal elements $\bra{p} \rho^S(t) \ket{p'}$ with $p\neq p'$ on the other hand are suppressed by a factor of $\Tr(\rho^B(0)\, U^{(p)}(t)\ad\, U^{(p')}(t)) \leq 1$, which in many situations decreases with $t$ on short time scales \cite{zeh96,PhysRevD.26.18,RevModPhys.75.715,Hornberger09}.
That is, an interaction of the form given in \texteqref{eq:einselectionhamiltonian} makes the state $\rho^S(t)$ almost diagonal in the basis of the pointer states.
This process is called \emph{einselection}.

The result on decoherence of Ref.~\cite{PhysRevE.81.05-1}, which we will discuss next, uses a theorem proved by \textcite{Linden10}.
The main concern of this work is to bound the \emph{speed} at which subsystems that equilibrate on average fluctuate around their equilibrium states.

As a measure of the \emph{speed} at time $t$ of the state $\rho^S(t) = \Tr_{\compl{S}}(\rho(t))$ of a subsystem $S$ during a time evolution $\rho \colon \R \to \Qst(\mcH)$ of a bipartite system with $\Vset = S \dunion B$ they propose
\begin{equation}
  v_S(t) \coloneqq \lim_{\Delta t \to 0} \frac{\tracedistance{\rho^S(t)}{\rho^S(t+\Delta t)}}{\Delta t} = \norm[1]{\frac{\del \rho^S}{\del t}(t)} ,
\end{equation}
where
\begin{equation}
  \frac{\del \rho^S}{\del t}(t) = \i\, \Tr_{\compl{S}}([\rho^S(t),\H]) .
\end{equation}
The main result of Ref.~\cite{Linden10} is an upper bound on the average speed of fluctuations.
Here, we give a slightly strengthened version of the result (remember the definition of the restricted Hamiltonian \texteqref{eq:restrictedhamiltonian}):
\begin{theorem}[Speed of fluctuations] \label{thm:speedoffluctuations}
  Given a Hamiltonian $\H \in \Obs(\mcH)$ of a composite quantum system with Hilbert space $\mcH$ and $\Vset = S \dunion B$ with spectral decomposition $\H = \sum_{k=1}^{d'} E_k\,\Pi_k$.
  Let $\H_I \coloneqq \H - \H_S - \H_B$ and for $\rho(0)$ the initial state of the system define the energy level occupations $p_k \coloneqq \Tr(\Pi_k\,\rho(0))$.
  Then
  \begin{equation}
    \taverage{v_S} \leq \norm[\infty]{\H_S+\H_I}\,d_S\,\sqrt{N(0)\,g((p_k)_{k=1}^{d'})} ,
  \end{equation}
  where $N(0)$ is the maximal number of degenerate energy gaps \eqref{eq:numeberofalmostdegenerategaps} and, as in Theorem~\ref{thm:equilibrationonaverage},
  \begin{equation}
    g((p_k)_{k=1}^{d'}) \coloneqq \min(\sum_{k=1}^{d'} p_k^2, 3  \maxprime_k p_k ) .
  \end{equation}
\end{theorem}
\begin{proof}
  Use \texteqref{eq:equilibrationonaverageforexpectationvalues} from Theorem~\ref{thm:equilibrationonaverage} for the case $T\to\infty$ instead of Eq.~(9) in the proof of Theorem~1 in Ref.~\cite{Linden10}.
\end{proof}

The above result essentially tells us that the conditions that are sufficient for equilibration on average (Theorem~\ref{thm:equilibrationonaverage}) also imply that, for most times $t$ during the evolution, the speed $v_S(t) \geq 0$ with which the state of a small subsystem changes is small compared to $\norm[\infty]{\H_S+\H_I}$.

We are now in the position to state the result of Ref.~\cite{PhysRevE.81.05-1}, which implies a behavior that is very similar to the einselection process described above.
\begin{theorem}[Decoherence under weak interaction \cite{PhysRevE.81.05-1}] \label{thm:decoherenceunderweakinteractoin}
  Let $\rho \colon \R \to \Qst(\mcH)$ be the time evolution of a bipartite quantum system with $\Vset = S \dunion B$ under a Hamiltonian $\H \in \Obs(\mcH)$ with the property that $\trunc{\H_S} S$ is non-degenerate.
  Denote the eigenvalues and normalized eigenstates of $\trunc{\H_S} S$ by $E^S_k$ and $\ket{E^S_k}$ respectively, then for all $t \in \R$
  \begin{align}
    \norm[\infty]{\H_S} + \frac{1}{2}\,v_S(t) &\geq \max_{\{(k,l)\}} \sum_{(k,l)} |E^S_k - E^S_l|\,|\bra{E^S_k} \rho^S(t) \ket{E^S_l}| \label{eq:decoherenceunderweakinteractoineq1}\\
    &\geq \max_{k,l\in[d_S]} |E^S_k - E^S_l|\, |\bra{E^S_k} \rho^S(t) \ket{E^S_l}| , \label{eq:decoherenceunderweakinteractoineq2}
  \end{align}
  where the maximization in \texteqref{eq:decoherenceunderweakinteractoineq1} is performed over all groupings of the elements of $[d_S]$ into disjoint pairs $(k,l)$.
\end{theorem}

Under the assumption that the energy level populations of the initial state are (except for the largest one) all sufficiently small and that $\H$ does not have any highly degenerate energy gaps (see Section~\ref{sec:equlibrationintheweaksense} for details), we know from Theorem~\ref{thm:speedoffluctuations} that $v_S(t)$ is much smaller than $\norm[\infty]{\H_S+\H_I}$ for all sufficiently small subsystems $S$ and most times $t$ during the evolution.
For those times, the left hand side of \texteqref{eq:decoherenceunderweakinteractoineq1} is approximately equal to $\norm[\infty]{\H_S}$.

Whenever this is the case, all off-diagonal elements of $\rho^S(t)$ in the eigenbasis of $\trunc{\H_S} S$ that belong to energy differences that are comparable in size with $\norm[\infty]{\H_S}$ must be small.
The latter is true not only for each of them individually, but, as \texteqref{eq:decoherenceunderweakinteractoineq1} shows, even sums of up to $\floor{d_S/2}$ of them, weighted with their respective energy difference, must be approximately smaller than $\norm[\infty]{\H_S}$.

Comparing this result with einselection, we see that if the interaction is weak enough compared to the energy gaps of $\trunc{\H_S} S$, then the eigenbasis of $\trunc{\H_S} S$ emerges as a natural ``pointer basis'' in the sense that the time evolution leads to decoherence in this basis, i.e., $\rho^S(t)$ is close to being diagonal in this basis for most times $t$ during the evolution.
Instead of the very specific assumptions on the interaction Hamiltonian and initial state we had to make to reach \texteqref{eq:timeevolvedreducedstatesundereinselectionhamiltonian}, we only needed the rather natural assumptions that also guarantee equilibration on average plus an assumption of sufficiently weak coupling.

\subsection*{Discussion}
%
A physically relevant situation where the above result seems particularly fitting is the following:
Imagine an atom or small molecule whose few lowest electronic excitations can be modeled by a finite dimensional quantum system with Hamiltonian $\H_S$.
The gaps of $\H_S$ typically correspond to visible or even ultra violet wave lengths.
Now, imagine the particle is part of a gas.
A coupling $\H_I$ that models collisions with other particles of the gas will have a strength that is comparable to the thermal energy of the particles in the gas and will hence be orders of magnitude smaller than the gaps of $\H_S$.
If the system is started in a suitable initial state and the Hamiltonian describing the gas has not too many degenerate energy gaps, Theorems~\ref{thm:speedoffluctuations} and \ref{thm:decoherenceunderweakinteractoin} imply that the atom or molecule is for most times during the evolution in a state that is close to diagonal in its energy eigenbasis, i.e., in a state that is a classical probabilistic superposition of its electronic energy eigenstates.

Theorem~\ref{thm:decoherenceunderweakinteractoin} does not imply anything physically meaningful about the time scales on which decoherence happens, but without further assumptions on the Hamiltonian \texteqref{eq:timeevolvedreducedstatesundereinselectionhamiltonian} from the usual decoherence approach is equally incapable of doing this.


\section{Typicality}
\label{sec:typicality}
%
Our disquisition of equilibration and decoherence has been literally very much in the spirit of \emph{pure state quantum statistical mechanics} in the sense that we have made statements that hold for all initial (and in particular also for pure) states that have certain properties.
We have thereby, up to now, managed to avoid the introduction of \emph{ensembles}, or as one could say not \emph{put any probabilities by hand}.

However, ensembles and averages with respect to certain postulated probability distribution do play important roles in statistical mechanics.
In this section we will review some arguments that can be used in the framework of pure state statistical mechanics to justify their use.
These approaches to explain the applicability of statistical mechanics are based on the insight that under certain assumptions most individual instances of a situation lead to a behavior that is very similarly to the average, or \emph{typical} behavior.

We will begin by reviewing the most influential articles on the subject in historic order, starting with the works of \textcite{Schroedinger1927} and von~\textcite{vonneumann1929}.
We will then state, prove, and discuss a general typicality theorem for uniformly random quantum state vectors.
We finish this section with a discussion of typicality in other ensembles and the most common objections against typicality arguments.

The strategy behind justifications for the use of ensembles is to argue that most states drawn according to some reasonable measure from a set of physically reasonable states have approximately the same properties, so that for computations it is practical to work with an average state.
This average state can, for example, turn out to be the state corresponding to a microcanonical or canonical ensemble.

The use of such \emph{typicality arguments} has a long history.
First considerations along these lines already appear in a work by \textcite{Schroedinger1927} from 1927.
After an introduction into (first order) perturbation theory and a discussion of resonance phenomena in quantum mechanics with a focus on energy exchange in weakly interacting systems he goes on discussing what he calls a ``statistical hypothesis''\footnote{German original \cite{Schroedinger1927}: \foreignlanguage{ngerman}{``Statistische Hypothese''}}.
He aims at describing the long time behavior of weakly interacting systems hoping to find thermodynamic behavior.
More specifically, he considers two systems that each have a pair of energy levels with the same gap.
The coupling between them that mixes the levels is assumed to be weak.
As his previous calculation had shown that the time averaged state depends on the initial state, he proposes to make an assumption about the initial energy level populations.
His assumption is that the populations of the levels are proportional to the products of the degrees of degeneracy of the non-interacting levels.
By introducing an entropy like quantity, he argues that if one of the systems is sufficiently large, this implies that when populations of energy levels whose reduced states on the small system are almost identical are combined, then the combined populations satisfy a canonical distribution.
By this, he effectively argues that initial states fulfilling his ``statistical hypothesis'' have reduced states on the small system that are well described by thermal states.

The concept of \emph{typicality} is even more prominent in an article by von~\textcite{vonneumann1929} from 1929.
His work has been translated by Tumulka \cite{Tumulka2010} and reviewed and refined by \textcite{0907.0108v1}.
Von~Neumann sets out to clarify ``how it can be that the known thermodynamic methods of statistical mechanics enable one to make statements about imperfectly (e.g., only macroscopically) known systems that in most cases are correct.''%
\footnote{German original \cite{vonneumann1929}: \foreignlanguage{ngerman}{``[\dots] wie es kommt, daß die bekannten thermodynamischen Methoden der statistischen Mechanik es ermöglichen, über mangelhaft (d.h.\ nur makroskopisch) bekannte Systeme meistens richtige Aussagen zu machen.''}}
He goes on describing that this means to clarify ``first, how the strange, seemingly irreversible behavior of entropy emerges, and second, why the statistical properties of the (fictitious) microcanonical ensemble can be assumed for the imperfectly known (real) systems, and that these questions will be tackled with the methods of quantum mechanics.''%
\footnote{German original \cite{vonneumann1929}: \foreignlanguage{ngerman}{``Insbesondere, wie erstens das eigentümliche, irreversibel scheinende Verhalten der Entropy zustande kommt, und warum zweitens die statistischen Eigenschaften der (fiktiven) mikrokanonischen Gesamtheit für das mangelhaft bekannte (wirkliche) System unterstellt werden dürfen. Und zwar sollen diese Fragen mit den Mitteln der Quantenmechanik angegriffen werden.''}}
He further argues that the phase space of classical systems \cite{Kinchin1949}, a central object in Gibbs' formulation of classical statistical mechanics \cite{Gibbs1902}, should, in the context of quantum mechanics, be replaced by a system of mutually commuting macroscopic observables that approximate the true non-commuting quantum observables.
Each sequence of eigenvalues of all macroscopic observables is associated with a \emph{phase cell}, i.e., the subspace spanned by the state vectors that all give precisely these measurement outcomes for the macroscopic measurements, but which are macroscopically indistinguishable from each other.
Following Ref.~\cite{0907.0108v1}, we denote the projector onto the phase cell characterized by the sequence $\nu$ of macroscopic measurement outcomes by $P_\nu$.
The approximation of the microscopic observables is to be taken coarse enough, such that, for example, the commuting macroscopic position and momentum observables do not get in conflict with Heisenberg's uncertainty relation for the true microscopic position and momentum operators.
One of von~Neumann's main results is his ``quantum ergodic theorem''\footnote{German original \cite{vonneumann1929}: \foreignlanguage{ngerman}{``Ergodensatz [...] in der neuen Mechanik''}}.
Essentially, he is able to show the following (for details see the original article and Theorem~1 in Ref.~\cite{0907.0108v1}):
Fix the dimensions $\rank(P_\nu)$ of the phase cells, if they are all neither too small not too large, then for any fixed Hamiltonian without degeneracies and non-degenerate energy gaps (see also Section~\ref{sec:equlibrationintheweaksense}), most decompositions of the Hilbert space into phase cells with these dimensions have the property that, for all initial states and most times during the evolution, the evolving state of the system and a suitable microcanonical state are approximately macroscopically indistinguishable.
This property is called ``normal typicality'' by the authors of Ref.~\cite{0907.0108v1}.
The result can actually be slightly generalized (Theorems~2 and 3 in Ref.~\cite{0907.0108v1}) and von~Neumann's theorem can be reformulated into a statement about all initial states, all decompositions into phase cells, and most Hamiltonians \cite{0907.0108v1}.

It is worth noting that the notion of typicality in Refs.~\cite{vonneumann1929,0907.0108v1} concerns not the quantum state (vector) but the set of macroscopic observables.
The statement holds for most decompositions of the Hilbert space in phase cells (with certain properties), or most Hamiltonians, but for \emph{all} initial states.
In the following, typicality will mostly concern the quantum state (vector), i.e., we will encounter statements that hold for \emph{most} (initial) state vectors.

Typicality arguments feature prominently in the PhD thesis of \textcite{slloydthesis} (see also Ref.~\cite{lloyed13}).
Essentially he shows that for any fixed observable, if quantum state vectors are drawn uniformly at random from a subspace of a Hilbert space (we will soon make this more precise), then the mean square deviation of the expectation value of the observable in such a random state from that in the corresponding microcanonical state is inverse proportional to the dimension of the subspace.

In a similar spirit, the concept of typicality is a cornerstone of the arguments in the book by \textcite{Gemmer09}.
As a measure of typicality the authors propose the \emph{Hilbert space variance} and derive bounds for the Hilbert space variance of various physically interesting quantities, ranging from expectation values of observables and distances of reduced states to entropies and purities.
As in Ref.~\cite{slloydthesis} and the present work, the aim is to use typicality to justify the methods of statistical mechanics and thermodynamics.

Many of the ideas of the works summarized above have later reappeared in an influential work by \textcite{Goldstein06} in which the term \emph{canonical typicality} was coined.
Ref.~\cite{Goldstein06} is intended to be a clarification and extension of the work of \textcite{Schroedinger1927}, which we discussed earlier, and remarks in his book \cite{Schroedinger1952} on statistical thermodynamics.
After a translation of the classical proof of the canonical ensemble from the microcanonical one to the quantum setting, the authors argue that the law of large numbers implies that if a state vector is drawn uniformly at random from a high dimensional subspace, its reduced state on a small subsystem will look similar to the reduced state of the microcanonical state corresponding to that subspace.

Before we go on, we must say more precisely what we mean by drawing a state vector \emph{uniformly at random} from a subspace.
Intuitively it should mean that any state from the subspace is as probable as any other.
Mathematically this is made precise in the notion of \emph{left/right invariant measures} \cite{halmos}.
Haar's theorem \cite{Haar1933} implies that for any finite $d$ there is a unique left and right invariant, countably additive, normalized measure on the unitary group $U(d)$ \cite{halmos}.
We refer to this measure as the \emph{Haar measure} on $U(d)$ and denote it by $\muhaar[U(d)]$.
Left and right invariant means that for any unitary $U \in U(d)$ and any Borel set $\mathscr{B}\subseteq U(d)$
\begin{equation}
  \muhaar[U(d)](\mathscr{B}) = \muhaar[U(d)](U\,\mathscr{B}) = \muhaar[U(d)](\mathscr{B}\,U) ,
\end{equation}
where $U\,\mathscr{B}$ and $\mathscr{B}\,U$ are the left and right translates of $\mathscr{B}$.
In this sense, the Haar measure $\muhaar[U(d)]$ is the uniform measure on $U(d)$.

The Haar measure on the group of unitaries that map a (restricted) subspace $\mcH_R \subseteq \mcH$ of dimension $d_R$ into itself induces in a natural way a uniform measure $\muhaar[\mcH_R]$ on state vectors $\ket\psi\in\mcH_R$.
We call state vectors drawn according to this measure, and also pure quantum states $\ketbra \psi \psi$ drawn according to the natural induced measure, \emph{Haar random} and write $\ket\psi \sim \muhaar[\mcH_R]$.

A practical way to obtain state vectors distributed according to this measure is to fix a basis $(\ket{j})_{j=1}^{d_R}$ for the subspace $\mcH_R$ and then draw the real and imaginary part of $d_R$ complex numbers $(c_j)_{j=1}^{d_R}$ from normal distributions of mean zero and variance one.
The state vector
\begin{equation} \label{eq:normalizedstateintermsofcoefficients}
  \ket{\psi} = \frac{\sum_{j=1}^{d_R} c_j \ket{j}}{\sqrt{\sum_{j=1}^{d_R} |c_j|^2}}
\end{equation}
is then distributed according to $\muhaar[\mcH_R]$, i.e., $\ket\psi \sim \muhaar[H_R]$ \cite{Zyczkowski2001}.
We will denote the probability that an assertion $\mathbb{A}(\ket{\psi})$ about a state vector $\ket\psi$ is true if $\ket\psi \sim \muhaar[\mcH_R]$ by $\probability_{\ket\psi \sim \muhaar[\mcH_R]}(\mathbb{A}(\ket\psi))$.

In the framework of \emph{measure theory} \cite{halmos}, \emph{typicality} can be seen as a consequence of the phenomenon of \emph{measure concentration} \cite{ledoux01,CHATTERJEE07}.
In particular a result known as \emph{Levy's lemma}, has been used in Refs.~\cite{Popescu06,Popescu05} to obtain theorems in the spirit of Refs.~\cite{slloydthesis,Goldstein06}, but with stronger bounds on the probabilities to observe large deviations from the (micro)canonical ensemble.
Refs.~\cite{Popescu06,Popescu05} focused mainly on reduced states of small subsystems of states drawn at random from high dimensional subspaces.
Based on the same techniques, in Ref.~\cite{Gogolin10-masterthesis}, similar results have been obtained for the expectation values of individual observables on the full system as well as their variances, and for sets of commuting observables, developing further ideas of Ref.~\cite{vonneumann1929} concerning macroscopic measurements.

Furthermore, an extension to the distinguishability under a restricted set of POVMs introduced in Section~\ref{sec:normsanddistancemeasures} is possible.
We summarize these results in a single theorem, which however is not optimal in terms of constants and scaling (compare Refs.~\cite{Popescu05,Gogolin10-masterthesis} for details).

\begin{theorem}[Measure concentration for quantum state vectors] \label{thm:measureconcentrationforquantumstatevectors}
  Let $R\subset\R$ and let $\mcH_R \subseteq \mcH$ be the subspace of the Hilbert space $\mcH$ of a system with Hamiltonian $\H \in \Obs(\mcH)$ that is spanned by the eigenstates of $\H$ to energies in $R$ and let $d_R \coloneqq \dim(\mcH_R)$.
  Then for every $\epsilon>0$ it holds that (i) for any operator $A \in \Bop(\mcH)$
  \begin{equation} \label{eq:measureconcentrationforobservables}
    \probability_{\ket\psi\sim\muhaar[\mcH_R]}\left( | \ex A {\ketbra\psi\psi} - \ex A {\rhomc[\H](R)} | \geq \epsilon \right) \leq 2\,\e^{-C\,d_R\,\epsilon^2/\|A\|_\infty^2} ,
  \end{equation}
  and (ii) for any set $\POVMs$ of POVMs
  \begin{equation} \label{eq:measureconcentrationforpovms}
    \probability_{\ket\psi\sim\muhaar[\mcH_R]}\left(\tracedistance[\POVMs]{\ketbra\psi\psi}{\rhomc[\H](R)} \geq \epsilon \right) \leq 2\,h(\POVMs)^2\,\e^{-C\,d_R\,\epsilon^2/h(\POVMs)^2} ,
  \end{equation}
  where $C = 1/(36\,\pi^3)$ and
  \begin{equation} \label{eq:definitionofhinthemeasureconcentrationtheorem}
    h(\POVMs) \coloneqq \min(|{\union \POVMs}|, \dim(\mcH_{\supp(\POVMs)})) .
  \end{equation}
\end{theorem}
Remember the definition of the microcanonical state $\rhomc[\H](R)$ from Section~\ref{sec:microcanonicalstates}.
\begin{proof}
  \texteqref{eq:measureconcentrationforobservables} is Theorem~2.2.2 from Ref.~\cite{Gogolin10-masterthesis}.
  We now prove \texteqref{eq:measureconcentrationforpovms} for $h(\POVMs)$ equal to the second argument of the $\min$ in \texteqref{eq:definitionofhinthemeasureconcentrationtheorem}.
  Let $S \coloneqq \bigcup_{M \in \union \POVMs} \supp(M)$ and remember that then for all $\rho,\sigma \in \Qst(\mcH)$
  \begin{equation}
    \tracedistance[\POVMs]\rho\sigma \leq \tracedistance{\rho^S}{\sigma^S} .
  \end{equation}
  Then Eq.~(75) in Section~VI.C of Ref.~\cite{Popescu05} yields the result.
  To finish the proof, note that \texteqref{eq:distinguishabilityunderrestrictedsetsofpovms} implies that for any $\rho,\sigma \in \Qst(\mcH)$
  \begin{align} \label{eq:distinguishabilityupperboundforproofofmeasureconcentrationtheorem}
    \tracedistance[\POVMs]\rho\sigma &\coloneqq \sup_{M \in \POVMs} \frac{1}{2}\,\sum_{k=1}^{|M|} |\Tr(M_k\,\rho) - \Tr(M_k\,\sigma)| \\
    &\leq \frac{1}{2}\,\sum_{M \in \union\POVMs } |\Tr(M\,\rho) - \Tr(M\,\sigma)| \\
    &\leq \frac{1}{2}\,|{\union\POVMs}| \sup_{M \in \union\POVMs } |\ex M \rho - \ex M \sigma | .
  \end{align}
  Together with Boole's inequality this yields that for every $\sigma \in \Qst(\mcH)$
  \begin{align}
    &\probability_{\ket\psi\sim\muhaar[\mcH_R]}\left(\tracedistance[\POVMs]{\ketbra\psi\psi}\sigma \geq \epsilon\right) \nonumber\\
    \leq &1 - \probability_{\ket\psi\sim\muhaar[\mcH_R]}\left(\bigcap_{M \in \union\POVMs} |\ex M {\ketbra\psi\psi} - \ex M \sigma |< \frac{2\,\epsilon}{|{\union\POVMs}|} \right) \\
    = &\probability_{\ket\psi\sim\muhaar[\mcH_R]}\left(\bigcup_{M \in \union\POVMs} |\ex M {\ketbra\psi\psi} - \ex M \sigma| \geq \frac{2\,\epsilon}{|{\union\POVMs}|} \right) \\
    \leq &\sum_{M \in \union\POVMs} \probability_{\ket\psi\sim\muhaar[\mcH_R]}\left( |\ex M {\ketbra\psi\psi} - \ex M \sigma| \geq \frac{2\,\epsilon}{|{\union\POVMs}|} \right) .
  \end{align}
  The proof of the result for $h(\POVMs)$ equal to the first argument of the $\min$ in \texteqref{eq:definitionofhinthemeasureconcentrationtheorem} can then be finished by choosing $\sigma = \rhomc[\H](R)$, using \texteqref{eq:measureconcentrationforobservables}, and the fact that for all $M \in \union \POVMs$ it holds that $\norm[\infty]{M} \leq 1$.
  Disregarding a favorable factor of $2$ and using the (highly non-optimal) bound $|{\union\POVMs}| < |{\union\POVMs}|^2$ yields the unified result as stated in the theorem.
\end{proof}

A physically particularly relevant case is when $\supp(\POVMs)$ is contained in some small subsystem $S \supseteq \supp(\POVMs)$ and $R = [E,E+\Delta]$ is some energy interval.
Then the theorem yields a probabilistic bound on the distance $\tracedistance{\ketbra\psi\psi^S}{\rhomc^S[\H]([E,E+\Delta])}$.
If $\ket\psi\sim\muhaar[\mcH_R]$ and the dimension $d_R$ of the microcanonical subspace $\mcH_R$ to the energies in the interval $[E,E+\Delta]$ fulfills $d_R \gg d_S$, then $\tracedistance{\ketbra\psi\psi^S}{\rhomc^S[\H]([E,E+\Delta])}$ is small with very high probability.
That is, the reduced state on $S$ of a random state from the subspace corresponding to the energy interval $R$ is indistinguishable from the reduction of the corresponding microcanonical state, with high probability.

The same holds in the more general setting that one has access only to a sufficiently small number of measurements, which in total have a sufficiently small number of different outcomes.
If the total number of different outcomes $|\union \POVMs|$ is much smaller than the dimension of the subspace corresponding to the energy interval $[E,E+\Delta]$, a random state from this subspace is with high probability indistinguishable from the microcanonical state.

For a family of Hamiltonians of locally interacting quantum systems with increasing system size, if $\Delta$ is kept fix and $E$ is chosen such that $R = [E,E+\Delta]$ is not too close to the boundaries of the spectrum of the Hamiltonian, then $d_R$ typically grows exponentially with the system size $|\Vset|$.
For a locally interacting system with a macroscopic number of constituents one would thus need to be able to distinguish an astronomically large number of different measurement outcomes to have a realistic chance of distinguishing a random state from a microcanonical state.

Similar methods as those used above were employed in Ref.~\cite{Linden09} to prove that for Haar random pure states from high dimensional subspaces the effective dimension (which we encountered in Section~\ref{sec:equlibrationintheweaksense}) with respect to a fixed Hamiltonian is of the order of the dimension of the subspace, with probability exponentially close to one.
The result can be generalized to certain measures over states that are product with respect to a bipartition $\Vset = S \dunion B$ \cite{Gogolin10-masterthesis}.

In addition to the Haar measure, other measures over quantum state vectors have been considered in the literature:

Refs.~\cite{Bender05,Brody07,1003.4982,1104.4625v1} introduce the \emph{mean energy ensemble}.
Instead of the uniform measure on a subspace corresponding to some energy interval, the mean energy ensemble consists of random state vectors which have a fixed energy expectation value with respect to some given Hamiltonian $\H$.
Under certain conditions on the spectrum of $\H$ it can be shown that the mean energy ensemble exhibits measure concentration \cite{1003.4982}.
In addition to that, it is possible to identify the typical reduced state of states drawn from the mean energy ensemble \cite{1003.4982}, and it can be shown that under certain conditions states from the mean energy ensemble typically have a high effective dimension \cite{Gogolin10-masterthesis}.

Ref.~\cite{Reimann07} considers an ensemble of quantum state vectors of the form given in \texteqref{eq:normalizedstateintermsofcoefficients}, in which the expansion coefficients $c_j = \braket{j}{\psi}$ have fixed modulus but random phases.
Concentration results, similar in spirit to Theorem~\ref{thm:measureconcentrationforquantumstatevectors}, can be shown for this ensemble that yield typicality whenever sufficiently many energy levels are populated.

Ref.~\cite{Bartsch09} extends the notion of typicality to the dynamics of systems.
Similarly as in the mean energy ensemble, the authors define an ensemble of initial states that share the same expectation value with respect to some given observable and then investigate the time evolution of this expectation value under a Hamiltonian.
The authors find \emph{dynamical typicality}, i.e., that states that initially give similar expectation values also typically lead to a similar dynamical evolution of these expectation values.

Typicality can also be used to speed up numerical calculations.
Instead of sampling over exponentially large sets of states, often drawing just a few representatives can already be sufficient to estimate expectation values \cite{Sugiura12}.
A new approach, which has recently been put forward in Refs.~\cite{Garnerone2010,Garnerone10-1,Garnerone2013,Garnerone2013a}, is to investigate and exploit typicality in the context of so-called \emph{matrix product states}.
The effects of typicality allow for the numerical approximation of thermal expectation values of observables in situations where naive approaches are infeasible \cite{Garnerone2013a}.
In Ref.~\cite{Steinigeweg2013} a method for numerically checking the validity of the \emph{eigenstate thermalization hypothesis} (see Section~\ref{sec:thermalizationunderassumptionsontheeigenstates}) is proposed that exploits techniques to apply exponentials of operators to random pure states.
Typicality ensures that only few such random states are needed to obtain conclusive results, thereby vastly reducing the computational cost.


\subsection*{Discussion}
%
Typicality arguments are frequently misunderstood and often criticized for being ``unphysical'' \cite{Bocchieri1958,Farquhar1957}.
Ref.~\cite{0907.0108v1}, for example, contains a very interesting review of the mostly negative reception of von~Neumann's quantum ergodic theorem (see also Section~\ref{sec:typicality}).
Whether the concept of typicality is really superior to other approaches towards the foundations of statistical mechanics and thermodynamics, such as \emph{ergodicity}, the principle of \emph{maximum entropy}, or \emph{ensembles}, is of course to some extent a matter of personal taste.

However, especially with respect to the latter, I think that typicality has an important advantage.
Instead of simply postulating that a certain ensemble yields a reasonable description of a certain physical situation, typicality shows, in a mathematically very well-defined way, when and why details do not matter.
If most states anyway exhibit the same or very similar properties, then this does provide a heuristic, but pretty convincing, argument in favor of the applicability of ensembles.
For me this is the main virtue of the typicality approach.
I see it not as a replacement of the ensemble approach, but rather as an argument supporting a description of large systems with ensembles.


\section{Time scales for equilibration on average}
\label{sec:timescales}
%
In this section we summarize what is known about the time scales on which subsystem equilibration to the reduction of the dephased state happens, i.e., on which time scales small subsystems equilibrate towards their time averaged state.
We will see that it is possible to go beyond what Theorem~\ref{thm:equilibrationonaverage} implies, but that all analytical results known to date that do so have the disadvantage of not being applicable to concrete Hamiltonians, but are only statements about all but few Hamiltonians from certain probability measures.

We will refer to and use methods of typicality that we discussed in Section~\ref{sec:typicality}.
This made it necessary to delay the discussion of this important topic until now.

We argued in the paragraphs following Theorem~\ref{thm:equilibrationonaverage} that the bounds in \texteqref{eq:equilibrationonaverageforexpectationvalues} and \texteqref{eq:equilibrationonaverageforrestricedpovms} can be expected to become meaningful only if $T$ is of the order of $\landauOmega(d_S^2\,d')$.
As $d'$ usually grows exponentially with the system size the equilibration times implied by Theorem~\ref{thm:equilibrationonaverage} become physically meaningless already for medium sized systems.

There are good reasons to believe that without further assumptions on the Hamiltonian no significantly better general bounds on the equilibration time can hold.
An example of a system that indeed can take exponentially long to equilibrate is a bipartite system in which the subsystem is only coupled to a low dimensional subspace of the Hilbert space of the bath.
It can then take exponentially long before the Hamiltonian on the bath has rotated the state of the bath into this subspace, thereby effectively leaving the subsystem uncoupled for extremely long times.
Such a couping to a low dimensional subspace is however necessarily non-local and hence unphysical.

Numerical evidence suggests that most natural, locally interacting systems started in reasonable initial states do not exhibit such extremely long equilibration times (see for example Refs.~\cite{Rigol08,Venuti09,1108.2703v1,Sirker2013,Fagotti2012,Eckstein2008,Gemmer09,Campos10,1108.2703v1,Rigol11,1110.4690v1,Rigol07,Rigol2006,1104.0154v1,Torres-Herrera2013}).
In addition, the results concerning equilibration during intervals of Ref.~\cite{cramer10_1}, which we presented in Section~\ref{sec:equlibrationinthestrongsense}, lead to more physical equilibration times.

As it is still unclear how the features of natural many body models, such as locality of interactions, can be exploited to derive tighter bounds on equilibration time scales, Refs.~\cite{1108.2985v3,1108.0374,1112.5295v1,1107.6035v1} instead consider \emph{random Hamiltonians}.
\textcite{1108.0374} go beyond the rather unrealistic scenario of Hamiltonians with Haar random eigenstates, which is why we will concentrate on this work in the following.

As a warm-up, we shall however consider exactly the situation of Hamiltonians with Haar random eigenvectors.
First, we define what a \emph{Haar random Hamiltonian} is:
Consider a system with Hilbert space $\mcH$ of dimension $d$ and fix an observable $G \in \Obs(\mcH)$.
Then for $U \sim \muhaar[U(d)]$ the operator
\begin{equation} \label{eq:defrandomhamiltonian}
  \H_G(U) \coloneqq U\,G\,U\ad
\end{equation}
is a \emph{Haar random Hamiltonian}.
Of course, $G$ and $\H_G(U)$ share the same spectrum and eigenvalue multiplicities for any unitary $U$, but the energy eigenstates / spectral projectors of $\H_G(U)$ are Haar random.
Fixing $G$ is thus equivalent to fixing the eigenvalues and degeneracies of the ensemble $\H_G(U),\ U \sim \muhaar[U(d)]$ of Haar random Hamiltonians.

A quantity that will play an important role in the theorems to come is
\begin{equation} \label{eq:deffouriertransformedcpectrum}
  f_G(t) \coloneqq \frac{1}{d}\,\sum_{k=1}^d \e^{- \i\,\tilde{E_k}\,t} ,
\end{equation}
where $(\tilde E_k)_{k=1}^d$ is the sequence of eigenvalues with respective multiplicity of $G$ (and hence also of $\H_G(U)$ for any unitary $U$).
The function $f_G$ can be interpreted as the \emph{Fourier transform} of the sequence $(\tilde E_k)_{k=1}^d$ \cite{1108.0374}.

We can now state the first result of Ref.~\cite{1108.0374}, which concerns quantum systems composed of so-called \emph{qubits}, i.e., quantum systems whose Hilbert space is $\C^2$:
\begin{theorem}[Equilibration under Haar random Hamiltonians {\cite[Result~1]{1108.0374}}] \label{thm:equilibrationunderrandomhamiltonians}
  Consider a bipartite system consisting of $|\Vset|$ many Qubits, i.e., $\Vset = S \dunion B$ and $\mcH = \bigotimes_{x\in\Vset} \mcH_{\{x\}}$ with $\mcH_{\{x\}} = \C^2$ for all $x \in \Vset$, starting in a fixed initial state $\rho(0) \in \Qst(\mcH)$.
  Then, for every $G \in \Obs(\mcH)$, every $t \in \R$, and every $\epsilon > 0$ it holds that
  \begin{equation} \label{eq:equilibrationunderrandomhamiltonsbound}
    \probability_{U \sim \muhaar[U(d)]}\left( \tracedistance{\rho^S(t)}{\omega_{H_G(U)}^S} > \frac{\sqrt{d_S}}{2\,\epsilon} \sqrt{ |f_G(t)|^4 + \frac{g_G^2}{d^2} + \frac{7}{d_B} }\right) < \epsilon ,
  \end{equation}
  where $\omega_{H_G(U)}^S \coloneqq \Tr_B(\$_{\H_G(U)}(\rho(0)))$ and $g_G \coloneqq \max_{k\in[d]} |\{l\oftype \tilde E_l = \tilde E_k \}|$ with $(\tilde E_k)_{k=1}^d$ the sequence of eigenvalues with respective multiplicity of $G$.
\end{theorem}
A very similar result is also contained in Ref.~\cite{1108.2985v3}.

Essentially, Theorem~\ref{thm:equilibrationunderrandomhamiltonians} connects the temporal evolution of the trace distance of $\rho^S(t)$ from the equilibrium state $\omega^S_G$ with the temporal evolution of $|f_G(t)|$.
If the bath is large and the Hamiltonian has only few degeneracies, then for most Haar random Hamiltonians the distance $\tracedistance{\rho^S(t)}{\omega_{H_G(U)}^S}$ is small whenever $|f_G(t)|$ is small.
This will make it possible to give bounds on equilibration time scales.

The above result can be extended to a more general ensemble of random Hamiltonians.
More specifically, consider again the setting of a composite system of $N$ Qubits and the ensemble $\H_G(U)$, but now with $G \in \Obs(\mcH)$ diagonal in some product basis and $U$ given by a \emph{random circuit} of \emph{circuit depth} $C \in Z^+$.
Here, a \emph{circuit} is a sequence of so-called \emph{quantum gates}, i.e., unitary quantum channels that each act on only one or two Qubits.
The gates can be members of a so called \emph{universal gate set}, i.e., a set of quantum gates such that any unitary can be approximated arbitrarily well by a circuit of gates from this set.
The \emph{circuit depth} of a circuit is the number of gates in the circuit.
Finally, a \emph{random circuit} is a circuit in which the gates have been drawn randomly according to some measure from a universal gate set.
We write $\muC$ for the measure on unitaries induced by random circuits of circuit depth $C$ with gates drawn uniformly at random from some fixed, finite universal gate set.
It is known that $\lim_{C\to\infty} \muC = \muhaar[U(d)]$  and that for large enough $C$ the measure $\muC$ approximates $\muhaar[U(d)]$ in the sense of being an \emph{approximate unitary design} \cite{Brandao2012}.
This holds regardless of which finite universal gate set is used.

For the random circuit ensemble of random Hamiltonians the following statement holds, which generalizes Theorem~\ref{thm:equilibrationunderrandomhamiltonians}:
\begin{theorem}[Equilibration under random circuit Hamiltonians {\cite[Result~3]{1108.0374}}] \label{thm:equilibrationundercircuitrandomhamiltonians}
   Consider a bipartite system consisting of $N \coloneqq |\Vset|$ many Qubits, i.e., $\Vset = S \dunion B$ and $\mcH = \bigotimes_{x\in\Vset} \mcH_{\{x\}}$ with $\mcH_{\{x\}} = \C^2$ for all $x \in \Vset$, starting in a fixed initial state $\rho(0)$.
  There exists a constant $\alpha \in \R$ that depends only on the universal gate set such that for every $G \in \Obs(\mcH)$ diagonal in a product basis, every $t \in \R$, every circuit depth $C \in \Z^+$, and every $\epsilon > 0$
  \begin{equation} \label{eq:equilibrationundercircuitrandomhamiltonsbound}
    \probability_{U \sim \muC}\left( \tracedistance{\rho^S(t)}{\omega_{H_G(U)}^S} > \frac{\sqrt{d_S}}{2\,\epsilon} \sqrt{ |f_G(t)|^4 + \frac{g_G^2}{d^2} + \frac{7}{d_B} + d^3\, 2^{-\alpha\,C/N}}\right) < \epsilon ,
  \end{equation}
  where $\omega_{H_G(U)}^S \coloneqq \Tr_B(\$_{\H_G(U)}(\rho(0)))$ and $g_G \coloneqq \max_{k\in[d]} |\{l\oftype \tilde E_l = \tilde E_K \}|$ with $(\tilde E_k)_{k=1}^d$ the sequence of eigenvalues with respective multiplicity of $G$.
\end{theorem}
As can be seen from \texteqref{eq:equilibrationundercircuitrandomhamiltonsbound}, a slightly super linear circuit complexity, i.e., $C = C(N) \notin \landauO(N)$, is sufficient to make the additional term in \texteqref{eq:equilibrationundercircuitrandomhamiltonsbound} (compared to \texteqref{eq:equilibrationunderrandomhamiltonsbound}) go to zero for large $N$.

If this is the case, and in addition $N$ is large enough, the bath is much larger than the subsystem, i.e., $d_B \gg d_S$, and $G$ has only few degeneracies, i.e., $g_G \ll d$, then the right hand side of both \texteqref{eq:equilibrationunderrandomhamiltonsbound} and \eqref{eq:equilibrationundercircuitrandomhamiltonsbound} is approximately equal to $|f_G(t)|^2\,\sqrt{d_S}/(2\,\epsilon)$.
Hence, the bounds are non-trivial for reasonably small $\epsilon$ for all $t$ for which $\sqrt{d_s}\,|f(t)|^2 \ll 1$.
For which times $t$ this is the case of course crucially depends on the spectrum that was fixed by fixing $G$.

The spectrum of the Ising model with transverse field, for example, leads to an approximately Gaussian decay of $|f(t)|^2$, implying an estimated equilibration time of the order of $\landauO(N^{-1/2})$ \cite{1108.0374}.
For more general locally interacting Hamiltonians on $D$-dimensional lattices one can show equilibration times of the order of $\landauO(N^{1/(5\,D)-1/2})$ \cite{1112.5295v1}.

This means that given an initial state $\rho(0)$, if $G$ is chosen to be the Hamiltonian of the transverse field Ising model and $U \sim \muhaar[U(d)]$, then the dynamics under the Haar random Hamiltonian $\H_G(U)$, which has the same spectrum as $G$, is, with high probability, such that the time evolution $\rho\oftype\R\to\Qst(\mcH)$ is such that the state of any small subsystem $S$ equilibrates to the reduced state of the dephased state on that subsystem, during a time of the order of $\landauO(N^{-1/2})$.


\subsection*{Discussion}
%
Do the above results solve the issue of equilibration times?
Despite them being very important contributions, I think it is fair to say that they still have at least two serious shortcomings:

First, the derived equilibration time scale appears to be rather unphysical.
Subsystems of larger systems should take longer to equilibrate, simply because excitations in locally interacting spin systems travel with a finite speed (see \cite{Kliesch2013} and the references therein).
One would expect that for locally interacting systems of $N$ spins on a $D$ dimensional regular lattice with nearest neighbor or short range interactions, subsystem equilibration should happen on a time scale of the order of $\landauTheta(N^{1/D})$, where $N^{1/D}$ is the \emph{linear size of the system}.
A subsystem equilibration time of the order of $\landauO(|N|^{-1/2})$, which becomes shorter with increasing system size, is clearly unphysical.

Second, irrespective of the above point, it is unfortunate that Theorems~\ref{thm:equilibrationunderrandomhamiltonians} and \ref{thm:equilibrationundercircuitrandomhamiltonians} do not say anything about concrete Hamiltonians, but are statements about typical Hamiltonians from an ensemble.
Even more worrying is the fact that one can show that for Haar random Hamiltonians the subsystem equilibrium state is the maximally mixed state \cite[Corollary 1]{1112.5295v1} and a similar statement can be shown for the random circuit ensemble of random Hamiltonians.
Systems to which the above results apply can thus never exhibit subsystem equilibration to an interesting, e.g., finite temperature, state.

The reason for both of these problems is that neither the model of Haar random Hamiltonians nor that of Hamiltonians whose diagonalizing unitary is given by a random circuit with high circuit complexity are good models for realistic, locally interacting quantum systems.
Simply put, even though random Hamiltonian ensembles have been successfully used to model certain features of realistic Hamiltonians in the context of \emph{random matrix theory} \cite{1102.0528v1,0412017v2,Guhr1998,1006.1634v1,Gemmer09,Tabor1989,Bohigas1984,Tao2012,mehta90}, the eigenstates of reasonable locally interacting quantum systems are far from Haar random.

What can be said for concrete Hamiltonians?
Ref.~\cite{Gogolin10-masterthesis} contains some \emph{lower} bounds on equilibration time scales.
For example, if a state has overlap only with energy eigenstates of the Hamiltonians in an energy interval of width $\Delta E$, then the equilibration time is at least of the order of $\landauOmega(1/\Delta E)$ (see also Ref.~\cite{PhysRevE.50.88}).
Similarly, if the Hamiltonian $\H$ of a bipartite system with $\Vset = S \dunion B$ is uncoupled, except for a small coupling Hamiltonian $H_I \coloneqq \H - \H_S - \H_B$, then the equilibration time is at least of the order of $\landauOmega(1/\norm[\infty]{\H_I})$ \cite[Section 2.6.3]{Gogolin10-masterthesis}.

Similarly, lower bounds on the equilibration/thermalization time follow from bounds on the rate of change of certain entropies \cite{MasterThesisHutter,Hutter11}.
In addition, in Ref.~\cite{Kastner11}, lower bounds on the equilibration time of the type $\landauOmega(N^{1/2})$ have been obtained for a class of spin systems with long range interactions.
For spin systems with short range interactions, Lieb-Robinson bounds immediately imply lower bounds on the equilibration time for certain initial states that are of the order of the linear size of the system.

Finally, in systems whose density of states can be approximated by a continuous function the \emph{Riemann-Lebesgue Lemma} \cite{Bochner49} can be used to give upper bounds on equilibration time scales \cite{Yukalov2011}.

Despite the large number of results the full problem still awaits a solution.


\section{Thermalization}
\label{sec:thermalization}
%
Given the findings presented in the last sections a natural question to ask is:
When do closed quantum systems in pure states that evolve unitarily not only equilibrate and decohere, but actually thermalize in the sense that under reasonable restrictions on the experimental capabilities they appear to be \emph{thermalized} or in \emph{thermodynamic equilibrium}?

To make this question meaningful we will define the term \emph{thermalization} in this section.
Then, in Section~\ref{sec:thermalizationunderassumptionsontheeigenstates} and \ref{sec:thermalizationunderassumptionsontheinitialstate}, we will discuss two complementary approaches to explain and understand thermalization in the framework of pure state quantum statistical mechanics in detail.
The first approach is the so-called \emph{eigenstate thermalization hypothesis} (ETH), the second is based on a quantum version of the classical derivation of the canonical ensemble from the microcanonical one, augmented with rigorous perturbation theory.
The first approach is based mostly on assumptions on the eigenspaces/eigenstates of the Hamiltonian, while the second one instead requires stronger assumptions on the initial state.
It is possible to interpolate between the two to some extent.
We will say more on that and on alternative notions of thermalization in Section~\ref{sec:othernotionsandhybridapproaches}.

Throughout this section a focus will be put on \emph{subsystem thermalization}, i.e., the thermalization of a small part (subsystem) of a large composite quantum system via the interaction with the rest of the system (bath).
The whole composite system (subsystem and bath together) is thereby assumed to be in a pure state evolving according to the standard Schrödinger/von Neumann equation under some Hamiltonian $\H$.
Let $S \subset \Vset$ be the vertex set of the subsystem and $B = \compl{S}$ that of the bath, then we will call the sum $\H_S + \H_B \eqqcolon \H_0$ of the two restricted Hamiltonians $\H_S$ and $\H_B$ (remember the definitions from Section~\ref{sec:localquantumsystems}) the \emph{non-interacting Hamiltonian} and $\H_I \coloneqq \H - \H_0$ the \emph{interaction Hamiltonian}.
%We will use $E^0_k$, $\Pi^0_k$ and $\ket{E^0_k}$ to denote the ordered energy eigenvalues, spectral projectors, and normalized eigenstates of $\H_0$.
We will say that a Hamiltonian $\H$ is non-interacting if $\H = \H_S + \H_B$.

Whenever the term \emph{bath} is used in the following it refers to this model of thermalization.
In particular we do not mean quantum systems that are already initially in a thermal state or other models of heat baths.
It is crucial to note that approaches that explain thermalization in quantum systems by investigating the behavior of systems coupled to such \emph{thermal baths} cannot solve the fundamental problem of thermalization, as they leave open the question how the thermal bath became thermal in the first place.

\vspace{2em}%hack!!!
\subsection{What is thermalization?}
\label{sec:whatisthermalization}
%
Whenever a term from one theory is used in a different context, a proper definition is mandatory.
This is particularly true for terms as involved as \emph{thermalization} and \emph{thermodynamic equilibrium} which, already in classical statistical mechanics, have several different meanings depending on the context.

To take account of the complex nature of the term thermalization we will not jump directly to a definition.
Instead, we will consider a number of conditions that each capture certain aspects of thermalization and whose fulfillment, depending on the context, one might or might not find necessary to say that a system has thermalized.
% In the following we will be mostly interested in the thermalization of small subsystems coupled to a much larger bath system, the two of which jointly evolve unitarily.
The catalog of properties that we will consider has been chosen with the setting of subsystem thermalization in mind.
Based on this discussion we will then carefully define what we consider sufficient to call a (sub)system thermalized, leaving open the possibility of defining other, possibly less strict, notions of thermalization.
In addition to that, we will also define the term \emph{subsystem initial state independence}, a property that we regard as a \emph{necessary} prerequisite for the thermalization of subsystems, and which we will discuss in more detail in Section~\ref{sec:absenceofthermalization}.

% Despite the careful assessment of the concept of thermalization our definitions will be less precise than the definitions in the sections on equilibration.
% This seems unavoidable given the complexity of the problem.
% Even though our definitions will fall short of a truly precise mathematical definition they are considerably more rigorous than most other characterizations of thermalization in closed quantum systems.

The aspects of thermalization that we will use as a guideline for our definition of thermalization are:
\begin{enumerate}
\item \label{item:thermalizationconditions:equilibration} \emph{Equilibration}:
  Equilibration is generally considered to be a necessary condition for thermalization.
  In the following we will mostly be concerned with \emph{subsystem equilibration on average} and \emph{apparent equilibration on average} of the whole system under restricted sets of POVMs (see also Section~\ref{sec:notionsofequilibration}).
\item \label{item:thermalizationconditions:subsysteminitialstateindependece} \emph{Subsystem initial state independence}:
  The equilibrium state of a small subsystem should be independent of the initial state of that subsystem.
  If a system exhibits some local exactly conserved quantities then one might still call it thermal and describe its equilibrium state by, for example, a so-called \emph{generalized Gibbs ensemble} \cite{Jaynes,PhysRev.108.17,Rigol07}.
  However, even such a behavior is often already considered to be non-thermal.
  We will take the more cautious point of view that a system should not be considered thermalizing if its equilibrium state depends on details of its own initial state, despite the absence of local exactly conserved quantities.
\item \label{item:thermalizationconditions:bathstateindependence} \emph{Bath state independence}:
  It is generally expected that the equilibrium expectation values of local observables on a small subsystem are almost independent of the details of the initial state of the rest of the system, but should rather only depend on its ``macroscopic properties'', such as the energy density, which one would expect to have an influence on the temperature of the thermalizing subsystem.
\item \label{item:thermalizationconditions:diafonalform} \emph{Diagonal form of the subsystem equilibrium state}:
  The equilibrium state of a small subsystem should be approximately diagonal in the energy eigenbasis of a suitably defined ``self-Hamiltonian''.
  If the interaction with the bath makes the state of the subsystem approximately diagonal in some basis one could call this \emph{decoherence} (see also Section~\ref{sec:decoherence}).
\item \label{item:thermalizationconditions:gibbsstate} \emph{Gibbs state}:
  Ultimately, one would like to recover the standard assumption of (classical) statistical physics that the equilibrium state is in some sense close to a Gibbs/thermal state.
\end{enumerate}

In the light of Condition~\ref{item:thermalizationconditions:equilibration} it seems to be a sensible approach to define thermalization \emph{on average} or \emph{during an interval} depending on the type of equilibration that goes along with thermalization.
Conditions~\ref{item:thermalizationconditions:subsysteminitialstateindependece} and \ref{item:thermalizationconditions:bathstateindependence} make clear that thermalization should be defined with respect to sets of initial states.
This leads us to the following definition of thermalization:
\begin{definition}[Thermalization on average] \label{def:thermlaizationonaverage}
  We say that a system with Hilbert space $\mcH$ and Hamiltonian $\H \in \Obs(\mcH)$ \emph{thermalizes} on average with respect to a set $\POVMs$ of POVMs and for a given set of initial states $\Qst_0 \subseteq \Qst(\mcH)$ if for each state $\rho(0) \in \Qst_0$, the system apparently equilibrates on average to an equilibrium state $\omega = \$_\H(\rho(0))$ that is close to a thermal state $\rhog[\H](\beta)$ for some $\beta \in \R$ in the sense that $\tracedistance[\POVMs]{\omega}{\rhog[\H](\beta)}$ is sufficiently small.
\end{definition}

Definition~\ref{def:thermlaizationonaverage} implicitly also defines \emph{thermalization on average of subsystems}.
Just choose $\POVMs$ to be the set of all POVMs with support on a subsystem $S$.
The Hamiltonian $\trunc {\H_S} S$ would then be a good choice for what we called \emph{self-Hamiltonian} in Condition~\ref{item:thermalizationconditions:diafonalform}.

Thermalization during intervals can be defined equivalently, but as we will not discuss it here, we keep the definition as simple as possible.

It seems worth emphasizing again that the above definition does not say that a system thermalizes if and only if the given conditions are met, but only says that it thermalizes if they are met.
It gives a set of conditions that are sufficient for thermalization.

An obvious question to ask is:
What are reasonable sets $\Qst_0$ of initial states?
Particularly important is the \emph{energy distribution} of the initial states, i.e., the sequence $(p_k)_{k=1}^{d'}$ of the energy level populations $p_k \coloneqq \Tr(\Pi_k\,\rho(0))$, as it is conserved under time evolution.
Taking the classical derivation of the canonical ensemble from the microcanonical one as a guideline, thermalization can only be expected to happen for initial states whose energy distribution is not too wide, i.e., the energies of the significantly populated levels must be in an interval small compared to $\norm[\infty]\H$.
We will see in Sections~\ref{sec:thermalizationunderassumptionsontheeigenstates} and \ref{sec:thermalizationunderassumptionsontheinitialstate} that such a condition will play an important role in proofs of thermalization.

In the above definition of thermalization on average we deliberately left open the question of what ``sufficiently small'' means.
This is ultimately to be decided in the specific situation at hand.
One would probably want that $\tracedistance[\POVMs]{\omega}{\rhog[\tilde H](\beta)}$ somehow suitably decreases with the size of the system.
However, we want to have a definition of thermalization that is applicable to finite systems.
Moreover, we want to avoid the technicalities of defining thermalization for sequences of quantum systems of increasing size.
This could have been be done in a similar way as in the definition of spreading transport (Definition~\ref{def:mixingtransport}), but would have lead to an equally involved definition.

We follow the same ratio when defining \emph{subsystem initial state independence}:
\begin{definition}[Subsystem initial state independence] \label{def:subsysteminitialstateindependence}
  We say that a composite system with Hilbert space $\mcH$ and Hamiltonian $\H \in \Obs(\mcH)$ satisfies \emph{subsystem initial state independence} for subsystem $S$ on average with respect to a given set of initial states $\Qst_0 \subseteq \Qst(\mcH)$ if for all $\rho(0) \in \Qst_0$ the equilibrium state on $S$ is sufficiently independent of its initial state in the sense that for every quantum channel $\Chann \in\Qch(\mcH)$ with support $\supp(\Chann) \subseteq S$ the states $\rho(0)$ and $\Chann(\rho(0))$ have the property that $\tracedistance{\Tr_{\compl{S}}[\$_\H(\rho(0))]}{\Tr_{\compl{S}}[\$_\H(\Chann(\rho(0)))]}$ is sufficiently small.
\end{definition}
In short: A system fulfills subsystem initial state independence if, for each initial state in $\Qst_0$, changing only the subsystem part of the initial state does not noticeably influence the equilibrium state of the subsystem.


\subsection{Thermalization under assumptions on the eigenstates}
\label{sec:thermalizationunderassumptionsontheeigenstates}
%
At the center of the first approach to show thermalization in quantum systems is the \emph{eigenstate thermalization hypothesis} (ETH).
There exist various version of the ETH in the literature and we will give a more precise definition below, but a minimal version of the ETH can informally be phrased as follows: ``A Hamiltonian fulfills the ETH if the expectation values of physically relevant observables in its energy eigenstates are approximately smooth functions of their energy.''
As we will see in this section, observables for which a system fulfills the ETH thermalize on average under reasonable conditions.
The ETH is usually said to date back to two works by \textcite{PhysRevA.43.20} and \textcite{PhysRevE.50.88}.
As the role of these works is however often misunderstood it is worth starting this section with a short historical review.

Already in 1985 \textcite{Jensen1985} investigated numerically how relatively small quantum systems equilibrate to a state that can be well described by statistical mechanics.
The computational power available at that time made it possible to study a spin-1/2 Ising chain with up to seven sites in a transverse field by means of exact diagonalization.
Ref.~\cite{Jensen1985} investigates the equilibration behavior of both global and local observables and compares time averages with microcanonical and canonical averages.
The authors conclude that ``both integrable and nonintegrable quantum systems with as few as seven degrees of freedom can exhibit statistical behavior for finite times.''
They also describe the reason for the statistical behavior they observe, which is essentially the mechanism that is today known as the ETH: ``If the expectation values [of an observable in the energy eigenstates] are smooth functions of the energy [\dots], then the short-time average of the observable will be very close to the ensemble average.''
In fact, it seems fair to say that the authors did anticipated large parts of the recent debate on equilibration and thermalization in closed quantum systems.
The last sentence of the abstract for example reads ``This work clarifies the impact of integrability and conservation laws on statistical behavior. The relation to quantum chaos is also discussed.''
It is remarkable that Ref.~\cite{Jensen1985} is nevertheless essentially completely ignored by almost the whole recent literature centered around such questions (Refs.~\cite{Yukalov2011,Reimann2013} being notable exceptions).

In Ref.~\cite{PhysRevA.43.20} a mechanism that can lead to the thermalization of quantum systems is identified, which the author calls \emph{eigenstate thermalization}.
A quantum and a classical version of a hard sphere gas serve as prototypical examples to illustrate this mechanism.
A central role is played by \emph{Berry's conjecture}.
It states that in certain quantum systems, whose classical counterparts exhibit \emph{classical chaos}, the energy eigenstates to energies in the bulk of the spectrum are superpositions of plain waves with random phases and random Gaussian amplitudes \cite{Berry1977}.
It is argued that in the hard sphere gas, whose classical version is indeed \emph{chaotic}, all energy eigenstates that satisfy Berry's conjecture have a single particle momentum distribution that is thermal.
Finally, thermalization is explained by the accumulation of relative phases between energy eigenstates due to time evolution.
This \emph{dephasing} destroys any fine tuned setting of the phases that might have been present in the coherent superposition of energy eigenstates that made up the initial state.
Such a fine tuning is necessary to get an initial state that is out of equilibrium.

Ref.~\cite{PhysRevE.50.88} aims at providing a quantum mechanical justification for the applicability of statistical ensembles.
The main idea is to model interacting composite quantum systems by starting with a non-interacting Hamiltonian that can be well understood, and then modeling generic effects of the interactions by adding a small random Hamiltonian --- very much in the spirit of random matrix theory \cite{Bohigas1984,Tao2012,mehta90}.
Due to the fact that composite quantum systems generically have exponentially dense spectra, i.e., either exponentially small gaps between neighboring eigenvalues and/or exponentially large degenerate eigenspaces, any extremely small perturbation will typically mix an exponentially large number of energy eigenstates of the non-interacting Hamiltonian.
This smears out their individual properties and should make the expectation values of physical observables in individual energy eigenstates of the perturbed Hamiltonian similar to those in a microcanonical state with a similar mean energy.

A much more rigorous formulation of the idea behind eigenstate thermalization can be found in a work by \textcite{tasaki98} (see also Ref.~\cite{Tasaki97}).
This article considers bipartite systems with $\Vset = S \dunion B$, whose non-interacting part $\H_0 = \H_S + \H_B$ of the Hamiltonian $\H = \H_0 + \H_I$ is non-degenerate.
The interaction Hamiltonian $\H_I$ is assumed to couple only neighboring energy levels, i.e., it is of the form
\begin{equation}
  \forall k\in[d]\itholds \bra{E^0_k} \H_I \ket{E^0_{k'}} = \lambda/2\,\delta_{|k-k'|,1}
\end{equation}
for some $\lambda\in\R$ such that $\levelspaceing^{\max}_B \ll \lambda \ll \levelspaceing^{\min}_S$ with $\levelspaceing^{\max}_B$ the maximal spacing between the energy eigenvalues of $\H_B$ and $\levelspaceing^{\min}_S$ the minimal level spacing of $\H_S$.
It is first argued heuristically and then proved, under some additional technical assumptions, that such Hamiltonians indeed exhibit eigenstate thermalization in the sense that for most $k$ and all observables $A_S$ with $\supp(A_S) \subseteq S$ it holds that $\bra{E_k} A_S \ket{E_k} \approx \Tr(A_S\, g[H_S](\beta(E_k)))$ (see Eq.~(5) and (6) in Ref.~\cite{Tasaki97}).

The \emph{eigenstate thermalization hypothesis} (ETH) gained wide popularity after a very influential article by \textcite{Rigol08}, which states the ETH as follows:
\begin{conjecture}[Eigenstate thermalization hypothesis as stated in Ref.~\cite{Rigol08}] \label{conjecture:ethrigolform}
  The expectation value $\bra{E_k} A \ket{E_k}$ of a few-body observable $A$ in an eigenstate $\ket E_k$ of the Hamiltonian, with energy $E_k$, of a large interacting many-body system equals the thermal [\dots] average of $A$ at the mean energy $E_k$.
\end{conjecture}
It is emphasized that \emph{thermal average} in this context can also mean the microcanonical average.
Ref.~\cite{Rigol08} studies a system of hard core bosons on a lattice.
It is demonstrated that the observed thermalization can be explained by the fact that certain physically relevant observables have expectation values in most energy eigenstates that indeed resemble those in a microcanonical state.

The validity of numerous variants of the ETH has been studied extensively, amongst others in Refs.~\cite{Santos10,Sirker2013,1102.0528v1,Biroli09,Singh,Ikeda11,Rigol08,Rigol11,1103.0787v1,Neuenhahn10,1108.0928v1,1111.6119,Mazets10,Polkovnikov11,Cassidy11,1104.3232v1,1004.2232v1,1201.0578v1,PhysRevLett.10-6,Polkovnikov11,Rigol09,1112.3424v1.pd,Cazalilla10,PhysRevB.82.17,Cazalilla11,Gritsev10,Iucci2009,Wilming2011,1108.0928v1,Steinigeweg2013,Beugeling2013,Ikeda2013a}.
The various \emph{eigenstate thermalization hypotheses} differ in whether they conjecture closeness to a microcanonical or a canonical average and concerning the type of observables they supposedly apply to.
\emph{Few body} and \emph{local} observables are the two most frequently encountered choices.

The bottom line of this large amount of (mostly numerical) investigations is as follows:
The energy eigenstates in the bulk of the spectrum, i.e., those to energies that are neither too low nor too high, of sufficiently large and sufficiently complicated composite quantum systems seem to generically fulfill some variant of the ETH for certain physically meaningful local or few body observables.
Many of the studies conclude that the fulfillment of the ETH is related to \emph{non-integrability} or \emph{chaos} \cite{1103.0787v1,1102.0528v1,Polkovnikov11,1108.0928v1,Neuenhahn10,Larson13,1201.0186v1,Beugeling2013,1112.3424v1.pd,Beugeling2013,Singh}.
Moreover, it is often suggested that systems fulfill the ETH and thermalize if and only if they are \emph{non-integrable} \cite{Rigol08,Rigol09,Rigol11,Biroli09,Znidaric09}.

The numerical investigation of the connection between quantum integrability and equilibration/thermalization has a long history \cite{Jensen1985}.
What precisely the term \emph{non-integrable} means in the context of many body quantum mechanics and especially in systems without a well-defined classical limit and the relation between \emph{\mbox{(non-)}integrability} and \emph{(exact) solvability} are however still the subject of a lively debate \cite{1111.3375v1,1012.3587v1,Braak11,Fine2013}.
We will come back to this issue in Section~\ref{sec:integrability}.

A slightly generalized and sharpened version of the ETH that captures the spirit of \emph{eigenstate thermalization} and applies to degenerate Hamiltonians is the following:
\begin{definition}[Eigenstate thermalization hypothesis (ETH)] \label{def:eth}
  A Hamiltonian $\H$ fulfills the \emph{eigenstate thermalization hypothesis} in a set $R \subset \R$ of energies with respect to a set $\POVMs$ of POVMs if and only if all its spectral projectors $\Pi_k$ to energies $E_k \in R$ have the property that there is a sufficiently smooth function $\beta\colon R \to \R^+$ such that for each $k$ with $E_k \in R$ it holds that for all normalized pure states $\psi \in \Qst(\mcH)$ with the property $\psi \leq \Pi_K$ the distinguishability $\tracedistance[\POVMs]{\psi}{\rhog[\H](\beta(E_k))}$ is sufficiently small.
\end{definition}
Again, we have deliberately left open what is meant by ``sufficiently smooth'' and ``sufficiently small''.

It is still open under which precise conditions the ETH holds in this or a similar form.
The rigorous derivations of Ref.~\cite{tasaki98} have so far not been generalized to more reasonable physical interactions.
Methods to analytically check the ETH in ``non-integrable models'' that are interesting in the context of condensed matter theory currently seem to be out of reach.
Very recently in Ref.~\cite{Mueller2013} a statement reminiscent of the ETH was proved under fairly general conditions.
More precisely, Ref.~\cite{Mueller2013} shows \emph{weak local diagonality} (Theorems~4 and 38) of the energy eigenstates of a certain type of Hamiltonian.
In the language used here a slightly simplified version of this statement can be formulated as follows:
\begin{theorem}[Weak local diagonality \cite{Mueller2013}]
  Consider a locally interacting spin system with Hilbert space $\mcH$ and Hamiltonian $\H \in \Obs(\mcH)$ whose interaction graph $(\Vset,\Eset)$ is a hypercubic lattice of spacial dimension $D$ and let $S \subset B \subset \Vset$ be subsystems.
  Then there exist constants $C,c,v > 0$, which depends only on $D$ and the local interaction strength $J \coloneqq \max_{X\in\Eset} \norm[\infty]{\H_X}$ of the Hamiltonian such that for any energy eigenstate $\ket E$ of $\H$ to energy $E$ there exists a state $\rho^B_E \in \Qst(\mcH_B)$ that satisfies for any two energy eigenstates $\ket{E^B_l}, \ket{E^B_m}$ of $\trunc {\H_B} B$ with energies $E^B_l$ and $E^B_m$
  \begin{equation}
    |\bra{E^B_l} \rho^B_E \ket{E^B_m}| \leq \e^{-\dist(S,\compl B)\,(E^B_l-E^B_m)^2/(8\,c\,v^2)}
  \end{equation}
  and at the same time
  \begin{equation}
    \norm[1]{\Tr_{B \setminus S}(\rho^B_E) - \Tr_{\Vset \setminus S}(\ketbra{E}{E}) } \leq C\,A^2\,J\,\sqrt{\frac{\dist(S,\compl B)}{4\,c\,v^2}}\,\e^{-c\,\dist(S,\compl B)/2} .
  \end{equation}
\end{theorem}
Essentially the theorem tells us that if $S$ is sufficiently far from the boundary of $B$, then for each energy eigenstate $\ket{E}$ of $\H$ there exists a state in $\Qst(\mcH_B)$ that is both approximately diagonal in the eigenbasis of $\trunc {\H_B} B$ and locally on $S$ hard to distinguish from $\ketbra{E}{E}$.
If one could improve the result to the effect that it would show local indistinguishability not only from an approximately diagonal state but from a thermal state then it would constitute a proof of an ETH like statement.
However, such a generalization can almost surely hold only under additional assumptions \cite{Mueller2013}.

The ETH, as defined in Definition~\ref{def:eth}, is sufficient for thermalization in the following sense:
\begin{observation}[Thermalization in systems that fulfill the ETH] \label{obs:ethissufficientforthermalization}
  Systems whose Hamiltonian $\H \in \Obs(\mcH)$ fulfills the ETH, as stated in Definition~\ref{def:eth}, for a set $R \subset \R$ of energies with respect to a set $\POVMs$ of POVMs, thermalize on average with respect to the set $\POVMs$, in the sense of Definition~\ref{def:thermlaizationonaverage}, for all initial states for which the system apparently equilibrates on average with respect the restricted set $\POVMs$ of POVMs (see also Section~\ref{sec:notionsofequilibration}) and whose energy distribution is sufficiently narrow and contained in $R$, i.e., $E_k \notin R \implies \Tr(\Pi_k\,\rho(0)) = 0$.
\end{observation}
That the ETH is sufficient for thermalization in this or a similar sense is widely known (see for example Ref.~\cite{tasaki98}).

It is worth noting that the strong requirement in Definition~\ref{def:eth} that the distinguishability $\tracedistance[\POVMs]{\psi}{\rhog[\H](\beta(E_k))}$ must be small for all normalized pure states $\psi \leq \Pi_K$ is crucial for the above observation to hold.
At the same time, this requirement obviously becomes harder to satisfy the more degenerate the Hamiltonian is.

If one takes the point of view that one should say that a system thermalizes only if it thermalizes in the sense of Definition~\ref{def:thermlaizationonaverage} for \emph{all} equilibrating initial states with a sufficiently narrow energy distribution, then fulfillment of the ETH is at the same time essentially also necessary for thermalization.
We will not make this statement fully rigorous, but the intuition behind it is as follows:
If the ETH is not fulfilled, there should always exist initial states with a narrow energy distribution that only have overlap with energy levels that, for certain observables or POVMs, produce a measurement statistic that is sufficiently far from that of the closest thermal state.
This distinguishability from the thermal state will then still be visible in the dephased state and hence survive dephasing and equilibration.

Such arguments, and the above mentioned apparent connection between the ETH and \mbox{(non-)}integrability, has lead some authors to proclaim \cite{Kollath07,Cramer2008,Rigol08,Rigol09,Banuls10} that non-integrable systems thermalize and integrable systems do not.
While there is evidence that in many models this is indeed the case, we will see in Section~\ref{sec:absenceofthermalization} and \ref{sec:integrability} that the situation is in fact more involved.


\subsubsection*{Discussion}
%
We have seen that the ETH as defined in Definition~\ref{def:eth} is by construction essentially sufficient and, in a certain sense, necessary for thermalization.
Is the problem of thermalization settled?
In my opinion, no.

First, a critical person could argue that the question ``When does a system thermalize?'' and the question ``When does a system fulfill the ETH?'' are essentially equally little understood \cite{Singh}.
I feel that, despite the great amount of new insights in the behavior of many body systems that was gained by investigating the validity of the ETH, such a person would have a point.
Saying that a system thermalizes if (and only if) it fulfills the ETH or is non-integrable (in some sense) is, at least at present, more a rephrasing of the problem rather than a solution.

Second, the ETH is necessary for thermalization only if one is willing to call a system thermalizing only if it thermalizes for a given set of POVMs for \emph{all} initial states with a sufficiently narrow energy distribution for which it also apparently equilibrates.
Hence, there is the possibility to show thermalization in systems that do not fulfill the ETH, if one is willing to restrict the class of allowed initial states.
As we will see in the following this can indeed be done.


\subsection{Thermalization under assumptions on the initial state}
\label{sec:thermalizationunderassumptionsontheinitialstate}
%
In this section we will discuss a second approach towards the problem of thermalization that is independent of the eigenstate thermalization hypothesis (ETH).
Instead of making strong assumptions concerning the properties of the energy eigenstates of the Hamiltonian we will show thermalization under stronger assumptions concerning the energy distribution of the initial state.
This alternative and complementing approach is inspired by an argument from classical statistical mechanics, which we will lift to the quantum setting.
The details of this approach were first worked out in Ref.~\cite{Riera2012}.

The first motivation for this work comes from the fact that explaining thermalization by using the eigenstate thermalization hypothesis has one important drawback --- that the ETH is indeed a \emph{hypothesis}.
One could make the provocative claim that this leads to the ironic situation that attempts to explain thermalization by the ETH have the following problem:
They essentially try to explain one phenomenon that is not well understood by another one that is almost as little understood.

The second motivation comes from the consideration that demanding thermalization of \emph{all} initial states with an energy distribution that is only required to be narrow but otherwise allowed to have arbitrary complex structure is asking for too much.

In the light of typicality arguments (Section~\ref{sec:typicality}) it seems plausible to restrict the class of initial states for which one tries to show thermalization, or even to be content with an argument that shows thermalization for most states from some measure.
In addition, certain restrictions on the initial states are anyway already necessary to prove equilibration on average in the first place (Section~\ref{sec:equlibrationintheweaksense}), and practical limits on the experimental capabilities can be used to argue that many initial states of macroscopic objects are essentially impossible to prepare \cite{Reimann12,Reimann2012,Reimann08}.

The third motivation comes from the known fact that in some systems the ETH is not fulfilled and this has been linked to the \emph{integrability} of these models, while \emph{non-integrability} is often associated with a fulfillment of the ETH and thermalization (see for example Refs.~\cite{Rigol07,Rigol08,Rigol11,Larson13,Polkovnikov10,Cassidy11,Gritsev10,Fioretto2010,1006.1634v1,Cazalilla11,1103.0787v1}).
What \emph{\mbox{(non-)}integrability} even means in the context of quantum mechanics is however far from settled \cite{1012.3587v1,Benet2003} (see also Section~\ref{sec:integrability}).
It is thus of interest to approach the problem of thermalization in a way that is independent of the concept of integrability.

As we will see in the following, restricting the class of initial states makes it possible to rigorously prove thermalization without any reference to the ETH for both spin and fermionic systems.
The overall structure of the argument is depicted in Fig.~\ref{fig:structureofthermalizationargument}.
The result that we will derive and discuss in this section can be combined with either the typicality theorems from Section~\ref{sec:typicality} or the dynamical equilibration theorems from Section~\ref{sec:equilibration}.
The former yields a kinematic thermalization statement (Observation~\ref{obs:thermalizationonofrandomstaates}) that holds for most Haar random states from a certain subspace.
The latter yields a dynamic thermalization result (Observation~\ref{obs:thermalizationonaverage}) that proves thermalization on average in the sense of Definition~\ref{def:thermlaizationonaverage} for all initial states from a certain class of states.
It is hence closer to the thermalization statement obtained under the ETH (Observation~\ref{obs:ethissufficientforthermalization}), which we discussed in the last section.

In essence, the proofs of the statements presented in this section are translations of the classical derivation of the canonical ensemble for small subsystems of large weakly interacting systems that are described by a microcanonical ensemble to the quantum setting.
The main difficulty is that in quantum mechanics the interaction between the small subsystem and the bath not only shifts the eigenenergies of the non-interacting Hamiltonian, but, in addition, significantly perturbs the energy eigenstates.
In many previous accounts of the thermalization problem this issue has been partially overlooked or at least not been addressed rigorously.
Compare for example Refs.~\cite{Popescu05,Popescu06,Goldstein06}.

\begin{figure}[bt]
  \centering
  \begin{tikzpicture}
    \node (plus) {$+$};
    \node[above of=plus,node distance=1.2cm] (goldstein) {
      \begin{minipage}{5.5cm}
          \begin{block}{}
            Classical level counting with no interaction\\\centerline{$\H_0 = \H_S + \H_B$}
          \end{block}
        \end{minipage}
      };
    \node[below of=plus,node distance=1.2cm] (perturbation) {
        \begin{minipage}{5.5cm}
          \begin{block}{}
            Perturbation theory for realistic weak coupling\\\centerline{$\|\H_I\|_\infty \ll k_B\,T$}
          \end{block}
        \end{minipage}
      };
    \coordinate (goldsteinne) at (goldstein.north east);
    \draw[rounded corners=0.2cm] (goldstein.north east) rectangle (perturbation.south west);
      \node[right of=goldstein,node distance=6cm,draw,rounded corners=0.2cm] (typicality) {
        \begin{minipage}{2.4cm}
          \begin{block}{}
            Typicality\\arguments
          \end{block}
        \end{minipage}
      };
      \draw[->] (goldsteinne |- goldstein) -- (typicality);
      \node[right of=typicality,node distance=3cm] (kinematic) {\emph{Kinematic}};
      \draw[->]  (typicality) -- (kinematic);
      \node[right of=perturbation,node distance=6cm,draw,rounded corners=0.2cm] (equilibration) {
        \begin{minipage}{2.4cm}
          \begin{block}{}
            Equilibration results
          \end{block}
        \end{minipage}
      };
      \draw[->] (goldsteinne |- perturbation) -- (equilibration);
      \node[right of=equilibration,node distance=3cm] (dynamic) {\emph{Dynamic}};
      \draw[->]  (equilibration) -- (dynamic);
    \end{tikzpicture}
  \caption{Structure of the proof of thermalization from Ref.~\cite{Riera2012}}
  \label{fig:structureofthermalizationargument}
\end{figure}

How does the interaction influence the Hamiltonian?
The eigenvalues of the interacting Hamiltonian are shifted at most by the operator norm of the interaction Hamiltonian with respect to those of the non-interacting Hamiltonian \cite[Theorem III.2.1]{bhatia}.
As long as the interaction is weak, in the sense that its operator norm is small compared to an energy uncertainty or measurement resolution, the change in the eigenvalues will thus be insignificant.

The energy eigenstates, or in the case of a degenerate Hamiltonian the spectral projectors, are much more fragile.
Naive perturbation theory breaks down \cite{Sakurai1995} as soon as the strength of the perturbation is larger than the gaps of the non-interacting Hamiltonian.
The gaps of a locally interacting quantum system are however usually exponentially small in the system size.
Indeed, if the non-interacting Hamiltonian $\H_0$ and the interaction Hamiltonian $\H_I$ are not diagonal in the same basis, the energy eigenstates of $\H = \H_0 + \H_I$ will usually be markedly different from those of $\H_0$.

Before we tackle this problem, let us consider the non-interacting case, i.e., a Hamiltonian of the form $\H_0 \coloneqq \H_S + \H_B$.
Let $\H_0$ and $\trunc {\H_S} S$ have spectral decompositions $\H_0 = \sum_k^{d'_0} E^0_k\,\Pi^0_k$ and $\trunc {\H_S} S = \sum_l^{d_S'} E^S_l\,\Pi^S_l$ respectively.
Moreover, let $(\ket{\tilde E^S_l})_{l=1}^{d_S}$ and $(\ket{\tilde E^B_m})_{m=1}^{d_B}$ be some orthonormal eigenbases with corresponding eigenvalues $(\tilde E^S_l)_{l=1}^{d_S}$ and $(\tilde E^B_m)_{m=1}^{d_B}$ of $\trunc {\H_S} S$ and $\trunc {\H_B} B$ respectively.
The Hamiltonians $\trunc {H_S} S$, $\trunc {H_B} B$, and $\H_0$ are allowed to have degeneracies, i.e., $l\neq l' \notimplies \tilde E^S_l \neq \tilde E^S_{l'}$ and  $m\neq m' \notimplies \tilde E^B_m \neq  \tilde E^B_{m'}$ and the bases are not unique.
Remember that, on the other hand, by definition, the elements of the sequences $(E^0_k)_{k=1}^{d'_0}$ and $(E^S_l)_{l=1}^{d'}$ are distinct.

We first look at the case of spin systems.
In such systems each of the spectral projectors $\Pi^0_k$ of $\H_0$ is of the form
\begin{equation}
  \Pi^0_k = \sum_{l,m\suchthat \tilde E^S_l+\tilde E^B_m=E^0_k} \ketbra{\tilde E^S_l}{\tilde E^S_l} \otimes \ketbra{\tilde E^B_m}{\tilde E^B_m} . \label{eq:energyprojectorsofnoninteractinghamiltonianinspinsystems}
\end{equation}
The microcanonical state $\rhomc[\H_0]([E,E+\Delta])$ to an energy interval $[E,E+\Delta]$ is hence proportional to
\begin{align}
  \rhomc[\H_0]([E,E+\Delta]) &\propto \sum_{k\suchthat E^0_k \in [E,E+\Delta]}\ \sum_{l,m\suchthat \tilde E^S_l+\tilde E^B_m=E^0_k} \ketbra{\tilde E^S_l}{\tilde E^S_l} \otimes \ketbra{\tilde E^B_m}{\tilde E^B_m} . \label{eq:firstequalityinthecountingargumentforspins}\\
  \intertext{Its reduced state $\rhomc^S[\H_0]([E,E+\Delta]) = \Tr_B \rhomc[\H_0]([E,E+\Delta])$ on $S$ therefore satisfies}
  \rhomc^S[\H_0]([E,E+\Delta]) &\propto \sum_{k\suchthat E^0_k \in [E,E+\Delta]}\ \sum_{l,m\suchthat \tilde E^S_l+\tilde  E^B_m=E^0_k} \ketbra{\tilde E^S_l}{\tilde E^S_l} \\
  &= \sum_{k\suchthat E^0_k \in [E,E+\Delta]}\, \sum_{l=1}^{d_S} \ketbra{\tilde E^S_l}{\tilde E^S_l} \, |\{m\oftype \tilde E^S_l+\tilde E^B_m=E^0_k \}| \\
  &= \sum_{k\suchthat E^0_k \in [E,E+\Delta]}\, \sum_{l=1}^{d_S'} \Pi^S_l \, |\{m\oftype E^S_l+\tilde E^B_m=E^0_k \}| \\
  &= \sum_{l=1}^{d_S'} \Pi^S_l \, |\{m\oftype E^S_l+\tilde E^B_m \in [E,E+\Delta] \}| \\
  &= \sum_{l=1}^{d_S'} \Pi^S_l \, \#_\Delta[\trunc{\H_B}B](E-E^S_l) ,
\end{align}
where
\begin{equation}
  \#_\Delta[\trunc{\H_B}B](E) \coloneqq |\{m\oftype \tilde E^B_m \in [E,E+\Delta] \}| = \rank\rhomc[\trunc{\H_B}B]([E,E+\Delta])
\end{equation}
is the \emph{number of orthonormal energy eigenstates} of the bath Hamiltonian $\H_B$ to energies in the interval $[E,E+\Delta]$.

For systems of fermions \texteqref{eq:energyprojectorsofnoninteractinghamiltonianinspinsystems} does not hold, because the Hilbert space of the joint system is not the tensor product of the Hilbert spaces of the subsystems.
However, the following quite lengthy calculation shows an equivalent result also for fermionic systems.
Readers not interested in the details can safely jump directly to Observation~\ref{obs:gibbsstatesasreductionsofmicrocanonicalstates}.

Denote by $f_x,f\ad_x$ the fermionic annihilation and creation operators on $\mcH$ and by $\tilde f_x, \tilde f\ad_x$ with $x \in S$ those acting on $\mcH_S$ and for $x \in B$ those acting on $\mcH_B$.
Furthermore, denote the vacuum state in $\mcH$ by $\ket{0}$ and the projectors in $\Bop(\mcH)$ onto the subspace with no particle in system $S$ or $B$ by $\ketbra{0}{0}_S$, and $\ketbra{0}{0}_B$ respectively.
The projectors $\ketbra{0}{0}_S$, $\ketbra{0}{0}_B$, and $\ketbra{0}{0}$ are all even operators and $\ketbra{0}{0} = \ketbra{0}{0}_S\,\ketbra{0}{0}_B$.
For each $l \in [d_S]$ let $p^{\H_S}_l$ be the representation of the eigenstate $\ket{\tilde E^S_l}$ as a polynomial in the fermionic operators on $\mcH_S$, i.e., $\ket{\tilde E^S_l} = p^{\H_S}_l((\tilde f_s,\tilde f\ad_s)_{s\in S})\,\ket{0}_S$, and likewise for $p^{\H_B}_m$.
Note that the $p^{\H_S}_l$ and the $p^{\H_B}_m$ are either even or odd polynomials as otherwise the projectors $\ketbra{\tilde E^S_l}{\tilde E^S_l}$ and $\ketbra{\tilde E^B_m}{\tilde E^B_m}$ would not be even.
Furthermore, note that commuting two polynomials that are both either even or odd gives a global minus sign only if both polynomials are odd.
As $\H_S$ and $\H_B$ are even operators it is strait forward to verify that the states $\ket{\tilde E^S_l + \tilde E^B_m} \coloneqq p^{\H_S}_l((f_s,f\ad_s)_{s\in S})\,p^{\H_B}_m((f_b,f\ad_b)_{b\in B})\,\ket{0}$ are eigenstates of $\H^0$ to energy $\tilde E^S_l + \tilde E^B_m$.
In fact, they form an orthonormal basis of $\mcH$ in which $\H^0$, $\H_S$, and $\H_B$ are jointly diagonal.
For the sake of brevity we omit the subscripts $_{s\in S}$ and $_{b\in B}$ in the following calculation.
It is again straight forward to verify that for any even operator $A \in \Bop(\mcH)$ with $\supp(A) \subseteq S$ it holds that
\begin{align}
  &\Tr\big(A\,\ketbra{\tilde E^S_l + \tilde E^B_m}{\tilde E^S_l + \tilde E^B_m}\big) \\
  = &\Tr\big(A\, p^{\H_S}_l((f_s,f\ad_s))\,p^{\H_B}_m((f_b,f\ad_b))\,\ketbra{0}{0}_S\,\ketbra{0}{0}_B\,p^{\H_B}_m((f_b,f\ad_b))\ad\,p^{\H_S}_l((f_s,f_s))\ad \big) \nonumber \\
  = &\Tr\big(A\, p^{\H_S}_l((f_s,f\ad_s))\,\ketbra{0}{0}_S\,p^{\H_S}_l((f_s,f\ad_s))\ad\,p^{\H_B}_m((f_b,f\ad_b))\,\ketbra{0}{0}_B\,p^{\H_B}_m((f_b,f\ad_b))\ad \big) \nonumber \\
  = &\Tr\big(A\,\ketbra{\tilde E^S_l}{\tilde E^S_l} \big) .
\end{align}
The last step can be shown by explicitly writing out the trace in the Fock basis and inserting an identity between the operators that are supported on $S$ and those supported on $B$.

Now, note that any operator $A \in \Bop(\mcH)$ with $\supp(A) \subseteq S$ can be written as a sum of an even and odd part and that only the even part can contribute to an expectation value of the form $\Tr(A\,\ketbra{\tilde E^S_l + \tilde E^B_m}{\tilde E^S_l + \tilde E^B_m})$.
The above calculation is hence sufficient to show that (remember the definition of the partial trace in \texteqref{eq:partialtrace})
\begin{equation}
  \forall l\in[d_S],m\in[d_B]\itholds \Tr_B(\ketbra{\tilde E^S_l + \tilde E^B_m}{\tilde E^S_l + \tilde E^B_m}) = \ketbra{\tilde E^S_l}{\tilde E^S_l} .
\end{equation}
Finally, realizing that
\begin{equation}
  \begin{split}
    \rhomc[\H_0]&([E,E+\Delta])\\
    &= \sum_{k\suchthat E^0_k \in [E,E+\Delta]}\ \sum_{l,m\suchthat \tilde E^S_l+\tilde E^B_m=E^0_k} \Tr_B(\ketbra{\tilde E^S_l + \tilde E^B_m}{\tilde E^S_l + \tilde E^B_m})
  \end{split}
\end{equation}
yields an expression equivalent to \texteqref{eq:firstequalityinthecountingargumentforspins} and the proof then proceeds analogously.

We summarize the result of the above calculation in the following observation:
\begin{observation}[Gibbs states as reductions of microcanonical states of the non-interacting Hamiltonians] \label{obs:gibbsstatesasreductionsofmicrocanonicalstates}
  Let $[E,E+\Delta]$ be an energy interval and $\H_0 = \H_S + \H_B$ a non-interacting Hamiltonian of a bipartite quantum system of spins or fermions with $\Vset = S \dunion B$. If for some $\beta \in \R$ it holds that
  \begin{equation} \label{eq:expoenntialdensityofstates}
    \#_\Delta[\trunc{\H_B}B](E) \propto \e^{-\beta\,E} ,
  \end{equation}
  then $\rhomc^S[\H_0]([E,E+\Delta])$ takes the well known form of a thermal state, i.e.,
  \begin{equation} \label{eq:noninteractingreducedmicrocanonicalstateisequaltocanonicalstate}
    \rhomc^S[\H_0]([E,E+\Delta]) \propto \sum_{l=1}^{d_S'} \Pi^S_l \, \e^{-\beta\,E^S_l} \propto \rhog[\trunc{\H_S}S](\beta) = \rhog^S[\H_0](\beta).
  \end{equation}
\end{observation}

Note how $\beta$, which was introduced in \texteqref{eq:expoenntialdensityofstates} simply as a parameter describing the shape of the number of states, ends up being the inverse temperature of the thermal state $\rhog[\trunc{\H_S}S](\beta)$ of the subsystem $S$.
Similar calculations (at least for spin systems) can be found for example in Refs.~\cite{Goldstein06,tasaki98,Reimann07,Gemmer09} and in many textbooks on statistical mechanics.

For finite dimensional baths the proportionality $\#_\Delta[\trunc{\H_B}B](E) \propto \e^{-\beta\,E}$ can never be exactly fulfilled simply because $\#_\Delta[\trunc{\H_B}B](E)$ is not continuous.
A detailed analysis \cite[Appendix A]{Riera2012} shows that if the logarithm of the number of states $\ln(\#_\Delta[\trunc{\H_B}B](E))$ can be sufficiently well approximated by a twice differentiable function whose second derivative is small compared to the width of the relevant energy range $[E-\norm[\infty]{\H_S},E+\norm[\infty]{\H_S}]$, then \texteqref{eq:noninteractingreducedmicrocanonicalstateisequaltocanonicalstate} is fulfilled approximately.
The first derivative of this approximation ends up being the inverse temperature of the thermal state, the second derivative enters the error bound.

It is widely known that natural locally interacting Hamiltonians $\H$ with bounded local terms ``generically'' have an approximately Gaussian number of states $\#_\Delta[\H](E)$ if the system size is sufficiently large \cite[Section 12.2]{Gemmer09} (see also Ref.~\cite{Hartmann2005} for some rigorous results).
It is more common to refer to the \emph{density of states} in this case, which is essentially the limit of $\#_\Delta[\H](E)/\Delta$ for $\Delta$ small and increasing system size.
If the bath Hamiltonian $\H_B$ is taken to be such a model with a nearly Gaussian density and number of states, the approximation by a twice differentiable function is possible and the distance $\tracedistance{\rhomc^S[\H_0]([E,E+\Delta])}{\rhog[\trunc{\H_S}S](\beta)}$ can be bounded \cite[Appendix B]{Riera2012} and is usually exponentially small in the size of the bath.
In the following we will call locally interacting systems that have this property ``generic''.

The value of $\beta$ for which $\tracedistance{\rhomc^S[\H_0]([E,E+\Delta])}{\rhog[\trunc{\H_S}S](\beta)}$ is small depends on $E$.
If $\#_\Delta[\trunc{\H_B}B]$ is indeed close to a Gaussian, then $\ln(\#_\Delta[\trunc{\H_B}B])$ can be well approximated by an inverted parabola.
Its first derivative, which is essentially the optimal $\beta$, is large for low values of $E$, thus associating them with low temperatures.
For values of $E$ in the center of the spectrum it goes to zero, corresponding to infinite temperature, and becomes negative for even higher values of $E$.

In conclusion, we can say that the reduction on $S$ of a microcanonical state to an energy interval $[E,E+\Delta]$ of a system that is a composite system with $\Vset = S \dunion B$ and without any interaction between $S$ and $B$, whose Hamiltonian $\H_B$ on $B$ is a ``generic'' many body Hamiltonian, will typically be exponentially close to a Gibbs state of $\H_S$ with an inverse temperature $\beta$ that depends in a reasonable way on $E$.
This works for all values of $E$ that are neither too low nor too high.
At the edges of the spectrum the number of states of the bath will be too low to allow for a good approximation of the number of states by a twice differentiable function.
In addition, $\Delta$ must be both small compared to $\norm[\infty]{\H}$ and large compared to the largest gaps in the spectrum of $\H$ in the relevant energy range.

Now we consider the influence of an interaction between $S$ and $B$.
The challenge posed by the fact that such an interaction will typically markedly perturb the energy eigenstates can be overcome by a perturbation theorem based on a result of \textcite{bhatia} (see also Refs.~\cite{Chandler70,bhatia83}) for projectors that are sums of spectral projectors.
\begin{theorem}[Stability of sums of spectral projectors (implied by Theorem~1 of Ref.~\cite{Riera2012})] \label{thm:stabiityofsumsofspectralprojectors}
  Given an energy interval $[E,E+\Delta]$ and two Hamiltonians $\H,\H' \in \Obs(\mcH)$ with spectral decompositions $H = \sum_k E_k\,\Pi_k$ and $H' = \sum_k E'_k\,\Pi'_k$.
  Let $P$ and $P'$ be projectors that are sums of the spectral projectors $\Pi_k$ and $\Pi'_k$ to energies in $[E,E+\Delta]$ of $\H$ and $\H'$ respectively, i.e.,
  \begin{align} \label{eq:defofPandPprime}
    P &\coloneqq \sum_{k\suchthat E_k\in[E,E+\Delta]} \Pi_k & &\text{and}& P' &\coloneqq  \sum_{k\suchthat E'_k\in[E,E+\Delta]} \Pi'_k .
  \end{align}
  Then for every $\epsilon>0$
  \begin{equation}
    \norm[1]{P-P'} \leq (\rank P + \rank P')\, \frac{\norm[\infty]{\H-\H'}}{\epsilon} + \rank P_\epsilon + \rank P'_\epsilon
  \end{equation}
  where
  \begin{align}
    P_\epsilon &\coloneqq \sum_{k\suchthat E_k\in[E,E+\epsilon]\union[E+\Delta-\epsilon,E+\Delta]} \Pi_k \\
    \intertext{and}
    P'_\epsilon &\coloneqq  \sum_{k\suchthat E'_k\in[E,E+\epsilon]\union[E+\Delta-\epsilon,E+\Delta]} \Pi'_k .
  \end{align}
\end{theorem}
% \begin{proof}
%   See proof of Theorem~1 in Ref.~\cite{Riera2012}.
% \end{proof}

The rather technical theorem stated above has immediate consequences for the stability of microcanonical states:
\begin{corollary}[Stability of microcanonical states \cite{Riera2012}] \label{corr:stabilityofmicrocanonicalstates}
  Given an energy interval $[E,E+\Delta]$ and two Hamiltonians $\H,\H' \in \Obs(\mcH)$ with spectral decompositions $H = \sum_k E_k\, \Pi_k$ and $H' = \sum_k E'_k\, \Pi'_k$ it holds that for every $\epsilon>0$
  \begin{equation} \label{eq:stabilityofmicrocanonicalstates}
    \tracedistance{\rhomc[\H]([E,E+\Delta])}{\rhomc[\H']([E,E+\Delta])} \leq \frac{\norm[\infty]{\H-\H'}}{\epsilon} + \frac{\Delta\Omega+\Omega_\epsilon}{2\,\Omega_{\max}} ,
  \end{equation}
  where $\Omega_{\min/\max} \coloneqq \min / \max (\rank \rhomc[\H]([E,E+\Delta]),\rank \rhomc[\H']([E,E+\Delta]))$, $\Omega \coloneqq \Omega_{\max} - \Omega_{\min}$, and
  \begin{align}
    \Omega_\epsilon \coloneqq &\rank \rhomc[\H]([E,E+\epsilon]\union[E+\Delta-\epsilon,E+\Delta]) \\
    + &\rank \rhomc[\H']([E,E+\epsilon]\union[E+\Delta-\epsilon,E+\Delta]) .
  \end{align}
\end{corollary}
\begin{proof}
  By the triangle inequality
  \begin{equation}
    \tracedistance{\rhomc[\H](I)}{\rhomc[\H'](I)} \leq \frac{\norm[1]{P-P'}+\Delta\Omega}{2\,\Omega_{\max}}
  \end{equation}
  with $P,P'$ defined as in \texteqref{eq:defofPandPprime}.
  Theorem~\ref{thm:stabiityofsumsofspectralprojectors} finishes the proof.
\end{proof}

What is the meaning of the corollary?
The statement is non-trivial if $\norm[\infty]{\H-\H'} \ll \Delta$.
Then one can expect that there exists an $\epsilon$ with the property that $\norm[\infty]{\H-\H'}\ll\epsilon\ll\Delta$, such that both $\norm[\infty]{\H-\H'}/\epsilon \ll 1$ and $(\Delta\Omega+\Omega_\epsilon)/(2\,\Omega_{\max}) \ll 1$.

Under the assumption of an approximately uniform density of states one finds that $\Omega_\epsilon/(2\,\Omega_{\max}) \approx 2\,\epsilon/\Delta$ and $\Delta\Omega/(2\,\Omega_{\max}) \lessapprox \norm[\infty]{\H-\H'}/\Delta$ such that the optimal choice for $\epsilon$ is approximately $\epsilon \approx \sqrt{\norm[\infty]{\H-\H'} \Delta/2}$, which yields
\begin{equation}
  \tracedistance{\rhomc[\H](I)}{\rhomc[\H'](I)} \lessapprox 4 \sqrt{\frac{\norm[\infty]{\H-\H'}}{\Delta}} .
\end{equation}

While the above example provides some intuition for how powerful Theorem~\ref{thm:stabiityofsumsofspectralprojectors} and Corollary~\ref{corr:stabilityofmicrocanonicalstates} are, the case of a uniform density of states is not the relevant situation if one is interested in showing thermalization.
As we have seen in the beginning of this section, for $\rhomc^S[\H_0]([E,E+\Delta])$ to become approximately thermal it is necessary that the number of states of the bath grows exponentially with $E$.
What happens in this case?

First, notice that the two terms in the right hand side of \texteqref{eq:stabilityofmicrocanonicalstates} are non-negative and hence must both be small individually for the inequality to become non-trivial.
For the interesting case $\H = \H_0 + \H_I$ and $\H' = \H_0$ this implies that it is necessary that $\norm[\infty]{\H_I} \ll \epsilon$, so that the fist term can become small.
For the second term we restrict our attention to $\Omega_\epsilon/(2\,\Omega_{\max})$ as $\Delta\Omega$ can reasonably be assumed to be smaller than $\Omega_\epsilon$.

If to good approximation
\begin{equation} \label{eq:approxexponentialdensityofstates}
  \#_\Delta[\H_0](E) \approx \#_\Delta[\H](E) \propto \e^{-\beta\,E} ,
\end{equation}
then \cite[Appendix H]{Riera2012}
\begin{equation}
  \frac{\Omega_\epsilon}{2\,\Omega_{\max}} \gtrapprox \frac{1-\e^{-\beta\,\epsilon}}{2\,(1-\e^{-\beta\,\Delta})} .
\end{equation}
That is, for Corollary~\ref{corr:stabilityofmicrocanonicalstates} to be non-trivial it must be possible to chose an $\epsilon$ such that
\begin{equation}
  \beta\,\norm[\infty]{\H_I} \ll \beta\,\epsilon \ll 1 .
\end{equation}
At the same time, if \texteqref{eq:approxexponentialdensityofstates} is fulfilled, then also \cite[Appendix H]{Riera2012}
\begin{equation}
  \frac{\Omega_\epsilon}{2\,\Omega_{\max}} \lessapprox \frac{\beta\,\epsilon}{1-\e^{-\beta\,\Delta}} .
\end{equation}
Under the reasonable assumption that $\Delta\Omega/(2\,\Omega_{\max}) \ll 1$ the choice $\epsilon = \sqrt{\norm[\infty]{\H_I}/\beta}$ yields
\begin{equation}
  \tracedistance{\rhomc[\H](I)}{\rhomc[\H_0](I)} \lessapprox 2 \frac{\sqrt{\beta\,\norm[\infty]{\H_I}}}{1-\e^{-\beta\,\Delta}} ,
\end{equation}
which gives a non-trivial upper bound as long as
\begin{equation} \label{eq:weakinteractioncondition}
  \norm[\infty]{\H_I} \ll 1/\beta \ll \Delta .
\end{equation}
Concluding, we can say that for reasonable bath Hamiltonians $\H_B$, and if the coupling is weak enough and $\Delta$ large enough such that \texteqref{eq:weakinteractioncondition} is fulfilled, then one can expect that
\begin{equation} \label{eq:expectedorderupperboundondeviationfromlocalthermal}
  \tracedistance{\rhomc^S[\H]([E,E+\Delta])}{\rhog[\trunc{\H_S}S](\beta)} \in \landauO\left(\sqrt{\beta\,\norm[\infty]{H_I}}\right) ,
\end{equation}
i.e., that the reduced state on subsystem $S$ of the microcanonical state is close to a Gibbs state of the restricted Hamiltonian truncated to $S$.

Corollary~\ref{corr:stabilityofmicrocanonicalstates} and the above discussion quantify the errors in the approximate equalities Eq.~(7) in Ref.~\cite{Popescu05} and Eq.~(18) in Ref.~\cite{Goldstein06}.

For the rest of this section we consider a bipartite quantum system with $\Vset = S \dunion B$ of spins of fermions with Hamiltonian $\H$.
Let $\H_0 \coloneqq \H_S + \H_B$ and $\H_I \coloneqq \H - \H_0$.
We are now in a position to state the \emph{kinematic} version of the thermalization result, which follows from the above discussion of Corollary~\ref{corr:stabilityofmicrocanonicalstates} and Theorem~\ref{thm:measureconcentrationforquantumstatevectors}.
\begin{observation}[Most Haar random states are locally thermal \cite{Riera2012}] \label{obs:thermalizationonofrandomstaates}
  Let $R \coloneqq [E,E+\Delta]$ be an energy interval and $\mcH_R \subseteq \mcH$ the subspace spanned by all eigenstates of $\H$ to energies in $R$ with dimension $d_R \coloneqq \dim(\mcH_R)$.
  If the bath has a ``generic'' locally interacting Hamiltonian with the property that for energies in $[E,E+\Delta]$ the logarithm of the number of states $\ln \#_\Delta[\trunc{\H_B}B]$ can be well approximated by an affine function with slope $\beta$ and if moreover $\Delta$ is sufficiently large and the interaction sufficiently weak such that
  \begin{equation}
    \norm[\infty]{\H_I} \ll 1/\beta \ll \Delta ,
  \end{equation}
  and the interval $R$ is sufficiently far from the edges of the spectrum, then for every $\epsilon > 0$
  \begin{equation} \label{eq:proababilityofrandomstatebeeingnotthermal}
    \begin{split}
      \probability_{\ket\psi\sim\muhaar[\mcH_R]}\left(\tracedistance{\ketbra\psi\psi^S}{\rhog[\trunc{\H_S}S](\beta)} \geq \epsilon + \delta(\H_B) + \landauO\left(\sqrt{\beta\,\norm[\infty]{H_I}}\right) \right)\\
      \leq 2\,d_S^2\,\e^{-C\,d_R\,{\epsilon}^2/d_S^2} ,
    \end{split}
  \end{equation}
  where $C = 1/(36\,\pi^3)$ and $\delta(\H_B)$ decreases fast with the size of the bath.
\end{observation}

To state the \emph{dynamic} result we introduce the notion of \emph{rectangular states} \cite{Riera2012}.
We call a state $\rho \in \Qst(\mcH)$ of a quantum system with Hilbert space $\mcH$ and Hamiltonian $\H \in \Obs(\mcH)$ \emph{rectangular} with respect to an energy interval $[E,E+\Delta] \subset \R$ if dephasing with respect to $\H$ yields the microcanonical state corresponding to $[E,E+\Delta]$.
For example, if $\H$ has no degeneracies, then a state is a rectangular state if, when expressed in the eigenbasis of $\H$, it has non-zero matrix elements only in some diagonal block and the same value for each entry on the diagonal in this block.
The class of rectangular states is not a very large class of states, but generally comprises a lot of pure states and usually also states that are \emph{out of equilibrium}, in the sense that their reductions on a small subsystem are well distinguishable from a thermal state and at the same time have a sufficiently widespread energy distribution such that Theorem~\ref{thm:equilibrationonaverage} can be used to guarantee equilibration on average.

% \begin{figure}[bt]
%   \centering
%   \hfill
%   \begin{subfigure}{0.3\textwidth}
%     \begin{tikzpicture}[scale=0.45,transform shape]
%       \path[use as bounding box] (-1,-0.3) rectangle (8.7,-8.7);
%       \foreach \i/\ri/\phii/\omegai in {1/0.137264/-159.008/1,2/0.0919508/93.4659/1,3/0.204039/-110.476/1,4/0.14146/28.212/2,5/0.116946/145.35/4,6/0.128177/59.1885/4,7/0.113461/-153.794/5,8/0.0667032/-167.439/6}
%       \foreach \j/\rj/\phij/\omegaj in {1/0.137264/-159.008/1,2/0.0919508/93.4659/1,3/0.204039/-110.476/1,4/0.14146/28.212/2,5/0.116946/145.35/4,6/0.128177/59.1885/4,7/0.113461/-153.794/5,8/0.0667032/-167.439/6}
%       {
%         \ifnum 2<\i \ifnum \i<7 \ifnum 2<\j \ifnum \j<7
%         \draw[very thin,-latex] (\i,-\j) -- +(\phii-\phij:\ri*\rj*15);
%         \fi \fi \fi \fi
%         \node[minimum size=1cm] at (\i,-\j) (n\i\j) {};
%       }

%       \draw[thick] (n11.north west) to[bend right=6] node[midway,anchor=base east] {\Large $\rho=$
%       } (n18.south west);
%       \draw[thick] (n88.south east) to [bend right=6] (n81.north east);
%       \fill[opacity=0.2,niceblue] (n33.north west) rectangle (n66.south east);
%     \end{tikzpicture}
%     \caption{}
%     \label{fig:rectangularstatesa}
%   \end{subfigure}\hfill
%   \begin{subfigure}{0.5\textwidth}
%     \begin{tikzpicture}[scale=0.7,transform shape]
%       \draw[->,thick] (0,0) -- (7,0) node[at end,below] {energy} ;
%       \draw[->,thick] (0,0) -- (0,5) node[midway,above,rotate=90] {population} ;
%     \end{tikzpicture}%
%     \caption{}
%     \label{fig:rectangularstatesb}
%   \end{subfigure}
%   \hspace*{\fill}
%   \caption{(\subref{fig:rectangularstatesa}): Example of a pure rectangular state represented in the eigenbasis of its Hamiltonian. (\subref{fig:rectangularstatesa}): Energy dististribution of a rectangular state and that of a state for which }
%   \label{fig:rectangularstates}
% \end{figure}

Nevertheless, all these states have a tendency to thermalize dynamically:
\begin{observation}[Thermalization on average \cite{Riera2012}] \label{obs:thermalizationonaverage}
  Let $R \coloneqq [E,E+\Delta]$ be an energy interval.
  Let the bath have a ``generic'' locally interacting Hamiltonian with the property that in an energy interval $[E,E+\Delta]$ the logarithm of the number of states $\ln(\#_\Delta[\trunc{\H_B}B])$ can be well approximated by an affine function with slope $\beta$.
  If $\Delta$ is sufficiently large and the interaction sufficiently weak such that
  \begin{equation}
    \norm[\infty]{\H_I} \ll 1/\beta \ll \Delta ,
  \end{equation}
  and the interval $R$ is sufficiently far from the edges of the spectrum, then the time evolution is such that the subsystem $S$ thermalizes on average, in the sense of Definition~\ref{def:thermlaizationonaverage}, for any initial state $\rho(0) \in \Qst(\mcH)$ that is rectangular with respect to $R$ in the sense that
  \begin{equation}
    \taverage[T]{\tracedistance{\rho^S(t)}{\rhog[\trunc{\H_S}S](\beta)}} \leq \sqrt{N(\epsilon)\,d_S^2\,g((p_k)_{k=1}^{d'}) }/2 + \delta(\H_B) + \landauO\left(\sqrt{\beta\,\norm[\infty]{H_I}}\right) ,
  \end{equation}
  where $\delta(\H_B)$ decreases fast with the size of the bath, and, as in Theorem~\ref{thm:equilibrationonaverage},
  \begin{align}
    N(\epsilon) &\coloneqq \sup_{E \in \R} |\{(k,l) \in [d']^2\suchthat k\neq l \land E_k - E_l \in [E,E+\epsilon] \}| \\
    g((p_k)_{k=1}^{d'}) &\coloneqq \min(\sum_{k=1}^{d'} p_k^2, 3  \maxprime_k p_k ) ,
  \end{align}
  with $(p_k)_{k=1}^{d'}$ the energy populations, i.e., $p_k \coloneqq \Tr(\Pi_k\,\rho(0))$, and $\maxprime_k p_k$ the second largest element in $(p_k)_{k=1}^{d'}$.
\end{observation}


\subsection*{Discussion}
%
The class of rectangular states seems fairly unnatural on first sight, and indeed the condition of being rectangular can be slightly weakened.
For small deviations from a rectangular state Observation~\ref{obs:thermalizationonaverage} still essentially holds, just an additional error must be taken into account.
If the deviation from rectangular is in a sense uncorrelated with the relevant properties of the energy eigenstates, then even relatively large deviations should be tolerable as the errors will not accumulate but rather cancel each other out.
In the worst case, however, the deviation from rectangular could be highly correlated with the expectation value of, say, a local observable.
Then, even small deviations from rectangular can lead to noticeable deviations of the equilibrium state from a thermal state.
We will see that this can indeed happen in natural models for natural initial states in Section~\ref{sec:anumericalinvestigationoftheviolationofinitialstateindependence} (see in particular Fig.~\ref{fig:absenceofthermalizationnumerics2}).
In this sense the condition of being rectangular is necessary for thermalization if no conditions on the energy eigenstates are to be imposed.

A comment on the notion of weak coupling used here is in order:
The condition that is needed for the above results to be non-trivial is (compare \texteqref{eq:expectedorderupperboundondeviationfromlocalthermal})
\begin{equation} \label{eq:weakcouplingcondition}
  \sqrt{\beta\,\norm[\infty]{H_I}} \ll 1 .
\end{equation}
This is a significant improvement over the condition that would be necessary to guarantee that naive perturbation theory on the level of individual energy eigenstates is applicable (namely that $\norm[\infty]{H_I}$ is much smaller than the gaps of $\H_0$).
While the gaps of $\H_0$ become exponentially small with the system size $\beta$ can be expected to be an intensive quantity, i.e., to be independent of the system size.

In the case of a 1D system with short range interactions and if $S$ is a set of consecutive sites $\norm[\infty]{H_I}$ is also intensive.
In this case, \texteqref{eq:weakcouplingcondition} is a physically natural condition to call the coupling \emph{weak}.
In the analogous situation in higher dimensional lattices, for example a system with nearest neighbor interactions on a 2D square lattice and $S$ the sites inside a ball around the origin, $\norm[\infty]{H_I}$ however scales with the surface of the region $S$, making the above bounds useless already for medium sized $|S|$.
Thus, the above results are not entirely satisfactory.

The reason for this is essentially that the trace distance is a very sensitive metric.
If $\rhomc^S[\H]([E,E+\Delta])$ and $\rhog[\trunc{\H_S}S](\beta)$ for the optimal $\beta$ only differ slightly on each of the sites along the boundary of $S$, then their trace distance (at least as long as it is sufficiently far from one) will be approximately proportional to the surface of $S$.
In consequence, the unfavorable scaling of the given error bounds is expected.

It should be possible to somehow improve the results by considering the distinguishability under POVMs that act only on parts of $S$ that are sufficiently far from the boundary by exploiting the locality of the interactions (see also Ref.~\cite{Zhang10}).
In general this seems difficult, but some steps in this direction are possible, and we will discuss this issue later in Section~\ref{sec:propertiesofthermalstatesofcompositesystems}.


\subsection{Hybrid approaches and other notions of thermalization}
\label{sec:othernotionsandhybridapproaches}
%
As we have seen in the last two sections, both approaches to explain thermalization, the eigenstate thermalization hypothesis and thermalization under assumptions on the initial state, have their advantages and drawbacks.
They can be understood as extreme scenarios.
In most cases where thermalization of closed quantum systems happens it is probably due to a mixture of the two effects.

An interpolation between the two previously discussed approaches is provided by the \emph{eigenstate randomization hypothesis} (ERH) of \textcite{Ikeda11}.
The ERH is a weaker condition than the ETH.
Instead of demanding that for certain observables the expectation values of all individual energy eigenstates with nearby energies give approximately the same expectation value (compare Conjecture~\ref{conjecture:ethrigolform} and Definition~\ref{def:eth}), the ERH requires only that the variance of certain coarse-grainings of the sequence of expectation values of an observable in the energy eigenstates becomes sufficiently small.
This, together with a condition on the smoothness of the energy distribution of the initial state that is milder than what we required when we introduced the class of rectangular states, is sufficient to prove a thermalization result that is similar in spirit to Observation~\ref{obs:thermalizationonaverage} \cite{Ikeda11}.
Again, numerical evidence for the validity of the ERH in certain models has been collected \cite{Ikeda11}.

As a final remark it seems worth repeating that the notion of thermalization used here is surely not the only reasonable one.
For example Ref.~\cite{0907.0108v1} works in the setting of macroscopic commuting observables of von~Neumann, which we discussed briefly in Section~\ref{sec:typicality}.
A system is declared to be in thermal equilibrium if there is a phase cell that is much larger than all others and the state of the system is almost completely contained in the subspace corresponding to this cell.

In the very recent work Ref.~\cite{Mueller2013} thermalization is investigated in the more specialized setting of translation invariant locally interacting system on cubic lattices.
In this work the state of a subsystem $S$ is considered thermal if it is close to the reduction of a thermal state of a larger subsystem $B \supset S$ and the distance decreases with increasing size of $B$, or more precisely with an increasing distance of $S$ from the complement of $B$.
Ref.~\cite{Mueller2013} contains theorems analogous to both the kinematic thermalization result (Observation~\ref{obs:thermalizationonofrandomstaates}) and the dynamical result on thermalization on average (Observation~\ref{obs:thermalizationonaverage}) that we discussed above.
The modified notion of thermalization makes the results of Ref.~\cite{Mueller2013} applicable in situations with a strong coupling of the subsystem to the bath where the results presented here become meaningless (see the discussion section of Section~\ref{sec:thermalizationunderassumptionsontheinitialstate}).
On the other hand, the authors of Ref.~\cite{Mueller2013} are unable to give concrete finite size bounds but are only able to make statements about the asymptotic behavior.
Moreover, the results of Ref.~\cite{Mueller2013} only work for temperatures at which the translation invariant system has a ``unique phase'' (see Ref.~\cite{Reed1980,Mueller2013} for more details) in the \emph{thermodynamic limit} of infinite lattice size.
At low temperatures this condition is often violated (for example in the 2D Ising model below the Curie temperature).

Many other definitions of thermalization or thermal equilibrium in quantum many body systems are possible.
For example, in the context of the ETH it is sometimes said that a system is \emph{thermal} if the expectation values of a given observable in the energy eigenstates of a system are, up to small fluctuations, smooth functions of the energy (compare for example Ref.~\cite{Beugeling2013}).



\section{Absence of thermalization}
\label{sec:absenceofthermalization}
%
This section complements the conditions derived in Section~\ref{sec:thermalization}, under which subsystem thermalization can be ensured, with conditions under which it can be guaranteed that a subsystem does not thermalize.
After first making more precise what type of absence of thermalization effect we will prove, we will, in Section~\ref{sec:violationofinitialstateindependence}, state, explain, and interpret the analytical results.
Finally, in Section~\ref{sec:anumericalinvestigationoftheviolationofinitialstateindependence}, the findings of numerical experiments are reported that demonstrate the relevance of the proven results.

As we had discussed in quite some detail in Section~\ref{sec:whatisthermalization}, finding a good definition for thermalization is a subtle issue.
The results concerning the absence of thermalization of this section do thus not refer to the arguable definition of thermalization on average we made in Definition~\ref{def:thermlaizationonaverage}, but instead give conditions under which \emph{subsystem initial state independence} is violated.

Of course, there are numerous, more or less trivial, reasons due to which a closed quantum system can fail to thermalize:

First, a system could simply fail to equilibrate.
Here we will not be interested in such situations, but concentrate on cases where a system does show some thermodynamic behavior, specifically equilibration on average, but still fails to thermalize.

Second, one can imagine situations where a system remains close to a non-thermal metastable state and only later, after a very long relaxation time, finally thermalizes on average.
The results concerning the absence of thermalization that we will state below are strong enough to exclude such situations.
They guarantee that the absence of thermalization is not a matter of time scales (see also Section~\ref{sec:timescales}).
Importantly, this also makes the results interesting for numerical tests of thermalization (see Section~\ref{sec:anumericalinvestigationoftheviolationofinitialstateindependence} for an example).

The kind of absence of thermalization effect we will prove is well illustrated by Fig.~\ref{fig:absenceofthermalizationnumerics2a}.
We will discuss this plot in more detail below.
For the moment it suffices to know that it shows the evolution of the expectation value of a local observable after a system has been started in two initial states that are identical on the bath but differ on a small subsystem so that the observable initially has expectation value $+1$ (upper line) or $-1$ (lower line) respectively.
As the plot shows, the expectation value first approaches and then fluctuates around an equilibrium value in both cases, but the two equilibrium values are well distinguishable.


\subsection{Violation of subsystem initial state independence}
\label{sec:violationofinitialstateindependence}
%
If a system does not exhibit any local exactly conserved quantities, subsystem initial state independence, as defined in Definition~\ref{def:subsysteminitialstateindependence}, with respect to a sufficiently large set of initial states $\Qst_0 \subset \Qst$, should be considered to be a necessary condition for thermalization, regardless of which precise definition of thermalization is adopted.

In this section we will provide conditions that are sufficient to guarantee a violation of \emph{subsystem initial state independence}.
A central role in the proofs is played by the entanglement in the spectral projectors of the Hamiltonian, or its energy eigenstates in the non-degenerate case.
For fermionic composite systems no unique and universally accepted definition of entanglement exists \cite{Banuls2007}.
Hence, we restrict ourselves to spin systems for the rest of this section.

As will become apparent shortly, it is a \emph{lack of entanglement in the eigenbasis} of the Hamiltonian that leads to the violation of subsystem initial state independence effect described here.
The central quantity in the following argument is the \emph{effective entanglement in the eigenbasis}.
Given a bipartite spin system with $\Vset = S \dunion B$, Hilbert space $\mcH$, and Hamiltonian $\H \in \Obs(\mcH)$ with spectral decomposition $\H = \sum_{k=1}^{d'} E_k\,\Pi_k$ we define for any pure state $\psi = \ketbra\psi\psi \in \Qst$ the \emph{effective entanglement in the eigenbasis} as
\begin{equation}
  R_{S|B}(\psi) \coloneqq \sum_{k=1}^{d'} p_k\,\tracedistance{\Tr_B(\Pi_k\,\psi\,\Pi_k)/p_k}{\psi^S} ,
\end{equation}
with $p_k \coloneqq \Tr(\Pi_k\,\psi)$ the energy level populations.
If the Hamiltonian is non-degenerate it takes the simpler form
\begin{equation} \label{eq:effectiveentranglementintheeigenbasisnondegenerate}
  R_{S|B}(\psi) = \sum_{k=1}^{d} p_k\,\tracedistance{\Tr_B(\ketbra{E_k}{E_k})}{\psi^S} .
\end{equation}
The name \emph{effective entanglement in the eigenbasis} will later be justified by Theorem~\ref{thm:entanglementintheeigenbasis}, which bounds $R_{S|B}$ by a quantity that is closely related to the \emph{geometric measure of entanglement}.
For the moment it suffices to know that for non-degenerate Hamiltonians the quantity $R_{S|B}(\psi)$, in a sense that is yet to be made precise, measures how entangled the eigenbasis of $\H$ looks like for the state $\psi$.
If the eigenstates of $\H$ are little entangled, and $\psi$ is a suitably chosen product state, then $R_{S|B}(\psi)$ will is small.

The intuition behind this statement is as follows:
If an energy eigenstate $\ketbra{E_k}{E_k}$ is little entangled, then $\Tr_B(\ketbra{E_k}{E_k})$ will be approximately a pure state.
But then the corresponding summand in \texteqref{eq:effectiveentranglementintheeigenbasisnondegenerate} can be large only if $\tracedistance{\Tr_B(\ketbra{E_k}{E_k})}{\psi^S}$ is large, i.e., if $\Tr_B(\ketbra{E_k}{E_k})$ and $\psi^S$ are quite different and at the same time $p_k = \Tr(\ketbra{E_k}{E_k}\,\psi)$ is large, which can happen only if $\ketbra{E_k}{E_k}$ and $\psi$ are similar.
As only one of the two previous conditions can be fulfilled for each $k$, each individual summand in \texteqref{eq:effectiveentranglementintheeigenbasisnondegenerate} is small and thus the whole sum not too large.

The effective entanglement in the eigenbasis is interesting because it quantifies how much closer the reduced states on $S$ of two different initial states can move during equilibration on average in the following sense:
\begin{theorem}[Distinguishability of dephased states {\cite[Theorem~1]{PhysRevLett.10-6}}] \label{thm:absenceofthermalizaton}
  Consider a bipartite spin system with $\Vset = S \dunion B$, Hilbert space $\mcH$ and Hamiltonian $\H \in \Obs(\mcH)$.
  For $j \in \{1,2\}$ let $\psi_j(0) = \psi^S_j(0) \otimes \psi^B_j(0) \in \Qst(\mcH)$ be two initial product states and set $\omega^{S(j)} \coloneqq \Tr_B(\$_\H(\psi_j(0)))$ then
  \begin{equation}
    \tracedistance{\omega^{S(1)}}{\omega^{S(2)}} \geq \tracedistance{\psi^S_1(0)}{\psi^S_2(0)} - R_{S|B}(\psi_1(0)) - R_{S|B}(\psi_2(0)) .
  \end{equation}
\end{theorem}
Remember that if the state of the subsystem $S$ equilibrates on average during the evolution under $\H$ for the two initial states, then the dephased states $\omega^{S(j)} = \Tr_B(\$_\H(\psi_{j}(0)))$ are the respective equilibrium states.
The theorem shows that if $R(\psi_1(0))$ and $R(\psi_2(0))$ are both small, then the subsystem equilibrium states $\omega^{S(1)}$ and $\omega^{S(2)}$ cannot be much less distinguishable than the initial states $\psi^S_1(0)$ and $\psi^S_2(0)$.

We now make rigorous the intuition that $R_{S|B}$ should be small for many initial states if $\H$ is non-degenerate and has little entanglement in its eigenbasis.
As said earlier, we are interested in situations where subsystem initial state independence is violated even though the subsystem equilibrates on average.
Equilibration on average can be ensured by using Theorem~\ref{thm:equilibrationonaverage} if $\sum_{k=1}^{d'} p_k^2$ is small and the Hamiltonian satisfies certain additional conditions (for details see Theorem~\ref{thm:equilibrationonaverage}).
The following theorem yields a constructive proof of the existence of many initial states that have both the properties needed to ensure equilibration on average according to Theorem~\ref{thm:equilibrationonaverage} and a small $R_{S|B}$ if $\H$ is non-degenerate and the eigenbasis is only little entangled:
\begin{theorem}[States with little effective entanglement in the eigenbasis] \label{thm:entanglementintheeigenbasis}
  Consider a bipartite spin system with $\Vset = S \dunion B$, Hilbert space $\mcH$, and non-degenerate Hamiltonian $\H \in \Obs(\mcH)$ with spectral decomposition $\H = \sum_{k=1}^d E_k \ketbra{E_k}{E_k}$.
  Let $(\ket{j})_{j=1}^{d_S}$ be an orthonormal basis for $\mcH_S$ and let $\mcH_R \subseteq \mcH_B$ be a subspace of $\mcH_B$ of dimension $d_R$.
  For every $j \in [d_S]$
  \begin{equation} \label{eq:averageeffectiveentanglementintheeigenbasisbound}
    \expectation_{\ket{\psi^B}\sim\muhaar[\mcH_R]}\left( R_{S|B}( \ketbra{j}{j} \otimes \ketbra{\psi^B}{\psi^B} ) \right) \leq 2\,d_S\,\delta ,
  \end{equation}
  where $\delta \coloneqq \max_{k\in[d]} \delta_k$ with
  \begin{equation} \label{eq:geometricentanglementdelta}
     \delta_k \coloneqq \min_{j\in [d_S]} \tracedistance{\Tr_B \ketbra{E_k}{E_K}}{\ketbra{j}{j}} ,
  \end{equation}
  and at the same time
  \begin{equation} \label{eq:highaverageeffectivedimension}
    \probability_{\ket{\psi^B}\sim\muhaar[\mcH_R]}\left( \sum_{k=1}^{d} p_k^2 > 4/d_R \right) \leq 2\,\e^{-C\,\sqrt{d_R}} ,
  \end{equation}
  with $C \coloneqq (\ln 2)^2/(72\,\pi^2)$.
\end{theorem}
$\expectation_{\ket{\psi^B}\sim\muhaar[\mcH_R]}$ and $\probability_{\ket{\psi^B}\sim\muhaar[\mcH_R]}$ respectively denote the expectation value and the probability with respect to Haar random state vectors $\ket{\psi^B}$ from the subspace $\mcH_R$ (for more details on the Haar measure, measure concentration, and typicality see Section~\ref{sec:typicality}).
\begin{proof}
  Without the additional claim in \texteqref{eq:highaverageeffectivedimension} the theorem is just Theorem~2 from Ref.~\cite{PhysRevLett.10-6} and \texteqref{eq:highaverageeffectivedimension} follows immediately from Theorem~2 in Ref.~\cite{Linden09}.
\end{proof}

We summarize the implication of the theorem in the following observation:
\begin{observation}[Absence of initial state independence] \label{obs:absenceofthermalization}
  Consider a bipartite spin system with $\Vset = S \dunion B$ and Hilbert space $\mcH$.
  Let $\mcH_R \subseteq \mcH$ be a subspace of dimension $d_R \coloneqq \dim(\mcH_R)$.
  If $d_R$ is large and the Hamiltonian $\H \in \Obs(\mcH)$ has not too many degenerate energy gaps (see Theorem~\ref{thm:equilibrationonaverage} for details) and an orthonormal basis $(\ket{j})_{j=1}^{d_S}$ for $\mcH_S$ exists for which $\delta$ in \texteqref{eq:geometricentanglementdelta} is small, then for every $j,j' \in [d_S]$ there exist many initial states of the bath $\psi^B(0) \in \Qst(\mcH_B)$ such that according to Theorem~\ref{thm:equilibrationonaverage} both $\ketbra{j}{j} \otimes \psi^B(0)$ and $\ketbra{j'}{j'} \otimes \psi^B(0)$ lead to subsystem equilibration on average, but despite them having exactly the same initial state on the bath, the corresponding subsystem equilibrium states $\omega^{S(j)}$ and $\omega^{S(j')}$remain well distinguishable for most times during the evolution, in the sense that their trace distance $\tracedistance{\omega^{S(j)}}{\omega^{S(j')}}$ is significantly larger than zero whenever $j \neq j'$, because of Theorem~\ref{thm:absenceofthermalizaton}.
\end{observation}
A couple of comments concerning Theorem~\ref{thm:entanglementintheeigenbasis} are in order:

At first sight, it seems unfortunate that $d_S$ appears in the right hand side of \texteqref{eq:averageeffectiveentanglementintheeigenbasisbound}.
After all, one would expect that the larger $S$ is, the easier it should be for $S$ to retain memory of its initial state.
Instead, the theorem seems to quickly become trivial for larger subsystems $S$.
However, this is not really a limitation.
If the subsystem $S$ of interest is itself a composite system, then for any interior subsystem $S^i \subset S$ and any two quantum states $\rho,\sigma \in \Qst(\mcH)$ it holds that $\tracedistance{\rho^S}{\sigma^S} \geq \tracedistance{\rho^{S^i}}{\sigma^{S^i}}$, i.e., for two states to be well distinguishable on $S$ it is sufficient that they can be well distinguished on any interior subsystem $S^i \subset S$.
Thus, Theorems~\ref{thm:entanglementintheeigenbasis} and \ref{thm:absenceofthermalizaton} can be applied to $S^i$ instead of $S$ and this then still gives a lower bound on $\tracedistance{\Tr_B[\$_\H(\psi_1(0))]}{\Tr_B[\$_\H(\psi_2(0))]}$, but in terms of $d_{S^i}$ instead of $d_S$.

Earlier it was claimed that Theorem~\ref{thm:entanglementintheeigenbasis} bounds $R_{S|B}$ by a quantity that is related to the geometric measure of entanglement.
The quantity that was meant is $\delta = \max_{k\in[d]} \delta_k$.
But \texteqref{eq:geometricentanglementdelta} does not quite look like the standard definition of the geometric measure of entanglement $\geometricentanglement_{S|B}$ (see \texteqref{eq:geometricmeasureofentanglementdefinition}).
However, it holds that \cite[Chapter 9.2]{nielsenchuang}
\begin{align}
  \delta_k &= \min_{j\in [d_S]} \tracedistance{\Tr_B(\ketbra{E_k}{E_K})}{\ketbra j j} \\
  &\leq \inf_{\ket\phi \in \mcH_B} \min_{j\in [d_S]} \tracedistance{\ketbra{E_k}{E_K}}{\ketbra j j \otimes \ketbra\phi\phi} \\
  &= 1 - \sup_{\ket\phi \in \mcH_B} \max_{j\in [d_S]} | \bra{E_k} (\ket j \otimes \ket\phi) |^2 . \label{eq:deltaklooksalmostlikethegeometricmeasureofentanglement}
\end{align}
So, $\delta_k$ is almost upper bounded by $\geometricentanglement_{S|B}(\ket{E_k})$.
Instead of the supremum over all $\ket{\psi_Y} \in \mcH_Y$ in \texteqref{eq:geometricmeasureofentanglementdefinition} equation \eqref{eq:deltaklooksalmostlikethegeometricmeasureofentanglement} only contains a maximum over all $\ket j$ in the basis $(\ket j)_{j=1}^{d_S}$.
This is however partially compensated by the fact that the basis $(\ket j)_{j=1}^{d_S}$ can be chosen freely and it makes sense to apply Theorem~\ref{thm:entanglementintheeigenbasis} to the optimal basis, i.e., the one that minimizes $\delta$.

Obviously, Theorem~\ref{thm:entanglementintheeigenbasis} can be further strengthened by maximizing $\delta$ only over those $k$ that jointly contain most of the probability weight of the initial state, i.e., whose populations $p_k$ already sum up to almost one.
This allows some of the $\delta_k$ to be large, as long as those of the significantly populated levels are small.

\subsubsection*{Discussion}
%
A statement complementing Observation~\ref{obs:absenceofthermalization} can be found in Ref.~\cite[Section B]{Linden09} (see also Ref.~\cite{MasterThesisHutter} for a generalization to mixed initial states and situations with initial correlations to reference system).
There it is shown that if the energy eigenstates of a non-degenerate Hamiltonian do contain a lot of entanglement, then subsystem initial state independence can be guaranteed.
The statement is however not exactly a converse statement as the measure of entanglement used there is quite different from our quantity $\delta_k$ in \texteqref{eq:geometricentanglementdelta}.

In a very similar spirit as above, absence of initial state independence has also been studied later in Ref.~\cite{PhysRevE.82.01}, which gives a condition that is necessary for subsystem initial state independence.
The article mostly studies a simplified version of this condition, which essentially demands that the reductions of most eigenvectors of the Hamiltonian must be sufficiently close to the maximally mixed state.

More recently, initial state independence was studied in Ref.~\cite{Hutter11,MasterThesisHutter}).
By using the \emph{decoupling method} \cite{Dupuis2010,1109.4348v1,Szehr2012} and the formalism of so-called \emph{smooth min and max entropies} \cite{Koenig08,Ciganovic2013}.
The authors show that it can be decided from just looking at one particular initial state whether a system satisfies initial state independence for most initial states.
Moreover, they give sufficient and necessary entropic conditions for initial state independence of most initial states.
The authors consider both subsystem initial state independence and bath initial state independence, i.e., the independence of the equilibrium state of the subsystem from the initial state of the bath.
The results concerning the absence of subsystem initial state independence of Ref.~\cite{Hutter11}, when compared to those of Ref.~\cite{PhysRevLett.10-6} discussed above, have the advantage that they apply to specific points in time instead of time averaged states and that the subsystem does not need to be small.
On the other hand they only hold for most/typical initial states.


\subsection{A numerical investigation of the violation of initial state independence}
\label{sec:anumericalinvestigationoftheviolationofinitialstateindependence}
%
In this section we will see that Theorems~\ref{thm:entanglementintheeigenbasis} and \ref{thm:absenceofthermalizaton} are physically meaningful and applicable to actual, reasonable physical many body Hamiltonians.
We will investigate a concrete, locally interacting system with strong interactions that is not integrable (more on that in Section~\ref{sec:integrability}) in any obvious way.
Still, we will provide strong numerical evidence that it fails to thermalize due to a lack of entanglement in its eigenbasis.
This section is partially based on material that was previously published in Ref.~\cite{PhysRevLett.10-6}.

Consider the spin-$1/2$ XYZ chain with $N$ sites, vertex set $\Vset = [N]$, and Hilbert space $\mcH = (\C^2)^{\otimes N}$ with random coupling and on-site field, whose Hamiltonian is given by $\H \coloneqq \H_0 + \H_1$ with
\begin{align}
  \H_0 &\coloneqq \sum_{x\in\Vset} h_x\,\sigma^Z_x \\
  \H_1 &\coloneqq \sum_{x \in \Vset} \vec{b}_x \argdot \vec{\sigma}^{\mathrm{NN}}_x ,
\end{align}
where $\vec{\sigma}^{\mathrm{NN}}_x = (\sigma^X_x\,\sigma^X_{x+1},\sigma^Y_x\,\sigma^Y_{x+1},\sigma^Z_x\,\sigma^Z_{x+1})^T$ and $\sigma^X_x,\sigma^Y_x,\sigma^Z_x$ are the Pauli matrices acting on site $x$.
The parameters $h_x \in \R$ and the components of $\vec{b}_x \in \R^3$ are chosen independently from normal distributions with zero mean and standard deviations $\sigma_0=1$ and $\sigma_1=0.4$, respectively.
It turns out that with unit probability the Hamiltonian $\H$ is non-degenerate and has non-degenerate energy gaps.
Related models have been extensively studied for example in Refs.~\cite{PhysRevB.82.17,1006.1634v1} in the context of many body localization.

Consider the bipartition $\Vset = S \dunion B$ with $S = \{1\}$ and $B = \{2,\dots,N\}$.
We are interested in the equilibration behavior under $\H$ when the system is started in energy eigenstates of $\H_0$.
To that end, a numerical exact diagonalization of $\H$ is carried out, which yields the eigenvalues and eigenvectors of $\H$ essentially up to machine precision.
This is done for different realizations of the random parameters $h_x$ and $\vec{b}_x$ and different system sizes $N$, which allows to compute the various quantities shown in Fig.~\ref{fig:absenceofthermalizationnumerics}.

Obviously, the energy eigenstates of $\H_0$ are products of the normalized eigenstates of the Pauli $Z$ operator $\sigma^Z$.
Consequently, for each energy eigenstate $\ket{E_k^0}$ of $\H_0$ the state vector $\sigma^X_1 \ket{E_k^0}$ is also an eigenstate of $\H_0$ and differs from $\ket{E_k^0}$ only on the first site, i.e., the subsystem $S = \{1\}$.
As can be seen in Fig.~\refsub{fig:absenceofthermalizationnumerics}{a}, the trace distance $\tracedistance{\omega^{S(1)}}{\omega^{S(2)}}$ of the reductions to $S$ of the two time averaged states $\omega^{(1)},\omega^{(2)}$ that result from starting the system in either $\ket{E_k^0}$ or $\sigma^X_1 \ket{E_k^0}$ is non-vanishing and almost independent of the system size.
At the same time, the effective dimension $\deff(\omega)$ (see \texteqref{eq:effectivedimension}) grows rapidly with increasing $N$, as can be seen in Fig.~\refsub{fig:absenceofthermalizationnumerics}{b}.
The \emph{equilibration coefficient} $\eqcoef \coloneqq d_S\,\sqrt{1/\deff(\omega)}/2$, which is the upper bound on the average trace distance to the equilibrium state implied by Theorem~\ref{thm:equilibrationonaverage} in the limit $T\to\infty$, is shown in the inset in Fig.~\refsub{fig:absenceofthermalizationnumerics}{b}.
This already shows that the subsystem equilibrates on average but that subsystem initial state independence is violated.

If, instead of averaging over the $\ket{E^0_k}$, the optimal pair of state vectors $\ket{E^0_k}$ and $\sigma^X_1\,\ket{E^0_k}$ is picked, then the time evolution is such that the states can even be distinguished for most times during the evolution (see Figs.~\refsub{fig:absenceofthermalizationnumerics}{c} and \refsub{fig:absenceofthermalizationnumerics}{d}).

\begin{figure}[p]
  \centering
  \includegraphics[width=\linewidth]{nonintegrable}
  \caption{(reproduced from \cite{PhysRevLett.10-6})
    For each of the product eigenvectors $\ket{E^0_k}$ of $\H_0$ the equilibration properties under the dynamics of $\H$ with $\sigma_0 = 1$ and $\sigma_1 = 0.4$ of the subsystem $S=\{1\}$ for the initial state $\ket{E^0_k}$ are compared with those of $\sigma^X_1 \ket{E^0_k}$, i.e., the same state but with the first spin flipped.
    Panels (a) and (b) display averages over energy eigenstates:
    (a)~Average $\delta_k$ for the energy eigenstates of $\H_0$ and average distance of the reduced dephased states $\mathbb{E}(\tracedistance{\omega^{S(1)}}{\omega^{S(2)}})$.
    (b)~Average effective dimension and equilibration coefficient.
    Panels (c) and (d) show quantities optimized over energy eigenstates:
    (c)~Maximum distinguishability $\max_k \Delta(\ket{E_k^{(0)}})$ where $\Delta(\ket{E_k^{(0)}}) = \tracedistance{\omega^{S(1)}}{\omega^{S(2)}} - \eqcoef(\psi^{(1)}_0) - \eqcoef(\psi^{(2)}_0)$ ($\Delta>0$ ensures distinguishability for most times. See the inset for an illustration of the meaning of the quantities.
    (d)~Effective dimension and equilibration coefficient of the state maximizing $\Delta(\ket{E_k^{(0)}})$.
    All quantities have been averaged over 100 realizations of the random parameters $h_x$ and $\vec{b}_x$.
    The error bars represent the standard deviation.
  }
  \label{fig:absenceofthermalizationnumerics}
\end{figure}

\begin{figure}[bt]
  \centering
  \begin{subfigure}{0.45\textwidth}
    \includegraphics[width=\linewidth]{nothermalization}
    \caption{}
    \label{fig:absenceofthermalizationnumerics2a}
  \end{subfigure}
  \begin{subfigure}{0.45\textwidth}
    \includegraphics[width=\linewidth]{nonrectangularstate}
    \caption{}
    \label{fig:absenceofthermalizationnumerics2b}
  \end{subfigure}
  \caption{For a random realization of $\H$ with $\sigma_0 = 1$, $\sigma_1 = 0.4$ on $N=10$ sites an eigenvector $\ket{E^0_j}$ of $\H_0$ from the center of the spectrum of $\H_0$ with the property that $\bra{E^0_j} \sigma^Z_1 \ket{E_j^0} = 1$ is picked.
    Panel (a) shows that the expectation value of $\sigma^Z_1$ in the time evolution under $\H$ starting in the initial states $\ket{E^0_j}$ (upper line) and $\sigma^X_1\,\ket{E^0_j}$ (lower line) equilibrates quickly and stays close to an equilibrium value for most times during the simulated evolution, but memory of the initial conditions prevents thermalization.
    Panel (b) shows that the squared moduli of the overlaps of the picked state with the energy eigenstates $\ket{E_k}$ of $\H$ are highly correlated with the expectation value $\bra{E_k}\sigma^Z_1\ket{E_k}$ of $\sigma^Z_1$ and fluctuate widely between neighboring energy eigenstates.
  }
  \label{fig:absenceofthermalizationnumerics2}
\end{figure}

There is strong evidence that it is precisely the mechanism described in Observation~\ref{obs:absenceofthermalization} that is responsible for this.
In Fig.~\refsub{fig:absenceofthermalizationnumerics}{a} the average of the quantity $\delta_k$, defined in \texteqref{eq:geometricentanglementdelta}, which enters the right hand side of the upper bound on the effective entanglement in the eigenbasis in Theorem~\ref{thm:entanglementintheeigenbasis}, is plotted.
As can be seen from the plot, it is low enough such that the bound in \texteqref{eq:averageeffectiveentanglementintheeigenbasisbound} is nontrivial and appears to be largely independent of the system size $N$.
A lack of effective entanglement in the eigenbasis is the cause for the observed violation of initial state independence.

As can be seen in Fig.~\refsub{fig:absenceofthermalizationnumerics}{c}, the described effect gets more pronounced for larger system sizes, providing evidence that the observed absence of thermalization effect is not just a finite size effect.

By looking more closely at one typical energy eigenstate $\ket{E^0_j}$ for one random realization of $\H$ for $N=10$ spins we further elucidate the effect.
Remember that the eigenstates of $\H_0$ are product states.
We pick a normalized eigenstate $\ket{E_j^0}$ with the property $\bra{E^0_j} \sigma^Z_1 \ket{E_j^0} = 1$ close to the center of the spectrum of $\H_0$.
It has a significant overlap with a reasonably large number of energy eigenstates of $\H$, but the overlaps vary significantly between eigenstates that are close in energy (see the inset of Fig.~\ref{fig:absenceofthermalizationnumerics2b}).
In other words: The state has a narrow energy distribution, but is highly ``non-rectangular'' (see also Section~\ref{sec:thermalizationunderassumptionsontheinitialstate}).

Due to the lack of entanglement in the eigenbasis the squared moduli of the overlaps $p_k = |\braket{E^0_j}{E_k}|$, i.e., the energy level populations, are highly correlated with the expectation value $\bra{E_k}\sigma^Z_1\ket{E_k}$ of $\sigma^Z_1$ in the corresponding energy eigenstate of $\H$ (Fig.~\ref{fig:absenceofthermalizationnumerics2b}).
In addition the expectation values of $\sigma^Z_1$ in eigenstates with neighboring energies fluctuate significantly.
One could say that the system does not fulfill the eigenstate thermalization hypothesis (see Section~\ref{sec:thermalizationunderassumptionsontheeigenstates}), as is expected for systems with \emph{quenched disorder} \cite{PhysRevB.82.17}.

As can be seen in Fig.~\ref{fig:absenceofthermalizationnumerics2a}, the expectation value of $\sigma^Z_1$ does equilibrate under the dynamics of $\H$ when the system is started in $\ket{E^0_j}$ or $\sigma^X_1\,\ket{E^0_j}$, but it remains close to its initial state for most times during the simulated evolution.
Memory of the initial condition prevents thermalization.

In addition to the numerical study of the absence of thermalization that was published in \cite{PhysRevLett.10-6} that we discussed above, there exist several other articles, including Refs.~\cite{1011.0781v1,Cazalilla11,Biroli09,Znidaric09}, that numerically and analytically study related effects.
Ref.~\cite{Biroli09} finds that the existence of few energy eigenstates that violate the \emph{eigenstate thermalization hypothesis} (see also Section~\ref{sec:thermalizationunderassumptionsontheeigenstates} and in particular Definition~\ref{def:eth}) can lead to absence of thermalization.
Ref.~\cite{Znidaric09} goes beyond the closed system setting and considers thermalization and its absence in systems that are coupled to thermal baths and finds that certain integrable models do not thermalize.
Ref.~\cite{1011.0781v1} studies quenches in a homogeneous XY quantum spin chain with transverse field starting in ground, excited, and thermal states.
The authors find that after certain quenches local observables fail to thermalize and relate this behavior to criticality.
Ref.~\cite{Cazalilla11} investigates equilibration and thermalization in exactly solvable models and finds that in such models correlation functions can retain memory of the initial conditions.

\vspace{1em} %hack!!!
\subsubsection*{Discussion}
%
In Ref.~\cite{PhysRevLett.10-6} it is rather briskly claimed that the XYZ chain studied above is \emph{non-integrable} according to various definitions of quantum (non-)integrability that exist in the literature (more on that in Section~\ref{sec:integrability}) and is advertised as an example of a system that does not thermalize despite being non-integrable.

While I do believe that the model discussed above is a good model for studies of the problematic aspects of the proposed definitions of quantum (non-)integrability it has at least one unwanted feature, namely that it is \emph{disordered}.
While the very recent results of Ref.~\cite{Mueller2013}, in particular Theorem~4 which proves of a weak version of the \emph{eigenstate thermalization hypothesis} (see Section~\ref{sec:thermalizationunderassumptionsontheeigenstates}) in translation invariant systems, make it seem less likely that a similar absence of thermalization effect due to a lack of entanglement in the eigenbasis can occur in translation invariant models, this issue is still far from settled.

Some authors associate quantum non-integrability with \emph{quantum chaos} (for more details see Section~\ref{sec:integrability}).
This aspect of the debate on (non-)integrability was almost entirely neglected in Ref.~\cite{PhysRevLett.10-6}, mainly due to the two following reasons:
First, not even in classical mechanics is chaos a necessary prerequisite for non-integrability and, second, there is no universally accepted definition of \emph{quantum chaos}.

Often, properties of the spectrum of a Hamiltonian, in particular the \emph{level spacing distribution}, are taken to be signatures of quantum chaos \cite{Haake10}.
A follow up numerical study of the spectral properties of the model unfortunately remained inconclusive.
For the small system sizes that can be studied numerically with todays computers, finite size effects make the energy eigenvalues neither clearly Poisson nor Wigner-Dyson distributed.
The outcome of the analysis turned out to be extremely dependent on details of the used unfolding technique.
The question of whether the disordered XYZ chain is a quantum chaotic system remains open.


\section{Integrability}
\label{sec:integrability}
%
In this section we discuss a concept that has recently started playing an important role in the debate on equilibration and thermalization in closed quantum systems --- the concept of \emph{integrability}.
It is often suggested or claimed that non-integrable systems thermalize, while integrable ones do not.
This wisdom has become folklore knowledge that is often invoked in discussion and talks on the topic (compare also Refs.~\cite{Rigol08,Rigol09,Rigol11,Biroli09,Znidaric09,1103.0787v1,1102.0528v1,Polkovnikov11,1108.0928v1,Neuenhahn10,Larson13,1201.0186v1,Beugeling2013,1112.3424v1.pd,Beugeling2013}).

In the following, we will review the current state of affairs concerning the usage of the term \emph{(quantum) integrability} in the context of equilibration and thermalization in closed quantum systems, comment on the concept of integrability and investigate to which extend the circumstantial evidence concerning the connection between \mbox{\mbox{(non-)}}integrability and thermalization can be substantiated.

To that end, in Section~\ref{sec:integrabilityinclassicalmechanics}, we will first recapitulate the definition of integrability in classical mechanics.
In Section~\ref{sec:integrabilityinquantumtheory} we will then discuss obstacles for a generalization of the concept of integrability to the quantum setting, collect and review existing notions of integrability, and critically assess them.
This assessment is largely based on the previous works Refs.~\cite{Weigert1992,1012.3587v1,PhysRevLett.10-6}.
We finish with some speculations on the connection of quantum \mbox{(non-)}integrability and computational complexity.


\subsection{In classical mechanics}
\label{sec:integrabilityinclassicalmechanics}
%
In classical mechanics \cite{Arnold78} \emph{(Liouville) integrability} is a well-defined concept.
In order to state and explain it we need to introduce some terminology first.

Consider a classical system with $n \in \Z^+$ degrees of freedom, each associated with a \emph{coordinate} $q_k$ and a corresponding \emph{momentum} $p_k$.
Then, in the Hamiltonian formalism, the $2\,n$ \emph{canonical coordinates} $(q_k)_{k=1}^n$ and $(p_k)_{k=1}^n$ span the \emph{phase space} $\Cst $ of the system \cite{Kinchin1949}.
We assume that the \emph{Hamiltonian function} $\CH\oftype\Cst\to\R$, i.e., the energy functional, of the system is time independent.
It then governs the time evolution of the system via \emph{Hamilton's equations} \cite{Arnold78}:
\begin{align}
  &\forall k\in[n]\itholds &\dot p_k &= - \frac{\del\CH}{\del q_k} &  \dot q_k &= \frac{\del\CH}{\del p_k}
\end{align}
The dot indicates the derivative with respect to time of the corresponding quantity, i.e., $\dot q_k$ is the temporal change of $q_k$.
Integrating these differential equations yields the \emph{phase flow} $g_\CH^t\oftype\Cst \to \Cst$, which maps the initial phase space vector of a system at time $0$ to that at time $t \in R$.
Define for any two functions $F,G\oftype\Cst\to\R$ their \emph{Poisson bracket} $(F,G)\oftype\Cst\to\R$ as
\begin{equation}
  (F,G) \coloneqq \lim_{t\to0} \frac{\ddel}{\ddel t}\, F \circ g_G^t .
\end{equation}
It turns out that $(\argdot,\argdot)$ is bilinear and skew-symmetric \cite{Arnold78}.
A function $F\oftype\Cst\to\R$ is called a \emph{first integral of motion} under the evolution induced by $\CH$ if $(F,\CH) = 0$.
More generally, if for $F,G\oftype\Cst\to\R$ it holds that $(F,G) = 0$, then $F$ and $G$ are said to be \emph{in involution}.

We can now define Liouville integrability:
\begin{definition}[Liouville integrability \cite{Arnold78}]
  A classical system with $n$ degrees of freedom is called (Liouville) integrable if it entails a sequence $(F_k)_{k=1}^{n}$ of $n$ independent first integrals of motion that are pairwise in involution.
\end{definition}
\emph{Liouville's theorem for integrable systems} shows that Liouville integrable systems can be solved, i.e., the time evolution can be explicitly calculated, in a systematic way by \emph{quadratures}, i.e., by direct integration of differential equations:
\begin{theorem}[Corollary of Liouville's theorem for integrable systems \cite{Arnold78}]
  If a system is Liouville integrable, its time evolution can be solved by quadratures.
\end{theorem}
In more detail: Liouville's theorem for integrable systems essentially ensures that, given the initial values of all canonical coordinates, the time evolution of an integrable system is confined to a \emph{smooth} \emph{submanifold} of the phase space that is \emph{diffeomorphic} to an $n$-dimensional \emph{torus}.
The time evolution is quasi periodic and can be described in terms of the so-called \emph{action angle coordinates} $(\varphi_k)_{k=1}^n$ that parametrize the torus.

The action angle coordinates can be explicitly constructed from the sequence $(F_k)_{k=1}^{n}$ of $n$ independent first integrals of motion and the values fixed for them.
Fixing different values for the $n$ first integrals of motion results in different tori.
In the coordinate system of the action angle variables the equations of motion are given by $2\,n$ simple ordinary differential equations of the form $\dot F_k = 0$ and $\dot \varphi_k = w_k$, with $w_k \in \R$ being constants that depend on the values that were fixed for the $n$ first integrals of motion.
For more details see for example \cite[Section 49]{Arnold78}.

If a Liouville integrable system is perturbed, i.e., the Hamiltonian function slightly changed, then the time evolution is generally not confined to a torus anymore and cannot be derived in a systematic way.
For small perturbations the Kolmogorov-Arnold-Moser (KAM) theorem ensures, under a so-called \emph{non-resonance condition}, that most tori are only deformed and the time evolution on them is then still quasi periodic \cite{Moradi2001,Tabor1989,Poeschel03}.

In summary we have:
Integrability in classical systems implies \emph{systematic solvability} and thereby yields a \emph{qualitative classification} of classical systems.
Liouville integrable systems are not \emph{ergodic} (see Section~\ref{sec:canonicalapproaches}) in the sense that their phase space trajectory does not explore the whole phase space, but is confined to a portion of it.
Whether or not this implies that integrable systems cannot thermalize depends on the definition of thermalization, but the motion of the system is quasi periodic and hence no convergence of the state of the system in the limit $t\to\infty$ is possible.
Non-integrability in classical systems is \emph{not} sufficient for ergodicity or chaos and hence also not sufficient for notions of mixing or thermalization based on these concepts.
Still, the concept of Liouville integrability yields a classification of systems with strong implications for their physical behavior.


\subsection{In quantum mechanics}
\label{sec:integrabilityinquantumtheory}
%
We now turn to \emph{integrability} in quantum mechanics.
Ideally, a notion of quantum integrability should yield a classification that divides quantum systems into two classes, \emph{integrable} ones and \emph{non-integrable} ones, with markedly different physical properties.
In addition it should in some sense be a generalization of Liouville integrability.
However, if one tries to generalize the concept of Liouville integrability to quantum systems in a straight forward manner, one immediately encounters problems:

Consider a quantum system with $d$ dimensional Hilbert space $\mcH$ and Hamiltonian $\H \in \Obs(\mcH)$.
An orthonormal eigenbasis $(\ket{\tilde E_k})_{k=1}^d$ of $\H$, with corresponding eigenvalues $(\tilde E_k)_{k=1}^d$, can always be constructed in a systematic way by diagonalizing the Hamiltonian.
The time evolution of an arbitrary initial state vector $\ket \psi \in \mcH$ is then given by
\begin{equation}\label{eq:timeevolutioninexpliciteform}
  t \mapsto \ket{\psi(t)} \coloneqq \sum_{k=1}^d |\braket{\tilde E_k}{\psi}|\,\e^{\i\,\tilde\varphi_k(t)}\,\ket{\tilde E_k} ,
\end{equation}
with $\tilde\varphi_k(t) \coloneqq \arg(\braket{\tilde E_k}{\psi})-\,\tilde E_k\,t$.
The overlaps $\braket{\tilde E_k}{\psi}$ can also be calculated systematically, so the time evolution of a (finite dimensional) quantum system can always be obtained in a systematic way for any Hamiltonian and any initial state.

The analogy to the situation of Liouville integrable systems is striking:
The dimension $d$ plays the role of the number $n$ of degrees of freedom of the system in the classical case.
The linear functionals $|\bra{\tilde E_k} \,\argdot\, | \oftype \mcH \to \R$, induced by the eigenvectors of $\H$, are analogous to the first integrals of motion in Liouville's theorem on integrable systems, and the time independent moduli of the overlaps $|\braket{\tilde E_k}{\psi}| = |\braket{\tilde E_k}{\psi(t)}|$ play the role of the values fixed for these constants of motion.
Finally, the functions $\tilde \varphi_k$ in the right hand side of \texteqref{eq:timeevolutioninexpliciteform} satisfy differential equations analogous to those of the action angle variables, namely $\dot{\tilde{\varphi}}_k = \tilde E_k$, and the time evolution indeed happens on a $d$-torus.
As in the classical case, the specific torus to which the evolution is confined depends on the values fixed for the conserved quantities.

Are all (finite dimensional) quantum systems \emph{integrable}?
The apparently simple form of the quantum mechanical equation of motion lead Schrödinger to the following comment in his 1927 article on the ``Energy exchange according to wave mechanics''\footnote{Original in German \cite{Schroedinger1927}: \foreignlanguage{ngerman}{``Energieaustausch nach der Wellenmechanik''}}, in which he tries to derive thermodynamic behavior from the unitary time evolution of quantum mechanics (see also Section~\ref{sec:typicality}):
``It seems that one has to abstain from an attempt to use some form of quasi-ergodic hypothesis to proclaim [ensemble averages] as the correct time averages. The equations [of motion] are far too transparent to admit such a hypothesis (they posses at least [dimension many] independent holomorphic integrals, namely the squared moduli of the overlaps with the eigenstates''%
\footnote{Original in German \cite{Schroedinger1927}: \foreignlanguage{ngerman}{``Auf den Versuch, durch irgendetwas der Quasiergodenhypothese Analoges diese Mittelwerte als richtige Zeitmittel hinzustellen, muß man wohl verzichten. Die Gleichungen (9) sind vie1 zu durchsichtig, um sich eine derartige Hypothese gefallen zu lassen (sie besitzen mindestens $\alpha$ unabhängige holomorphe Integrale, nämlich die Amplitudenquadrate der \glqq{}Normalschwingungen\grqq{})''.}}
It seems that the dynamics of quantum systems is far less rich than that of classical systems and that quantum systems simply cannot be non-integrable or ergodic.

Of course, simply classifying all (finite dimensional) quantum systems as integrable cannot be the solution.
After all, as we have seen in the previous sections, under certain conditions closed quantum systems can exhibit behavior that is reminiscent of the behavior of non-integrable classical systems, such as equilibration and thermalization.
Moreover, if one believes that quantum mechanics is indeed a fundamental theory, then it must also be able to somehow produce the non-integrable, and sometimes even ergodic or chaotic, behavior observable on the classical level.
It is thus tempting to define quantum versions of the notion of chaos, ergodicity and integrability via a classical limit.
As we will see shortly this is only one of the many approaches that have been pursued in the literature.

Before going on, it is reasonable to give a set of conditions that a good notion of \mbox{(non-)}integrability for quantum systems should satisfy.
Inspired by the work of \textcite{1012.3587v1}, we demand that a definition of quantum integrability should:
\begin{enumerate}[label={(Condition~\arabic*)},ref={\arabic*},leftmargin=*]
\item \label{item:reasonablenotionofquantumintegrabilitycondition4} have implications for the physical behavior.
\item \label{item:reasonablenotionofquantumintegrabilitycondition1} be applicable to a large class of quantum systems.
\item \label{item:reasonablenotionofquantumintegrabilitycondition2} be unambiguous.
\item \label{item:reasonablenotionofquantumintegrabilitycondition3} be decidable for concrete models.
\end{enumerate}

Unfortunately none of the existing frequently used notions of quantum integrability seems to fulfill all these criteria.
The following is a (probably incomplete) list of the different definitions of quantum integrability that have been introduced, together with some (purely exemplary) references in which the corresponding definition appears or is used (see also Refs.~\cite{1012.3587v1,PhysRevLett.10-6,sutherland04,Weigert1992}).
A system is \emph{quantum integrable}:
\begin{enumerate}[leftmargin=*]
\item \label{item:npotionsofintegrability_conservedquantity} If it exhibits $n$ physically meaningful conserved mutually commuting independent operators \cite{Rigol07,Braak11,1109.5904v1,Barthel08,Hawkins2008,Jensen1985} (see also Ref.~\cite{Weigert1992} and the references therein).
\item \label{item:npotionsofintegrability_betheansatz} If it is integrable by the Bethe ansatz \cite{sutherland04,Ikeda2013a,Beugeling2013}.
\item \label{item:npotionsofintegrability_nondiffractive} If it exhibits nondiffractive scattering \cite{sutherland04}.
\item \label{item:npotionsofintegrability_classicallimit} If it has a classical limit that is integrable \cite{Castagnino2006}.
\item \label{item:npotionsofintegrability_levelstatistics} If its level statistics follows a Poisson law and is non-integrable if it is of Wigner-Dyson type \cite{Casati1985,1103.0787v1,PhysRevB.82.17,Znidari2013,1111.3375v1,Atas12,Tabor1989,Bohigas1984,Fine2013,Jensen1985}.
\item \label{item:npotionsofintegrability_nolevelrepulsion} If it does not exhibit level repulsion \cite{Stepanov2008,Berry1977a}.
\item \label{item:npotionsofintegrability_quantumnumbers} If (many of) its eigenfunctions can be labeled in a certain way with quantum numbers \cite{Braak11,Berry1977a}.
\item \label{item:npotionsofintegrability_exaxclysolvable} If it is exactly solvable in any way \cite{Beugeling2013,Fendley1995,Braak11,Jensen1985}.
\end{enumerate}
In the first definition \emph{physically meaningful} can have very different meanings.
It can, for example, in the case of composite systems, refer to local operators.
The number $n$ is usually taken to be equal to the number of degrees of freedom of the model or the number of constituents in the case of composite systems.
Similarly, \emph{independent} can have several meanings, \emph{linearly independent} and \emph{algebraically independent} being popular choices.
Usually all non-interacting composite systems and systems such as the Hydrogen atom fall in this category.
Models that are integrable according to this definition are often also integrable according to one of the other definitions given above (especially Definitions~\ref{item:npotionsofintegrability_betheansatz}, \ref{item:npotionsofintegrability_quantumnumbers}, and \ref{item:npotionsofintegrability_exaxclysolvable}).
Many of the definitions of integrability of this type suffer from the severe problem that if the definition is taken seriously, all quantum systems classify as integrable and hence it violates Condition~\ref{item:reasonablenotionofquantumintegrabilitycondition4} (see the discussion above and Ref.~\cite{Weigert1992} for a critic of such notions of integrability).

Definitions~\ref{item:npotionsofintegrability_betheansatz}, \ref{item:npotionsofintegrability_nondiffractive}, and \ref{item:npotionsofintegrability_classicallimit} are only applicable to restricted classes of models and hence violate Condition~\ref{item:reasonablenotionofquantumintegrabilitycondition1} in the above list.
The same holds, although arguably in a weaker sense, for Definition~\ref{item:npotionsofintegrability_nolevelrepulsion}, which is only applicable to systems which have a natural tuning parameter.

Definitions~\ref{item:npotionsofintegrability_levelstatistics} and \ref{item:npotionsofintegrability_nolevelrepulsion} suffer from the problem that also certain models that are usually regarded as integrable can have spectra that would classify them as non-integrable \cite{Benet2003,Berry1977a}.
In fact, it is trivial to construct such examples.
In a composite systems of, say, spin-1/2 systems, one can simply take a Hamiltonian that is diagonal in the usual Pauli-$Z$ product basis and which hence should simply be integrable and set its spectrum to be that of some non-integrable model.
Moreover, natural tunable models are known that exhibit thermodynamic behavior in both the regime that would be classified as integrable and the one that would be classified as non-integrable according to this definition \cite{Jensen1985}.
Hence, these definitions violate Condition~\ref{item:reasonablenotionofquantumintegrabilitycondition2} and \ref{item:reasonablenotionofquantumintegrabilitycondition4}.

Especially Definitions~\ref{item:npotionsofintegrability_conservedquantity} and \ref{item:npotionsofintegrability_exaxclysolvable} suffer from the problem that it might simply be a lack of imagination that prevents one from solving a given model and thus violate Condition~\ref{item:reasonablenotionofquantumintegrabilitycondition3}.
This is well illustrated by the recent (partial) solution of the Rabi model, which was long thought to be non-integrable (see Ref.~\cite{Braak11} and the references therein).


\subsection*{Discussion}
%
In conclusion, it seems fair to say that the situation is complicated and the problems do not end here.

What about the commonly invoked concept of \emph{(quantum) non-integrability}?
Should any system that is not integrable according to any of the above definitions be called non-integrable?
It has been demonstrated that various notions of \emph{quantum non-integrability} do not imply thermalization \cite{Larson13,PhysRevLett.10-6} (see also the discussion section of Section~\ref{sec:absenceofthermalization}).
So, should a definition of quantum non-integrability be sufficient for thermalization and if so in which sense?

What about the thermodynamic limit?
In this limit certain questions concerning the behavior of quantum lattice systems can even become \emph{algorithmically undecidable} \cite{Cubitt2011}.
This leaves open the possibility that finding a criterion for quantum \mbox{(non-)}integrability that fulfills Condition~\ref{item:reasonablenotionofquantumintegrabilitycondition2} and \ref{item:reasonablenotionofquantumintegrabilitycondition3} even in the thermodynamic limit could be more difficult than one might naively expect.

Surely, general claims that ``non-integrable quantum systems thermalize'' seem unjustified at present.
A commonly accepted definition of quantum integrability does not exist.
Time will tell whether the thoughtful, but rather complicated proposal made in Ref.~\cite{1012.3587v1} can fill this gap in our understanding of quantum many body systems (an assessment of this proposal is beyond the scope of this thesis).

Alternatively one could try to construct a notion of quantum non-integrability based on the \emph{computational complexity} of solving a given system.
This would make the question of quantum integrability a qualitative one rather than a quantitative one.
In the spirit of Ref.~\cite{1108.0374}, one could for example try to measure the complexity of a Hamiltonian by the (minimal) circuit length of the diagonalizing unitary (see also Section~\ref{sec:timescales} for a discussion of Ref.~\cite{1108.0374}).

After all, the dimension $d$ of the Hilbert space of a system is not really analogous to the number of degrees of freedom of a classical system.
It generally grows exponentially with the number of constituents, and so does the computational complexity of naive approaches to diagonalize the Hamiltonian, or find the ground state, or simulate the time evolution.
At least the latter is reminiscent of the problem encountered when one tries to simulate classical chaotic systems.
For some quantum systems, for example if some conserved quantities are known or if the Hamiltonian has a special structure, the computational complexity of solving it can be drastically reduced \cite{Anders2003,Braunstein2005,Adesso2007,White1992,Schollwock2005,Hallberg2006}.
These systems could then maybe be classified as integrable.


\section{Decay of correlations and stability of thermal states}
\label{sec:propertiesofthermalstatesofcompositesystems}
%
The approach towards statistical mechanics taken in this work is genuinely quantum.
Hence, it is natural to ask whether the thermodynamic concepts that we encountered in the previous sections still make sense on very small scales.
Particularly interesting in this respect is the concept of temperature.
Thermometry of extremely small systems is now experimentally feasible \cite{GaoBan02,PotGueBir97,PengPeng13}.
What is the meaning of temperature on the nanoscale?
In which sense is temperature really \emph{intensive}?
This is often claimed in thermodynamics, but far from obvious, at least in the context of quantum mechanics.
We shall call this the \emph{locality of temperature problem}.

This problem has previously been addressed in Refs.~\cite{Hartmann2003,Hartmann2004,Hartmann2004a,0908.3157v3}, and more recently extensively studied in Ref.~\cite{Kliesch2013a}.
The recent results build upon and follow a tradition of previous, more mathematically inclined works on clustering of correlations in classical systems \cite{Rue99_stat_mech_book}, translational invariant Hamiltonians of continuum systems \cite{Gin65}, and Hamiltonians on cubic lattices \cite{Greenberg1969,Greenberg1969a,Park1995} (see also the book by \textcite{BratteliRobinson2}).

In Ref.~\cite{Kliesch2013a} three theorems are proven:
A \emph{truncation formula}, which allows to express the influence of sets of local Hamiltonian terms on the expectation value of an observable in the thermal state of a locally interacting quantum system in terms of a correlation measure.
A \emph{clustering of correlations result}, which shows that above a universal critical temperature this correlation measure exhibits an exponential decay.
And finally, a result that ensures \emph{local stability} of thermal states above a universal critical temperature and thereby partially solves the locality of temperature problem.

To begin with, we introduce a quantity that measures correlations.
We have previously touched upon the issue of quantifying correlations in Section~\ref{sec:correlationsandmomemts}.
Here we measure correlations by a generalization of the covariance that we introduced in \texteqref{eq:covariance}.
We define for any $\tau \in [0,1]$, any two operators $A,B \in \Bop(\mcH)$, and any quantum state $\rho \in \Qst(\mcH)$ the \emph{generalized covariance}
\begin{equation} \label{eq:generalizedcovariance}
 \cov_\rho^\tau(A,B)
 \coloneqq \Tr(\rho^\tau A\, \rho^{1-\tau} B) - \Tr(\rho\, A) \Tr(\rho \, B) \, .
\end{equation}
The choice $\tau = 1$ gives the usual covariance introduced in Section~\ref{sec:correlationsandmomemts}.
The reason for our more general definition is that the \emph{generalized covariance} naturally appears in the \emph{truncation formula}, which is the first theorem of this section:
\begin{theorem}[Truncation formula {\cite[Theorem~1]{Kliesch2013a}}] \label{thm:truncationformula}
  Consider a spin or fermionic system with Hilbert space $\mcH$ and let $\H \in \Obs(\mcH)$ be a locally interacting Hamiltonian with edge set $\Eset$.
  Let $I \subset \Eset$ and define for $s \in [0,1]$ the interpolating Hamiltonian $\H(s) \coloneqq \H - (1-s) \sum_{X \in I} \H_X$.
  Then, for any operator $A \in \Bop(\mcH)$,
  \begin{equation}\label{eq:truncation_error_in_terms_of_cov}
    \begin{split}
      &\Tr\bigl(A\,\rhog[\H(0)](\beta)\bigr)-\Tr\bigl(A\,\rhog[\H(1)](\beta)\bigr) \\
      = &\beta\,\int_0^1  \int_0^1 \cov_{\rhog[\H(s)](\beta)}^\tau(A, \sum_{X \in I} \H_X)\,\dd\tau\,\dd s .
    \end{split}
  \end{equation}
\end{theorem}
The left hand side of \texteqref{eq:truncation_error_in_terms_of_cov} is the difference between the expectation value of an arbitrary observable $A \in \Bop(\mcH)$ in the thermal states of $\H(0)$ and $\H(1)$ respectively.
Note that $\H(1) = \H$ and that $\H(0)$ contains all terms of $\H$ except those with support contained in $I$.
The truncation formula quantifies how the expectation value of $A$ changes when these terms are added or removed, hence the name, and tells us that this change can be expressed in terms of the generalized covariance.

It is important to note that \texteqref{eq:truncation_error_in_terms_of_cov} is an \emph{equality}.
The generalized covariance exactly captures the response of expectation values in the thermal state to local changes of the Hamiltonian, i.e., here the adding/removing of the terms corresponding to the edges in $I$.

Define for any subsystem $X \subset \Vset$ the set $X_\partial \subset \Eset$ of edges that overlap with both $X$ and its complement, i.e.,
\begin{equation}
  X_\partial \coloneqq \{ Y \in \Eset\oftype Y \cap X \neq \emptyset \land Y \nsubseteq X \} .
\end{equation}
Again, we extend this notation to operators $A \in \Bop(\mcH)$ and define
\begin{equation}
  A_\partial \coloneqq \{ Y \in \Eset\oftype Y \cap \supp(A) \neq \emptyset \land Y \nsubseteq \supp(A) \} .
\end{equation}
Consider a bipartite system with $\Vset = S \dunion B$.
If $I$ in Theorem~\ref{thm:truncationformula} is chosen to be $S_\partial$, then $\H(0) = \H_S + \H_B$ is the Hamiltonian without the terms that connect $S$ and $B$, i.e., the non-interacting Hamiltonian.
If $\supp(A) \subset S$, then the truncation formula tells us that the expectation value of $A$ in $\rhog[\H](\beta)$ is similar to that of $\trunc A S$ in $\rhog_S[\H](\beta) = \rhog[\trunc{\H_S} S](\beta)$ if and only if the right hand side of \texteqref{eq:truncation_error_in_terms_of_cov} is small.

In other words:
\begin{observation}[Locality of temperature \cite{Kliesch2013a}]
  Temperature can be defined locally on a given length scale if and only if the averaged generalized covariance is small compared to $1/\beta$ on that length scale.
\end{observation}

We now give conditions under which this can be guaranteed.
More precisely, we will formulate a theorem that ensures an exponential decay of the generalized covariance $\cov^\tau_{\rhog(\beta)}(A,B)$ with the \emph{graph distance} $\dist(A,B)$ (remember the definitions from Section~\ref{sec:localquantumsystems}) between the supports of the two operators $A,B \in \Bop(\mcH)$ above a universal critical temperature.
Together with the truncation formula this will allows us to prove the local stability result promised earlier.

The following theorem applies to all Hamiltonians whose interaction (hyper)graph has a finite \emph{growth constant}.
To explain what this means we need some additional notation.
A subset $F \subset \Eset$ of the edge set \emph{connects} $X$ and $Y$ if $F$ contains all elements of some sequence of pairwise overlapping edges such that the first overlaps with $X$ and the last overlaps with $Y$ and similarly for sites $x,y \in V$.
A subset $F \subset E$ of the edge set $\Eset$ that connects all pairs of its elements is called \emph{connected} and connected subsets $F$ are also called \emph{animals} \cite{Miranda2011,Penrose1994}.
The size $|F|$ of an animal $F$ is the number of edges it contains.
It turns out that for many interesting (hyper)graphs the number of animals of a given size that contain a given edge grows exponentially with the size, but not faster.
That is, they have a finite \emph{growth constant}.
More precisely, the growth constant of a (hyper)graph $\mcG = (\Vset,\Eset)$ is the smallest constant $\animalc$ satisfying
\begin{equation}\label{eq:animal_bound}
  \forall k\in \Z^+\itholds \sup_{X \in \Eset} |\{ F\subset\Eset \text{ connected}\oftype X \in F \land |F|=k \}| \leq \animalc^k .
\end{equation}
For example, the growth constant $\animalc$ of the interaction graph of nearest neighbor Hamiltonians on $D$ dimensional cubic lattices can be bounded by $2\,D\,\e$ (see Lemma~2 in Ref.~\cite{Miranda2011}).
Moreover, there is a finite growth constant $\animalc$ for any regular lattice \cite{Penrose1994}, and there exist upper bounds on the growth constants of so-called spread-out graphs \cite{Miranda2011} that make it possible to bound the growth constant of the interaction hypergraphs of all $l$-local $k$-body Hamiltonians on regular lattices \cite{Kliesch2013a}.
Where $l$-local $k$-body on a regular lattice means that $\Vset$ can be mapped onto the sites of a regular lattice such that $\Eset$ contains only subsystems which consist of at most $k$ sites that are all contained in a ball (measured in the graph distance of the regular lattice) of diameter $l$.
For details see \cite{Kliesch2013a}.
Apart from all $l$-local $k$-body Hamiltonians on regular lattices this also makes the following results indirectly applicable to systems with exponentially decaying interactions (such Hamiltonians can be exponentially well approximated by $l$-local $k$-body Hamiltonians) but not to Hamiltonians with algebraically decaying interactions, such as for example Coulomb or dipole interactions.

We can now state the clustering of correlations result:
\begin{theorem}[Clustering of correlations at high temperature {\cite[Theorem~3 and 16]{Kliesch2013a}}] \label{thm:clustering}
  Consider a locally interacting system of spins or fermions with Hilbert space $\mcH$ and Hamiltonian $\H \in \Obs(\mcH)$ with \emph{local interaction strength} $J \coloneqq \max_{X\in\Eset} \norm[\infty]{\H_X}$ and interaction (hyper)graph $\mcG = (\Vset,\Eset)$ with growth constant $\animalc$.
  Define the \emph{critical temperature}
  \begin{equation}\label{eq:crit_beta_def}
    \beta^\ast \coloneqq \ln((1+\sqrt{1+4/\animalc})/2)/(2\,J)
  \end{equation}
  and the \emph{thermal correlation length}
  \begin{equation} \label{eq:correlation_length_def}
    \xi(\beta) \coloneqq \left|1/\ln\left(\animalc\, \e^{2\,|\beta|\,J}(\e^{2\,|\beta|\,J}-1)\right)\right|  \, .
  \end{equation}
  Then, for every $|\beta|<\beta^\ast$, parameter $\tau \in [0,1]$, and every two operators $A,B \in \Bop(\mcH)$ with
  $\dist(A, B) \geq \xi(\beta)\, \left|\ln\left(\ln(3)\, (1-\e^{-1/\xi(\beta)})/\min(|A_\partial |,|B_\partial |)\right)\right|$ ,
  \begin{equation} \label{eq:clustering}
    |\cov^\tau_{\rhog(\beta)} (A, B)| \leq \frac{4\,a \norm{A}_\infty\, \norm{B}_\infty}{\ln(3)\, (1-\e^{-1/\xi(\beta)})}\,\e^{-\dist(A,B)/\xi(\beta)} .
  \end{equation}
\end{theorem}

The above theorem implies that in thermal states above the critical temperature the correlations between any two $A,B \in \Bop(\mcH)$ decay exponentially with their distance $\dist(A,B)$.
Importantly, the critical temperature \eqref{eq:crit_beta_def} is independent of global properties of $\H$ but only depends on the local interaction strength $J$ and the growth constant $\animalc$ of its interaction (hyper)graph.

In the context of this work, the most important implication of Theorem~\ref{thm:clustering} is the following result, which proves stability of thermal states above the critical temperature against local perturbations.
More precisely, it shows that changing the Hamiltonian of a locally interacting quantum system only outside of a subsystem $S$, i.e., only the terms that are not completely contained in $S$, has only limited influence on how thermal states to temperatures above the critical temperature look like in the interior $S^i \subset S$ of $S$ if the distance between $S^i$ and $S_\partial$ is large enough:

\begin{theorem}[Universal locality at high temperatures {\cite[Theorem~4 and 17]{Kliesch2013a}}]\label{thm:intensivity}
  Let $\H$ be a Hamiltonian satisfying the conditions of Theorem~\ref{thm:clustering},
  let $\beta^\ast$ and $\xi(\beta)$ be defined as in \texteqref{eq:crit_beta_def} and \texteqref{eq:correlation_length_def},
  let $|\beta|< \beta^\ast$, and let $S^i \subset S \subseteq V$ be subsystems with
  $\dist(S^i, S_\partial) \geq \xi(\beta)\, \left|\ln\left(\ln(3)\, (1-\e^{-1/\xi(\beta)})/|S^i_\partial|\right)\right|$.
  Then
  \begin{equation}\label{eq:stabilityofthermalstatesbound}
    \tracedistance{\rhog^{S^i}[\H](\beta)}{\rhog^{S^i}[H_S](\beta)} \leq \frac{ v\, |\beta|\, J }{1-\e^{-1/\xi(\beta)}}\,\e^{- \dist(S^i, S_\partial) /\xi(\beta)} ,
  \end{equation}
  where $v \coloneqq 2\, |S^i_\partial|\,|S_\partial|/\ln(3)$.
\end{theorem}

If the conditions of the above theorem are met and the interior subsystem $S^i$ is sufficiently far from the boundary $S_\partial$ of $S$ such that $\dist(S^i, S_\partial)$ is large and hence the right hand side of \texteqref{eq:stabilityofthermalstatesbound} small, then the reduced state $\rhog^{S^i}[\H](\beta)$ on $S^i$ of the thermal state of $\H$ is almost independent of the terms of the Hamiltonian $\H$ that are not in the restricted Hamiltonian $\H_S$.


\subsection*{Discussion}
%
Theorem~\ref{thm:intensivity} is not unexpected, but it is nevertheless remarkable that it can be shown in this generality for systems of both locally interacting spins and fermions.
Even more so, because, as we have seen in the discussion of equilibration (Section~\ref{sec:equilibration}) and especially in the section on equilibration time scales (Section~\ref{sec:timescales}), a major obstacle for improving the statements we were able to make is that it seems to be hard to use the structure of natural many body Hamiltonians, namely that interactions are usually few body and often short range.
Theorem~\ref{thm:intensivity} is an instance of a result whose proof heavily relies on the locality structure of locally interacting Hamiltonians and is able to exploit their structure.

It is interesting to plug in the numbers of a specific model to see how physical the derived critical temperature is.
As a concrete example consider the ferromagnetic two dimensional isotropic Ising Model without external field.
The critical temperature of Theorem~\ref{thm:clustering} and \ref{thm:intensivity} is
$1/(\beta^\ast\,J) = 2/\ln((1+\sqrt{1+1/\e})/2) \approx 24.58$, whereas the \emph{Curie temperature}, i.e., the temperature at which the phase transition between the paramagnetic and the ferromagnetic phase happens is known to be
$1/(\beta_c\,J) = 2/\ln(1+\sqrt{2}) \approx 2.27$ \cite{Bhattacharjee1995}.

The critical temperature below which the above theorems work is off by about one order of magnitude.
It is however a \emph{universal upper bound} independent of details of the particular model.
Given how difficult it is to calculate or even bound critical temperatures in lattice models (both classical and quantum) and that good bounds are known only for very few models the existence of such a non-trivial and universal upper bound is remarkable.

Besides being of fundamental interest, Theorem~\ref{thm:intensivity} has some obvious computational implications:
It implies that for all $|\beta| < \beta^\ast$ reduced states of thermal states can be approximated with a computational cost independent of the system size and polynomial in the reciprocal approximation error \cite{Kliesch2013a}.
The proof of Theorem~\ref{thm:clustering} is based on a \emph{cluster expansion} (see Lemma~6 in Ref.~\cite{Kliesch2013a}) previously used in Ref.~\cite{Hastings06} to show that thermal states above a critical temperature can be approximated by so-called \emph{matrix product operators} (MPOs).
The subtleties of this approximation are often misunderstood.
For details see the appendix of Ref.~\cite{Kliesch2013a}.

\cleardoublepage

\chapter{Conclusions}
\label{sec:conclusions}
%
We have seen that finite dimensional quantum systems in pure states evolving unitarily according to the Schrödinger equation exhibit a wealth of phenomena that can rightfully be called \emph{thermodynamic}.
Individual observables and whole subsystems have a tendency to evolve towards equilibrium values/states and then stay close to them for most times during the evolution or extended time intervals.
The equilibrium properties can be calculated with a maximum entropy principle implied by quantum mechanical dynamics alone.
A weak interaction with an environment naturally leads to decoherence in the energy eigenbasis and under additional conditions even equilibration to a thermal state, i.e., thermalization, can be guaranteed.

Entanglement and quantum mechanical uncertainty play key roles in these processes.
The immensely large dimension of the Hilbert space of composite quantum systems allows to justify methods of statistical physics with typicality arguments.
A quantum information inspired approach allows to relate the possibility of defining temperature in composite systems locally to the absence of certain long range correlation and provides tools to give conditions under which these correlations decay exponentially.
This helps to delineate the boundaries of the applicability of thermodynamic concepts and makes it possible to give universal bounds on critical temperatures.

Despite the coherent picture formed by the results discussed in this thesis many problems still await a full solution.
Among the most interesting open questions is the problem to give physically reasonable bounds on the time scales for equilibration and thermalization in concrete locally interacting quantum systems.
The notion of weak coupling used in the proof of thermalization is still not entirely satisfactory.
Both of these issues relate to the more general problem of finding ways to exploit the structure present in locally interacting quantum systems.
Advances towards a better understanding of the relation of equilibration, transport, disorder, and chaos are still hampered by a lack of a commonly accepted and reasonable definition of the concept of integrability in quantum systems.

It is the hope that the review provided in this exposition will help to stimulate further research on these topics.

\cleardoublepage

%%%% Bibliography%%%%%%%%%%%%%%%%%%%%%%%%%%%%%%%%%%%%%%%
%\bibliography{bibliography}
\emergencystretch 2.5em
\printbibliography
\emergencystretch 1em

\cleardoublepage

\appendix

%%%%Back matter%%%%%%%%%%%%%%%%%%%%%%%%%%%%%%%%%%%%%%%%%%%
\chapter{Back matter}

\section{Acknowledgements}

\begin{quotation}
  ``Mathematical discoveries, small or great, are never born of spontaneous generation. They always presuppose a soil seeded with preliminary knowledge and well prepared by labor, both conscious and subconscious.''
\end{quotation}
\begin{flushright}
  --- ascribed to Henri Poincar\'e
\end{flushright}

In a self-praising way I could claim that this thesis is the product of months of writing up of the knowledge acquired during three years of hard work.
In the light of the quotation above it is however certainly more just to call it a product of a livelong progress of learning from dedicated teachers, supported by my friends and family, and, in the end, only a small brick on the tower of ideas erected by generations of scientists before me.

There are so many people to whom I am indebted and so many whom I want to thank, but it is hard to weight the importance of the influence of individuals that it seems impossible to order them properly.
Thus, I decided to simply go with a roughly chronological order:

I would like to thank my parents for setting off the spark of curiosity in me, which is the drive behind an ongoing pursuit for knowledge that has been a source of great joy throughout my life.

I would like to thank all the close members of my family who have accompanied me on my way.
In particular Käthe, Hermann, my brother Simon, and Philip Broser who introduced me to programming.

I would like to thank the dedicated among my teachers. In particular, Martin Bolkart, who was a great teacher in elementary school, Mr.\ Loho, who admitted me into the Gymnasium despite my miserable performance in a German orthography test, and Mr.\ Denk, who helped me develop my talents in the final years of high school.

I would like to thank Haye Hinrichsen, under whose supervision I was able to do my first steps as scientist.

I would like to thank Andreas Winter, who introduced me to the field of quantum information and to whom I am greatly indebted for all the opportunities he offered to me.

I would like to thank my supervisor Jens Eisert who has been restlessly supporting me in countless ways.
I am grateful for him creating a near perfect research environment, for getting me involved in so many wonderful projects, for his enthusiasm, encouragement, advice, and trust.

I would like to thank the current and former members of the QMIO group, in particular Martin Kliesch, Arnau Riera, Markus Müller, Leandro Aolita, Earl Campbell, Michael Kastoryano, Mathis Friesdorf, Henrik Wilming, and Albert Werner, for sharing their ideas in hours of discussions, their advice, wisdom, and support.
It is you, who have made the last three and a half years the wonderful, enjoyable, and productive time I can now look back to.

Last but not least, I would like to thank my wonderful partner Manuela for her support and interest in my life and work, for being the open-hearted, outspoken, and inspiring person she is.

This work was supported by the Studienstiftung des deutschen Volkes.

\cleardoublepage

\section{Abstract}
%
This thesis fathoms out the capabilities of the theory of quantum mechanics to explain thermodynamic behavior.
It covers in particular equilibration and thermalization in closed quantum systems, typicality, time scales for equilibration, quantum integrability and its connection to thermalization, decoherence, and a maximum entropy principle.
Together, the presented results form the body of the theory of pure state quantum statistical mechanics.
With almost 300 references, ranging from the groundbreaking works of the early 20th century to the most recent discoveries (up to 2013), this work arguably constitutes the most comprehensive review of the literature on equilibration and thermalization in closed quantum systems.
All results are presented in a unified notation and many are slightly strengthened or generalized.

%This thesis fathoms out the capabilities of the theory of quantum mechanics to explain thermodynamic behavior. It covers in particular equilibration and thermalization in closed quantum systems, typicality, time scales for equilibration, quantum integrability and its connection to thermalization, decoherence, and a maximum entropy principle. Together, the presented results form the body of the theory of pure state quantum statistical mechanics. With almost 300 references, ranging from the groundbreaking works of the early 20th century to the most recent discoveries (up to 2013), this work arguably constitutes the most comprehensive review of the literature on equilibration and thermalization in closed quantum systems. All results are presented in a unified notation and many are slightly strengthened or generalized.

\cleardoublepage

\selectlanguage{ngerman}
\section{Zusammenfassung}
%
Diese Arbeit lotet aus, inwieweit thermodynamisches Verhalten auf der Basis von Quantenmechanik erklärt werden kann.
Behandelt werden insbesondere: Equilibrierung und Thermalisierung in abgeschlossenen Quantensystemen, Typikalität, die Zeitskalen auf denen Equilibrierung stattfindet, Integrabilität in der Quantenmechanik, Dekoherenz und ein Prinzip der maximalen Entropie.
Zusammengenommen bilden die präsentierten Resultate die Theorie der ``pure state quantum statistical mechanics''.
Mit fast 300 Referenzen aus allen Phasen der Entwicklung des Feldes, von den Anfängen im frühen 20. Jahrhundert bis zu den jüngsten Ergebnissen (bis einschließlich 2013), gibt die Arbeit die bisher wohl umfassendste Übersicht zum Thema Equilibrierung und Thermalisierung in geschlossenen Quantensystemen.
Alle Resultate werden in einer vereinheitlichten Notation präsentiert und viele leicht verbessert oder verallgemeinert.

%Diese Arbeit lotet aus, inwieweit thermodynamisches Verhalten auf der Basis von Quantenmechanik erklärt werden kann. Behandelt werden insbesondere: Equilibrierung und Thermalisierung in abgeschlossenen Quantensystemen, Typikalität, die Zeitskalen auf denen Equilibrierung stattfindet, Integrabilität in der Quantenmechanik, Dekoherenz und ein Prinzip der maximalen Entropie. Zusammengenommen bilden die präsentierten Resultate die Theorie der ``pure state quantum statistical mechanics''. Mit fast 300 Referenzen aus allen Phasen der Entwicklung des Feldes, von den Anfängen im frühen 20. Jahrhundert bis zu den jüngsten Ergebnissen (bis einschließlich 2013), gibt die Arbeit die bisher wohl umfassendste Übersicht zum Thema Equilibrierung und Thermalisierung in geschlossenen Quantensystemen. Alle Resultate werden in einer vereinheitlichten Notation präsentiert und viele leicht verbessert oder verallgemeinert.


\cleardoublepage

\section{Eigenständigkeitserklärung}
%
Hiermit bestätige ich, dass ich die vorliegende Arbeit selbstständig und nur mit Hilfe der angegebenen Hilfsmittel angefertigt habe.
Alle Stellen der Arbeit, die wörtlich oder sinngemäß aus Veröffentlichungen oder aus anderweitigen fremden Quellen entnommen wurden, sind als solche kenntlich gemacht.
Ich habe die Arbeit noch nicht in einem früheren Promotionsverfahren eingereicht.

%{\begin{flushright} \vspace{9mm}Vorgelegt am \rule{35mm}{0.2pt} von \rule{60mm}{0.2pt}\\{\footnotesize Christian Gogolin} \end{flushright}}
{\begin{flushright}\rule{60mm}{0.2pt}\\{\footnotesize Christian Gogolin} \end{flushright}}
\cleardoublepage

%%%%List of publications%%%%%%%%%%%%%%%%%%%%%%%%%%%%%%%
% \markboth{List of publications}{List of publications}
% \markright{List of publications}
%\section{List of publications}
\section{Liste der Publikationen des Verfassers}
%
\vspace{1em}
Diese Doktorarbeit basiert teilweise auf den folgenden Artikeln des Autors:
{%\small
\begin{enumerate}
\item \fullcite{Gogolin09}
\item \fullcite{Gogolin10-masterthesis}
\item \fullcite{PhysRevE.81.05-1}
\item \fullcite{1012.1215v1}
\item \fullcite{PhysRevLett.10-6}
\item \fullcite{Hinrichsen11}
\item \fullcite{1105.3986v1}
\item \fullcite{Riera2012}
\item \fullcite{1111.3965v1}
\item \fullcite{Kliesch2013}
\item \fullcite{Gogolin2013}
\item \fullcite{Kliesch2013a}
\item \fullcite{Steinigeweg2013}
\end{enumerate}

\newpage
\newcommand{\cvitem}[2]{\item[\hfill #1:] #2}
\newcommand{\cvdoubleitem}[4]{\cvitem{#1}{#2}\cvitem{#3}{#4}}
\newcommand{\cvitemleftmargin}{\hspace{3cm}}
\section{Lebenslauf}
%
\vspace*{\fill}
\begin{center}
  Der Lebenslauf ist in der Online-Version aus Gründen des Datenschutzes nicht enthalten.
\end{center}
\begin{center}
  For reasons of data protection, the curriculum vitae is not included in the online version.
\end{center}
\vspace*{\fill}
\newpage
\vspace*{\fill}
\begin{center}
  Der Lebenslauf ist in der Online-Version aus Gründen des Datenschutzes nicht enthalten.
\end{center}
\begin{center}
  For reasons of data protection, the curriculum vitae is not included in the online version.
\end{center}
\vspace*{\fill}
\newpage
\vspace*{\fill}
\begin{center}
  Der Lebenslauf ist in der Online-Version aus Gründen des Datenschutzes nicht enthalten.
\end{center}
\begin{center}
  For reasons of data protection, the curriculum vitae is not included in the online version.
\end{center}
\vspace*{\fill}
\newpage
\vspace*{\fill}
\begin{center}
  Der Lebenslauf ist in der Online-Version aus Gründen des Datenschutzes nicht enthalten.
\end{center}
\begin{center}
  For reasons of data protection, the curriculum vitae is not included in the online version.
\end{center}
\vspace*{\fill}

\end{document}

\paragraph{Persönliche Daten}
\begin{labeling}{\cvitemleftmargin}
  \cvitem{Vorname/Name}{Christian Gogolin}
  \cvitem{Geburtsdatum}{16.08.1985 (Karlsruhe)}
%  \cvitem{Staatsangehörigkeit}{Deutsch}
\end{labeling}

\paragraph{Ausbildung}
\begin{labeling}{\cvitemleftmargin}
  % \cvitem{1992--1993}{Henry Dunant Grundschule Sossenheim, Frankfurt a.M.}
  % \cvitem{1993--1996}{Montessorischule Würzburg}
  % \cvitem{1992--1996}{Grundschule in Frankfurt am Main und Würzburg}
  \cvitem{1996--2005}{Mozart- und Schönborngymnasium Würzburg}%,\\Abschluss mit dem Abitur mit Gesamtnote 1.7}
  % \cvitem{10.2005}{Beginn des Studiums der Physik an der \emph{Julius--Maximilians--Universität Würzburg} im Studiengang FOKUS Physik des \emph{Elite Netzwerks Bayern}}
  \cvitem{10.2005}{Aufnahme in den zulassungsbegrenzten Studiengang FOKUS~Physik des Elitenetzwerks Bayern an der \emph{Julius-Maximilians-Universität Würzburg}}
  % \cvitem{08.2007}{Erwerb des Vordiploms mit Gesamtnote sehr gut}
%  \cvitem{04.2008}{Aufnahme in die \emph{Studienstiftung des deutschen Volkes}}
  \cvitem{30.04.2008}{Abschluss der Bachelorarbeit mit Note sehr gut}
  \cvitem{12.04.2010}{Master of Science with Honors, Gesamtnote 1.0}
  \cvitem{05.2010--09.2010}{Forschungsaufenthalt an der \emph{University of Bristol}}
  \cvitem{seit 01.10.2010}{Doktorarbeit bei Jens Eisert an der \emph{Universität Potsdam} und \emph{Freien Universität Berlin}.}
\end{labeling}

\paragraph{Preise und Stipendien}
\begin{labeling}{\cvitemleftmargin}
  \cvitem{04.2008--04.2010}{Stipendium der \emph{Studienstiftung des deutschen Volkes}}
  \cvitem{05.2010--09.2010}{Stipendium für ein Auslandssemester an der \emph{University of Bristol} der \emph{Studienstiftung des deutschen Volkes}}
  \cvitem{03.2011}{Publikationspreis des Leibniz-Kollegs Potsdam (Preisgeld 2.500\,\euro)}
  \cvitem{10.2011--10.2013}{Promotionsstipendium der \emph{Studienstiftung des deutschen Volkes}}
\end{labeling}

\newpage
\paragraph{Impact}
\begin{labeling}{\cvitemleftmargin}
  \cvitem{Zitationen}{
    (laut Google Scholar vom 01.04.2014)\\
    \begin{tabular*}{6cm}{@{\extracolsep{\fill}}l l l l}
      gesamt: & 215 & meistzitierte Arbeit: & 91  \\
      h-index: & 7 & i10-index: & 6 \\
    \end{tabular*}}
  \cvitem{Presse}{
    Der Artikel \emph{A dissipative quantum Church Turing theorem} wurde in einem \href{http://physics.aps.org/articles/v4/72}{\emph{Viewpoint in Physics}} diskutiert.\\
    Der Artikel \emph{Quantum measurement occurrence is undecidable} wurde auf \href{http://phys.org/news/2012-07-classical-problem-undecidable-quantum.html}{\emph{Phys.org}} und in der online Zeitschrift \href{http://ko.com.ua/sushhestvuyut_klassicheskie_zadachi_nereshaemye_kvantovymi_kompyuterami_64736}{\foreignlanguage{russian}{Компьютерное Обозрение}} aufgegriffen.}
\end{labeling}

\paragraph{Forschungsaufenthalte und ausgewählte Vorträge}
\begin{labeling}{\cvitemleftmargin}
\cvitem{10.2008-11.2008}{Forschungsaufenthalt am \emph{Centre for Quantum Technologies}, Singapur und Besuch des Workshops \emph{Quantum Algorithms and Complexity Theory}}
\cvitem{02.2009}{Vortrag am \emph{Atomistic Simulation Centre} der \emph{Queen's University Belfast}, UK}
\cvitem{08-09.2009}{Sommerakademie \emph{Empirische Konjunkturanalyse} der \emph{Studienstiftung des deutschen Volkes} in Rot an der Rot}
\cvitem{10.2009}{Workshop des \emph{QCCC} in Bad Tölz (Poster)}
\cvitem{11.2009}{Vortrag an der \emph{Universität Potsdam}}
\cvitem{11.2009--12.2009}{Besuch an der \emph{University of Bristol} und der \emph{Queen's University Belfast}, UK}
\cvitem{01.2010}{\emph{QIP 2010} Konferenz in Zürich, Schweiz (Poster)}
\cvitem{05.2010--09.2010}{Forschungsaufenthalt an der \emph{University of Bristol}, UK}
\cvitem{06.2010}{Workshop \emph{Fundamentals of Physics and Information} in Zürich, Schweiz (Poster)}
\cvitem{07.2010}{Vortrag am \emph{University College London}, UK}
\cvitem{07.2010}{Wissenschaftlicher Besuch mit Vortrag an der \emph{Boston University}, USA}
\cvitem{09.2010}{Symposium \emph{Quantum Thermodynamics} in Stuttgart}
\cvitem{01.2011}{Vortrag auf der \emph{QIP 2011} in Singapur}
\cvitem{01.2011}{Besuch am \emph{Centre for Quantum Technologies} in Singapur}
\cvitem{02.2011}{Vortrag an der \emph{Leibniz Universität Hannover}}
\cvitem{03.2011}{Vortrag auf der DPG Frühjahrstagung in Dresden}
\cvitem{05.2011}{Workshop \emph{Conceptual Foundations and Foils for Quantum Information Processing} (Poster) und Besuch am \emph{Perimeter Institute}, \mbox{Waterloo}, Kanada}
\cvitem{06.2011}{Workshop \emph{Quantum Information} im \emph{Centro de Ciencias de Benasque}, Spanien}
\cvitem{08.2011}{Eingeladener Vortrag auf dem \emph{Workshop for Quantum Information and Foundations of Thermodynamics} in Zürich, Schweiz}
\cvitem{10.2011}{Vortrag auf dem Workshop \emph{Many body quantum dynamics of closed systems} in Barcelona, Spanien}
\cvitem{10.2011}{Eingeladener Vortrag auf dem QCCC Workshop in Bernried}
\cvitem{03.2012}{Vorträge auf den DPG Tagungen in Stuttgart und Göttigen}
\cvitem{06.2012}{Gastvorlesung \emph{Unentscheidbarkeit in der Quantenmechanik} an der \emph{Albert-Ludwigs-Universität Freiburg}}
\cvitem{07.2012}{Teilnahme am \emph{Lindau Nobel Laureat Meeting}}
\cvitem{08.2012}{Vortrag auf dem Workshop \emph{Open many body systems, quantum computing and correlations, classical and quantum} am \emph{CICC}, Cuernavaca, Mexiko}
\cvitem{04.2013}{Vortrag auf dem Workshop \emph{Equilibration and thermalization in quantum systems} am \emph{Wallenberg Center}, Stellenbosch, Südafrika}
\cvitem{12.2013}{Vortrag an der \emph{ETH Zürich}, Schweiz}
\cvitem{01.2014}{Eingeladener Vortrag auf der \emph{COST conference on quantum thermodynamics} in Potsdam Griebnitzsee}
\cvitem{01.2010}{\emph{QIP 2014} Konferenz in Barcelona, Spanien (Poster)}
\cvitem{02.2014}{Vortrag am \emph{Institute of Photonic Sciences (ICFO)} in Barcelona, Spanien}
\cvitem{03.2014}{Vortrag auf der DPG Tagung in Berlin}
\end{labeling}

\end{document}

%%% Local Variables:
%%% mode: latex
%%% TeX-master: t
%%% End:
